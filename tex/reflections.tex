\section{Reflections}
The purpose of this section is to give a reflection upon the technologies I have used throughout this project.
I will outline my feelings on each of these technologies in the following sections.

\tocless\subsection{TensorFlow}
There are many alternative machine learning libraries used for the development of CNNs such as Caffe, Gluon and MXNet.
I used TensorFlow due to Google's reputation and prevalence on the Internet.
While TensorFlow has a large learning curve, I believe this is due to the complexity of CNNs and machine learning in general rather than the library.
I used Python in conjunction with TensorFlow in this project.

The online support for TensorFlow is superb.
From resources such as Stack Overflow to Google's forums for questions relating to the topic.
While I don't have experience with the other libraries, I would definitely recommend TensorFlow to beginners in machine learning as the documentation and support is outstanding.

\tocless\subsection{Python}
Python is a very simple and generic programming language supporting the paradigms of object oriented, functional, imperative and procedural.
I enjoyed working with the language for the most part.
It seems like the language is written to make things as easy as possible for the developer with many helpful in-built functions.

Python is a dynamically typed language meaning that type checking is carried out at runtime as opposed to statically typed when it is carried out at compile time.
Personally, I prefer statically typed languages.
This could be partly due to my experience in Java (which is statically typed) by I find when dealing with code that has been written by another developer, something as simple as knowing the types of function parameters is taken for granted.

\tocless\subsection{Flask}
Flask is a web framework that integrates with Python.
I cannot stress how nice it is to use.
It has a very small learning curve and its makes prototyping web services very easy.

\tocless\subsection{AWS}
I have had previous experience with Amazon Web Services on a summer internship and while working on college projects.
Due to this past experience, there was no hesitation in choosing a cloud provider for my prototype application.
A simple EC2 instance is very easy to setup using AWS.

\tocless\subsection{Android Studio}
Android Studio is the IDE used for the development of Android applications.
It is made by JetBrains who also developed the IDE InteliJ so there are many similarities there.
For the most part it is like InteliJ which is very nice to use for Java development.
The UI design in Android Studio is also very good as most of the UI development can be done in a drag and drop fashion which is very helpful for prototyping.

\tocless\subsection{Java}
I have been using Java for four years and it is a very nice language to use especially with the introduction of streams and other functional paradigms addition.

\tocless\subsection{Latex}
I used Latex for the report for this FYP due to the suggestion from my supervisor.
Overall, I find it a very good resource to use.
The fact that you focus more on content than presentation is very helpful for such a large report and the use of version control is very beneficial.
The positioning of images and tables is sometimes a bit frustrating, but I found Latex the most useful when I wanted to move sections around in the report and for references.