Tensorflow is a deep learning api for various machine learning paradigms. I will
be using tensorflow to create neural networks.
Tensorflow has two utilisations, through a Graphics Processing Unit (GPU) and
also through a Central Processing Unit (CPU).
GPU computation is recommended for CNN training. 

\subsubsection*{Central Processing Unit Computation}
It is quite easy to get tensorflow up and running if you are only using a CPU to
train. I have successfully installed tensorflow cpu on both Windows and Ubuntu.
For Windows you can download and install using the tensorflow website and on ubuntu you can
using apt-get. Once installed, tensorflow can be imported into any python shell
or script for use. Tensorflow can also be used in C. There will be various
python implementations of neural networks in Chapter 3.

\subsubsection*{Graphics Processing Unit Computation}
For use with a GPU, the set up for tensorflow is a bit more complicated. Firtsly
you use check that the GPU in your machine is compatible for CUDA 8.0 using the
NVIDIA website. If your GPU is compatible, you must install CUDA after signing
up as an NVIDIA developer. CUDA 8.0 is compatible with tensorflow. You alos need
to install cudnn6. The NVIDIA website contains tutorials to install these. Once
these are installed, download and install tensorflow-gpu. This can imported into
python similar to CPU computation.
