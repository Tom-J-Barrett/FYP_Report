\chapter{Discussion and Conclusion}
\section{Summary}
The purpose of this final year project was to explore the use of deep learning for nutritional assessment.
This exploration was carried firstly by conducting a comprehensive literature review of research already undertaken in the area of food image recognition and also some research into the methods used for object detection with CNNs.
Once this survey of literature was completed, a method from these papers was selected to be replicated.
The work done by \parencite{yanaiFood} yielded promising results in retraining models using transfer learning and because of this, their work was to be replicated.

Some deviations were made from \parencite{yanaiFood} in regard to the dataset used.
The Food-101 dataset was used for this FYP, in conjunction with some additional classes from ImageNet \parencite{imagenet}.
In contrast, \parencite{yanaiFood} used the UECFOOD100 dataset \parencite{uecFood}.
The change in dataset was used due to the availability of a larger dataset in Food-101 which has 1,000 images per class.

The Food-101+ dataset was created by adding seven additional classes to the Food-101 dataset.
This Food-101+ dataset was used to train a model with the final Top-1 accuracy of 66.6\% and a Top-5 accuracy of  85.96\%.
This model was created by retraining the final layer of the Inception-V3 \parencite{rethinkingInception} model using transfer learning and TensorFlow.
A subset of the Food-101+ dataset was also trained using 13 classes in the same way.
This model achieved a Top-1 accuracy of 92.6\% and a Top-5 accuracy of 100\%.

Once the models were trained, they were analysed under various topics such as
dealing with composite images using a sliding window approach, the effect of colour on classification of food types and the effect of image quality on the classification of food types.

After the above empirical studies had been carried out, a lightweight prototype application was created to demonstrate the use of the TensorFlow model for food image classification.
This prototype application was created for a smart phone running on Android.
This application is called NutriLog.
NutriLog is able to send an image of food (acquired either from the camera or gallery of the users phone) to a server to be analysed.
The server that the image is sent to is running a Python Flask application that runs the image through a TensorFlow model and returns a prediction to NutriLog.
NutriLog then collects nutritional information on the prediction using the Nutritionix API and logs information for user metrics.
The user can then view their daily, weekly or monthly calorie intake.
NutriLog could be very useful for those who wish to keep track of their calorie intake.

\section{Reflections}
The purpose of this section is to give a reflection upon the technologies I have used throughout this project.
I will outline my feelings on each of these technologies in the following sections.

\tocless\subsection{TensorFlow}
There are many alternative machine learning libraries used for the development of CNNs such as Caffe, Gluon and MXNet.
I used TensorFlow due to Google's reputation and prevalence on the Internet.
While TensorFlow has a large learning curve, I believe this is due to the complexity of CNNs and machine learning in general rather than the library.
I used Python in conjunction with TensorFlow in this project.

The online support for TensorFlow is superb.
From resources such as Stack Overflow to Google's forums for questions relating to the topic.
While I don't have experience with the other libraries, I would definitely recommend TensorFlow to beginners in machine learning as the documentation and support is outstanding.

\tocless\subsection{Python}
Python is a very simple and generic programming language supporting the paradigms of object oriented, functional, imperative and procedural.
I enjoyed working with the language for the most part.
It seems like the language is written to make things as easy as possible for the developer with many helpful in-built functions.

Python is a dynamically typed language meaning that type checking is carried out at runtime as opposed to statically typed when it is carried out at compile time.
Personally, I prefer statically typed languages.
This could be partly due to my experience in Java (which is statically typed) by I find when dealing with code that has been written by another developer, something as simple as knowing the types of function parameters is taken for granted.

\tocless\subsection{Flask}
Flask is a web framework that integrates with Python.
I cannot stress how nice it is to use.
It has a very small learning curve and its makes prototyping web services very easy.

\tocless\subsection{AWS}
I have had previous experience with Amazon Web Services on a summer internship and while working on college projects.
Due to this past experience, there was no hesitation in choosing a cloud provider for my prototype application.
A simple EC2 instance is very easy to setup using AWS.

\tocless\subsection{Android Studio}
Android Studio is the IDE used for the development of Android applications.
It is made by JetBrains who also developed the IDE InteliJ so there are many similarities there.
For the most part it is like InteliJ which is very nice to use for Java development.
The UI design in Android Studio is also very good as most of the UI development can be done in a drag and drop fashion which is very helpful for prototyping.

\tocless\subsection{Java}
I have been using Java for four years and it is a very nice language to use especially with the introduction of streams and other functional paradigms addition.

\tocless\subsection{Latex}
I used Latex for the report for this FYP due to the suggestion from my supervisor.
Overall, I find it a very good resource to use.
The fact that you focus more on content than presentation is very helpful for such a large report and the use of version control is very beneficial.
The positioning of images and tables is sometimes a bit frustrating, but I found Latex the most useful when I wanted to move sections around in the report and for references.

\section{Future Work}
While retraining the Inception-V3 model yielded promising results for food image classification there are some problems associated.
The model trained, does not deal well with composite image and while a sliding window could be used to classify different portions of the image, it is not a viable method as it is very time intensive.
Another issue is that there is a clear correlation between the number of classes and the accuracy of the model.
This is clearly seen in the 108 class model yielding a Top-1 accuracy of 66.6\% while the 13 class model yielded a Top-1 accuracy of 92.6\%.
Possible solutions to these issues are outlined below.

\tocless\subsection{Resolving Issues with Composite Images}
The model trained for this project relies on the one-shot approach where an image is passed straight to the model preferable containing only one food item.
A possible solution to this problem is to segment the image before using the classifier.
There has been extensive research into the segmentation of food images carried out, but it was out of scope for this FYP.
Colour and texture are the most common features to segment food images by.
If it was possible to separate each food item in a food image and feed each of these individual items through the classifier then the problem of composite image would be resolved.

Another option would be to take an object detection approach to food image classification.
In Chapter 2 there was a survey of literature carried about
using CNNs for object detection as in \parencite{maskRcnn}.
If the objects (food items) could be detected in this way and then passed through the classifier then dealing with composite images would not be such a challenge.

While both of these methods are promising approaches to solving this issue, the object detection seems to be a good approach. This is due to the success of \parencite{maskRcnn}, although the method has not been used for food images before.

\tocless\subsection{Resolving Issues with Class Size}
As stated above, the accuracy of Inception-V3 model declines the more classes are in the training set.
In order to solve this problem changes to the architecture of the model or
changes to the dataset may have to be introduced.
There are many CNN model architectures that could be viable options for food image recognition.
\parencite{nutrinet} created a model architecture aimed specifically towards food image recognition called Nutrinet and experiments on this architecture could be carried out.
The dataset used for training could also be extended with more images per class to increase the accuracy.
