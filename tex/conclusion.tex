\chapter{Discussion and Conclusion}
The purpose of this final year project was to explore the use of deep learning for nutritional assessment.
This exploration was carried firstly by conducting a comprehensive literature review of research already undertaken in the are of food image recognition and also some research into the methods used for object detection with CNNs.
Once this survey of literature was completed, a method from these papers was selected to be replicated.
The work done by \parencite{yanaiFood} yielded promising results in retraining models using transfer learning and because of this, their work was to be replicated.

Some deviations were made from \parencite{yanaiFood} in regards to the dataset used.
The Food-101 dataset was used for this FYP, in conjunction with some additional classes from Imagenet \parencite{imagenet}.
In contrast, \parencite{yanaiFood} used the UECFOOD100 dataset \parencite{uecFood}.
The change in dataset was used due to the availability of a larger dataset in Food-101 which has 1000 images per class.

The Food-101+ dataset was created by adding seven additional classes to the Food-101 dataset.
This Food-101+ dataset was used to train a model with the final Top-1 accuracy of 66.6\% and a Top-5 accuracy of  85.96\%.
This model was created by retraining the final layer of the Inception-V3 \parencite{rethinkingInception} model using transfer learning and Tensorflow.
A subset of the Food-101+ dataset was also trained using 13 classes in the same way.
This model achieved a Top-1 accuracy of 92.6\% and a Top-5 accuracy of 100\%.

Once the models were trained, they were analysed under various topics such as
dealing with composite images using a sliding window approach, the effect of colour on classification of food types and also the effect of image quality on the classification of food types.

After the above empirical studies had been carried out, a lightweight prototype application was created to demonstrate the use of the Tensorflow model for food image classification.
This prototype application was created for a smart phone running on Android.
This application is called NutriLog.
NutriLog is able to send an image of food (acquired either from the camera or gallery of the users phone) to a server to be analysed.
The server that the image is send to is running a Python Flask application that runs the image through a Tensorflow model and returns a prediction to NutriLog.
NutriLog then collects nutritional information on the prediction using the Nutritionix API and logs information for user metrics.
The user can then view their daily, weekly or monthly calorie intake.
NutriLog could be very useful for those who wish to keep track of their calorie intake.

\section{Future Work}
While retraining the Inception-V3 model yielded promising results for food image classification there are some problems associated.
The model trained, does not deal well with composite image and while a sliding window could be used to classify different portions of the image, it is not a viable method as it is very time intensive.
Another issue is that there is a clear correlation between the amount of classes and the accuracy of the model.
This is clearly seen in the 108 class model yielding a Top-1 accuracy of 66.6\% while the 13 class model yielded a Top-1 accuracy of 92.6\%.
Possible solutions to these issues are outlined below.

\subsection*{Resolving Issues with Composite Images}
The model trained for this project relies on the one shot approach where an image is passed straight to the model preferable containing only one food item.
A possible solution to this problem is to segment the image before using the classifier.
There has been extensive research into the segmentation of food images carried out but it was out of scope for this FYP.
Colour and texture are the most common features to segment food images by.
If it was possible to separate each food item in a food image and feed each of these individual items through the classifier then the problem of composite image would be resolved.

Another option would be to take an object detection approach to food image classification.
In Chapter 2 there was a survey of literature carried out on the subject of
using CNNs for object detection as in \parencite{maskRcnn}.
If the objects (food items) could be detected in this way and then passed through the classifier then dealing with composite images would not be such a challenge.

While both of these methods are promising to solve this issue, the object detection seems to be the most interesting due to the success of \parencite{maskRcnn} and the fact that the approach has not been used for food images before.

\subsection*{Resolving Issues with Class Size}
As stated above, the accuracy of Inception-V3 model declines the more classes are in the training set.
In order to solve this problem changes to the architecture of the model or
changes to the dataset may have to be introduced.
There are many CNN model architectures that could be viable options for food image recognition.
\parencite{nutrinet} created a model architecture aimed specifically towards food image recognition called Nutrinet and experiments on this architecture could be carried out.
The dataset used for training could also be extended with more images per class to increase the accuracy.
