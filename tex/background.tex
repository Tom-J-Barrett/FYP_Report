\chapter{Background}
\label{background}

\section{Introduction to Machine Learning}
In \textcite{ML}, Machine Learning is defined as "the question of how to
construct computer programs that auotmatically improve with experience".
MAchine Learning has blossomed in recent years with applications across multiple
domains using vastly different paradigms and technologies.

There are many ways in which Machine Learning can be used in the modern world,
many of which are being utilised to great affect.
Some of these applications, are image recognition, natural language processing,
medical diagnosis and many more.
There may be fear that Machine Learning will start to take away many jobs from
humans but this may not be the case. Imagine a doctor, having to diagnose a
patient, a machine can offer suggestions based on very large datasets of what
the diagnosis is. This is not to say that a Machine would be
perscribing patients, but merely act as an assisstant to the doctor.

Machine Ethics is a large problem tat comes hand in hand with Machine Learning.
There is a very important question of who takes the blame when things go wrong,
that is why I think it is important that we only use Machine Learning to advise
and not to determine but this can be very difficult in a world where, for
example, an autonomous car has to decide between crashing into a vechile beside
them with an unknown number of people inside or the two children playing in the
street.

One of the most exciting avenues in Machine Learning, in my opinion, is Computer
Vision. Computer Vision can be used in many areas to improve our lives. As
mentioned earlier, autonomous cars are only possible when a machine can
determine what objects are around it. Computer Vision can allow a machine to
recognise skin diseases in an image. The applications are nearly limitless and
that is without taking into account other uses.

\section{Machine Learning Paradigms}
There are many Machine Learning paradigms, all of which I will
discuss briefly below, but the main area of my focus for this project is in
Artificial Neural Networks (ANN). This is because I have researched extensively into Convolutional Neural Networks which are based on ANN's.
\subsection{Artificial Neural Networks}
An Artificial Neural Network is a bio-inspired system that is used to model the human brain in how it learns from experience.
The ANN uses this model to build a very complex web of connected units called
artificial neurons.
An ANN has a set of inputs that take in a value, sometimes from network outputs
and produce a single result or classification.
While an ANN is bio-inspired from the human brain, there are many elements of
the brain that are not present in ANN and many new elements in ANN that are not
modelled from the human brain.

Before I can talk about Convolution Neural Networks which are vital the image
processing, I will have to talk about the perceptron learning algorithm, the multi
layer percepton, and backpropogation.

\subsubsection{Percetron}
\subsubsection{Multi Layered Perceptron and Backpropogation}
Multi Layer Perceptrons (MLP) are a class of feed forward Artificial Neural Networks
\subsubsection{Convolutional Neural Networks}
Convolutional Neural Networks (CNN's) are esentially a Multi Layered Percetron with a
special stucture. CNN's have one major difference from a MLP, they have extra
laer of convolution and pooling.

\begin{figure}
    \includegraphics{XtoRec}
    \caption{Image to Classify}
    \label{fig:XtoRec}
\end{figure}
\begin{figure}
     \includegraphics{XtoComp}
     \caption{Image to Compare}
     \label{fig:XtoComp}
\end{figure}
Figure \ref{fig:XtoRec} show an image that we want to compare against
Figure \ref{fig:XtoComp}.
For humans, it is quite easy to determine that these images are very similar but
for a computer this task is suprisingly difficult.

\begin{figure}
    \includegraphics{ImageFeature}
    \caption{Image Feature to Search}
    \label{fig:feature}
\end{figure}
\begin{figure}
    \includegraphics{ConvImage}
    \caption{Convoluted Image}
    \label{fig:convoluted}
\end{figure}
So what a CNN does, to combat this problem, is to take a small feature from
Figure \ref{fig:XtoRec} and compare it to a subsection of Figure \ref{fig:XtoComp}.
The CNN multiplies the feature and a section of Figure \ref{fig:XtoComp}, adds
up the results and divides by 9. This then gives a decimal value of how likely
it is that the feature is in the part of the image, as seen in Figure
\ref{fig:convoluted}.
This is called filtering. The Convolutional layer is composed of carrying out
this filtering for every single possible location in Figure \ref{fig:XtoComp}.

\begin{figure}
    \includegraphics{PooledImage}
    \caption{Pooled Image}
    \label{fig:pooled}
\end{figure}
Next is the Pooling Layer, what this layer does, is it takes the convoluted
layer output, you can use Figure \ref{fig:convoluted} as reference, and from a
user defined size ie. 2x2, gets either the highest decimal value (Max pooling)
or the average value (Mean pooling) and records that as the new value for the
section. This is then applied to the entire image. As we can see in Figure
\ref{fig:pooled} we know have a much smaller image stack in which to classify,
thus making the computation easier.

In between the Convolution and Pooling layer, there is sometimes a Normalization
layer. This Normalization layer creates Rectified Linear Units (RLU's). In other
words, if we take Figure \ref{fig:convoluted}, it changes all minus values to
zero.

\subsubsection{Fully Convolutional Networks}

\subsection{Other Paradigms}
\subsubsection{Decision Trees}
\subsubsection{Meta/Ensemble Classifiers}
\subsubsection{Logistic Regression}
\subsubsection{Support Vector Machine}
\subsubsection{Regression Analysis}
\subsubsection{Unsupervised Learning}
\subsubsection{Reinforcement Learning}

\section{Overview of Machine Vision Approaches to Identification and Classification}
\subsection{Identification}
\subsection{Classification}

\section{API's}

\section{Evaluating the Output}


