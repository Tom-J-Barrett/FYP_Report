\subsection*{Python Flask}

\subsection*{Implementation}
\subsubsection*{Backend service code}
\begin{lstlisting}[style=Python]
import base64
import label_image
import time

app = Flask(_name_)

@app.route('/classifyImage/', methods=['POST'])
def classify():
    imgdata = base64.b64decode(request.form.get('image'))
    filename = 'images/' + str(time.strftime("%Y%m%d-%H%M%S")) + '.jpg'
    with open(filename, 'wb') as f:
        f.write(imgdata)

    results = label_image.runModel(filename)
    predictions = results[0]

    result_String = ""
    for i in predictions :
        result_String += str(i) + ","
    return result_String

if _name_ == '_main_':
    app.run(host='0.0.0.0', threaded=True)
\end{lstlisting}

\subsubsection*{Method called from 'label\_image.py'}
\begin{lstlisting}[style=Python]
def runModel(file_name):
  model_file = \
    "models/output_graph13.pb"
  label_file = "labels/output_labels13.txt"
  input_height = 299
  input_width = 299
  input_mean = 128
  input_std = 128
  input_layer = "Mul"
  output_layer = "final_result"

  graph = load_graph(model_file)
  t = read_tensor_from_image_file(file_name,
                                  input_height=input_height,
                                  input_width=input_width,
                                  input_mean=input_mean,
                                  input_std=input_std)

  input_name = "import/" + input_layer
  output_name = "import/" + output_layer
  input_operation = graph.get_operation_by_name(input_name)
  output_operation = graph.get_operation_by_name(output_name)

  with tf.Session(graph=graph) as sess:
    results = sess.run(output_operation.outputs[0],
                      {input_operation.outputs[0]: t})
  results = np.squeeze(results)

  top_k = results.argsort()[-5:][::-1]
  labels = load_labels(label_file)

  setIndex = False

  top5_results = [None] * 5
  index = 0
  for i in top_k:
    top5_results[index] = labels[i]
    index += 1

  final_results = [top5_results, results]
  return final_results
\end{lstlisting}

\subsection*{Deployment}
In order to deploy the Flask application to an AWS EC2 instance the following steps had to be followed:

% \begin{itemize}
%     \item{Upload the output_graph.pb file (Tensorflow model), the output_labels.txt file and the source code to the server using SFTP}
%     \item{SSH into the server}
%     \item{Create a folder called 'images' in the home directory to store uploads images}
%     \item{Place model and label files into folders called 'models' and 'labels' respectively}
%     \item{Ensure Tensorflow, NOHUP (Ignores the hangup signal in Linux) Python and Flask are installed on the server}
%     \item{Ensure no processes are running on port 5000 using the command 'lsof -i :5000'}
%     \item{Run the server.py code using the command 'nohup python server.py &'}
% \end{itemize}