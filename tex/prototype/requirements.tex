Lightweight requirements elicitation was carried out for this FYP.
Since prototyping itself is an elicitation technique \parencite{reqs}, the requirements for this prototype application did not need to be extensive.
Myself and my supervisor acted as stakeholders for this project as users of the system.
Two methods of elicitation were used for this project sourced from \parencite{reqs}.
These elicitation techniques were data gathering from an existing system and brain storming.

In a previous FYP, a mobile application was developed for nutritional assessment.
The use of this application provided the opportunity to elicit requirements based on data gathered from this system. Brainstorming also took place between the stakeholders for requirements elicitation.

\tocless\subsection{Functional Requirements}
The functional requirements for this application can be seen in Table \ref{requirements}.

\begin{longtable}{|p{.75cm}|p{3.5cm}|p{6cm}|p{2.5cm}|}
\hline
\textbf{ID} & \textbf{Title}                                              & \textbf{Description}                                                                                    & Dependencies                                   \\ \hline
R1          & Open application from Android phone                         & A user can open the application from a list of applications on their Android phone.                     &                                                \\ \hline
R2          & Choose to take a photo                                      & A user can choose to take a photo of a food item.                                                       & R1                                             \\ \hline
R3          & Take a photo                                                & A user can take a photo using the application.                                                          & R1, R2                                         \\ \hline
R4          & Confirm the photo                                           & A user can confirm the photo captured is sufficient.                                                    & R1, R2, R3                                     \\ \hline
R5          & Retry photo capture                                         & A user can retry the capturing of an image.                                                             & R1, R2, R3                                     \\ \hline
R6          & Send the image to be classified                             & A user can send an image of a food item to be classified.                                               & R1, R2, R3, R4                                 \\ \hline
R7          & Cancel sending an image                                     & A user can cancel sending and image.                                                                    & R1, R2, R3, R4                                 \\ \hline
R8          & Classify the food image                                     & The system can classify an image as a food type.                                                        & R1, R2, R3, R4, R6                             \\ \hline
R9          & View the food image classification                          & A user can view the classification made of the image.                                                   & R1, R2, R3, R4, R6, R8                         \\ \hline
R10         & View the calorie information of the food classification     & A user can view the calorie information of the classified food type.                                    & R1, R2, R3, R4, R6, R8                         \\ \hline
R11         & Change the food classification data                         & A user can change the food classification data if they are unhappy with it.                             & R1, R2, R3, R4, R6, R8, R9                     \\ \hline
R12         & Change the calorie information                              & A user can change the calorie information if they are unhappy with it.                                  & R1, R2, R3, R4, R6, R8, 10                     \\ \hline
R13         & Submit the food classification and calorie data for logging & A user can submit the data calculated such as classification data and calorie information to be logged. & R1, R2, R3, R4, R6, R8, R9, R10                \\ \hline
R14         & Save food log data                                          & The system can save information for logging.                                                            & R1, R2, R3, R4, R6, R8, R9, R10, R13           \\ \hline
R15         & Choose to classify an image from the phone's storage        & A user can choose to select an image from the phone's storage to be classified.                         & R1                                             \\ \hline
R16         & Choose an image from phone's storage                        & A user can choose an image from the phone's storage to classify.                                        & R1, R14                                        \\ \hline
R17         & Choose to view food logs                                    & A user can choose to view the food logs saved on their device.                                          & R1                                             \\ \hline
R18         & View food logs by day                                       & A user can view the food logs saved on the device by day.                                               & R1, R2, R3, R4, R6, R8, R9, R10, R13, R17      \\ \hline
R19         & View food logs by week                                      & A user can view the food logs saved on the device by week.                                              & R1, R2, R3, R4, R6, R8, R9, R10, R13, R17      \\ \hline
R20         & View food logs by month                                     & A user can view the food logs saved on the device by month.                                             & R1, R2, R3, R4, R6, R8, R9, R10, R13, R17      \\ \hline
R21         & View total calories recorded per day                        & A user can view the total calorie count recorded per day.                                               & R1, R2, R3, R4, R6, R8, R9, R10, R13, R17      \\ \hline
R22         & View total calories recorded per week                       & A user can view the total calorie count recorded per week.                                              & R1, R2, R3, R4, R6, R8, R9, R10, R13, R17      \\ \hline
R23         & View total calories recorded per month                      & A user can view the total calorie count recorded per month.                                             & R1, R2, R3, R4, R6, R8, R9, R10, R13, R17      \\ \hline
R24         & Delete a food log                                           & A user can delete an individual food log.                                                               & R1, R2, R3, R4, R6, R8, R9, R10, R13, R17, R18 \\ \hline
\caption{Functional Requirements}
\label{requirements}
\end{longtable}

\tocless\subsection{Non-Functional Requirements}
The nature of a prototype application does not call for non-functional requirements.
In spite of this some non-functional requirements were kept in mind during the development of this application such as:
\begin{itemize}
	\item{Performance - for example, how long it takes for a image to be classified.}
	\item{Extensibility - is it easy to extend the functionality of the system.}
	\item{Maintainability - how easy it is to maintain the system.}
	\item{Usability - how easy it is to use the system.}
\end{itemize}

\clearpage