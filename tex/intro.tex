\chapter{Introduction}
\label{intro}

\section{Summary}
This project explores the use of indentification and classification of food images for use in a calorie measurement android application.
Food calorie consumption is a huge problem in the modern world.
Over 25\% of the population in Ireland is obese and this figure is likely to rise over the coming years.
A mobile application that could help keep track of a user's calorie intake by taking pictures of their meals would be a great help.
The area of Machine Vision is a very difficult topic to address as it is a very hard task for computers to undertake.
We, as humans, take vision for granted as we can soon see, from the study of Machine Vision, that there are many difficult steps that have to be made for full identification and classification of an image.

Brief History of Machine Vision.....

When looking into calorie measurement using an image, there are three questions that have to be answered:
\begin{itemize}
\item{Where are the Regions of Interest (ROI) in this food image?}
\item{What food types are in these ROI's?}
\item{What is the portain size of each food type?}
\end{itemize}
In this project, my main focus will be on the first question, 'Where are the Regions of Interest (ROI) in this food image?'.

Many researchers in various machine vision labs have attempted to solve this problem using different methodolgies.
There has been promising results from some papers but these are mostly under highly constrained circumstances.
When mixed foods are introduced to the problem, many of the methods fail.
Convolution Neural Networks (CNN) have had very promising results in the field of image classification in the recent years but to get to the classification step, image segmentation is first needed, otherwise known as image identification.

I have researched many different methods of image segmentation but it seems that CNN's have had the best results for multiple objects in one image and I hope to apply these results for many foods in an image.

Summary of application....

\section{Objectives}

\section{Methodology}

\section{Overview of Report}

\section{Motivation}