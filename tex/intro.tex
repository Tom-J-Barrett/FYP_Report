\chapter{Introduction}
\label{intro}

\section{Overview}
This project explores the use of indentification and classification of food images for use in a calorie measurement android application.
Food calorie consumption is a huge problem in the modern world.
Over 25\% of the population in Ireland is obese and this figure is likely to rise over the coming years.
A mobile application that could help keep track of a user's calorie intake by taking pictures of their meals would be a great help.
The area of Machine Vision is a very difficult topic to address as it is a very hard task for computers to undertake.
We, as humans, take vision for granted as we can soon see, from the study of Machine Vision, that there are many difficult steps that have to be made for full identification and classification of an image.

When looking into calorie measurement using an image, there are three questions that have to be answered:
\begin{itemize}
\item{Where are the Regions of Interest (ROI) in this food image?}
\item{What food types are in these ROI's?}
\item{What is the portain size of each food type?}
\end{itemize}
In this project, my main focus will be on the first question, 'Where are the Regions of Interest (ROI) in this food image?'.

Many researchers in various machine vision labs have attempted to solve this
problem using different methodologies.
There has been promising results from some papers but these are mostly under highly constrained circumstances.
When mixed foods are introduced to the problem, many of the methods fail.
Convolution Neural Networks (CNN) have had very promising results in the field of image classification in the recent years but to get to the classification step, image segmentation is first needed, otherwise known as image identification.

I have researched many different methods of image segmentation but it seems that CNN's have had the best results for multiple objects in one image and I hope to apply these results for many foods in an image.

The application that I am proposing to solve the problem statement above is an easy to use Android mobile phone application.
The idea is, that when a user is about to eat their meal, they can simply take a picture of their meal for computation.
From here, the application would take the image, find the objects (ROI) in the image and take note of them.
Concurrently, the application would attempt to classify each object detected.
Once this is done, the size of each food type would be measured and through this an overall calorie count would be displayed for the user.
This could be logged for user metrics. 
I will focus on finding the objects for this application as I would not be able to implement the full system due to time constraints.

\section{Objectives}
I have a few objectives for this project and I will explain each one in detail.
\subsubsection{Understanding of Convolutional Neural Networks}
In the project, I will be using Convolutional Neural Networks (CNN's) for object identification in Food Images.
I will be using an API for this due to time constraints but it is a key
objective for me to develop a deep understanding of CNN's as they are quite pivotal in the current Machine Vision Industry and I find bio-inspired systems very interesting.

\subsubsection{Learn about different image identification and classification techniques}
Although, I will be using CNN's for my implementation, I will not be turning a blind eye to other methods of indentification and classification.
I have done extensive research on many different methods prior to my decision to use CNN's.
I think it is very important to learn about other methods as different methods
are better suited for some situations and it would be best to know about these methods due to the inevitability of their use.

\subsubsection{Develop real world skills in Machine Vision}
Machine Vision is a growing field in computer science and I think it is a very interesting field to study.
My main objective is develop the skills necessary in order to partake in Machine Vision projects in industry or to do further research in academia.

\section{Methodology}
The following methodology were adapter for this project:

\subsection{ Define the research question}
The first step to this project was to define the research question.
I knew that I wanted to look into the general area of 'Food Identification and
Classification' but the scope of this is too broad for an FYP.
Therfore I decided that the research question would be to look at the food identification aspect ie. region of interest detection.

\subsection{ Literature Review}
Once I had defined the question, finding related work was the next milestone.
There are many attempts at Food Image Classification and these were not difficult to find, using Google Scholar, but many of these papers glossed over the segmentation aspect and relied on third parties for this step.
Because of this, I had to follow quite a few references to different papers that focused solely on image segmentation.

\subsection{ Explore different image identification methods}
I had collected various image identification from the literature review that I carried out so there were many options to evaluate.
I wanted to try something with Convolutional Neural Networks so more traditional
methods of identification using colour and texture was not so strenuously explored.

\subsection{Select an image identification method}

\subsection{Research technologies and develop skills in these technologies}
As I used a Convolutional Neural Network (CNN) in this project, I leveraged many resources to enrich my understanding of the process.
I decide to use Tensorflow as my main resource in creating a CNN so I followed online tutorials for this technology.
I also enrolled in a Deep Learning Course on Udacity to enhance my understanding and skills.

\subsection{Build a prototype of the application}

\subsection{Compare and analyse results to other implementations}


\section{Overview of Report}

This report is broken down into various main headings:

\textbf{Introduction}

This section is to give on overview of what this project is about, how I will approach this project and why I am doing it.

\textbf{Background}

I will be giving some information on the background of the subject that I am focusing on.

\textbf{Experiments}

In this part of the report I will carry out various experiments using
tensorflow.

\textbf{Empirical Studies with Food Images}

This section will analyse the results I have obtained and compare them against various metrics and other implementations.

\textbf{Discussion and Conclusion}

In this section, I will discuss my results from the empirical studies and conclude my findings.


\section{Motivation}
I find the topic of Computer Vision a very interesting one.
It excites me, to be able to 'teach' a machine how to see as we do.
For this reason, I really wanted to learn about Neural Networks
and this was a large motivator for this project.

Once I had a topic that I wanted to research, I needed a focus or problem statement for this research.
I find that it is much more rewarding to work on something that positively impacts both myself and other people so I decided that I wanted to research something that fit this requirement.

Food calorie consumption is a very big problem in the modern world.
Over 25 percent of the population in Ireland are obese.
A mobile application that could help keep track of a user's calorie intake by taking a picture of their meals would be a big help to combat this problem.
This problem statement works very well for me because of it's application use and because of its complexity.
Identifying and recognising food is much more difficult than say recognising faces as it has no uniform shape.
Therefore, this problem would also be very beneficial to developing skills in the computer vision area.
