\subsection*{Objective}
In many of the papers that have been researched where food image classification was carried out, they attempted to classify a lot less than 108 food types as has been the case for experiments previously shown.
\parencite{novelSVM} used 12 classes, \parencite{pouladzadeh2014measuring} had 15, \parencite{LSL_2015} attempted to classify 11 classes of food, 20 classes were used in \parencite{chen2010toward} and \parencite{snap} predicted 15 classes.
Due to the lower number of classes in these papers, it was decided to retrain inception on a subset of the food-101 extended dataset to benchmark results.
13 classes were selected from food-101 for training.

\begin{table}[h]
\centering
\caption{Food-101+ subset}
\label{my-label}
\begin{tabular}{|l|p{9cm}|}
\hline
\textbf{Network Architecture} & Inception-V3          \\ \hline
\textbf{Dataset}              & Food-101+ subset  \\ \hline
\textbf{APIs and Libraries}   & TensorFlow and NumPy                                                       \\ \hline
\end{tabular}
\end{table}

\subsection*{Results}
A final test accuracy of 92.6\% was recorded for this experiment which performs quite high in comparison to the data in Table \ref{classes_accuracy}.
A Top 5 accuracy of 100\% was calculated with an average prediction probability of 0.89 using 130 images.
This was calculated using the script defined in \ref{parameterTuning}.

\begin{table}[]
\centering
\caption{Accuracy of other studies}
\label{classes_accuracy}
\begin{tabular}{|l|l|l|}
\hline
\textbf{Reference}                       & \textbf{Classes} & \textbf{Accuracy}      \\ \hline
Novel SVM                       & 12      & 92.6\%        \\ \hline
Measuring Calorie and Nutrition & 15      & 90.4\%       \\ \hline
Large Scale Learning            & 11      & 78.0\%          \\ \hline
Toward Dietary Assessment       & 20      & 91.7\% \\ \hline
Snap-n-eat                      & 15      & 85.0\%         \\ \hline
\end{tabular}
\end{table}

\subsection*{Analysis}
It would make sense the accuracy of our model would increase when the number of classes are reduced as the margin of error is decreased.
The Top 5 accuracy of the model was very successful.
There were only 10 images per class tested though (totalling at 130 images) so if the number of images increased, this accuracy would probably decrease slightly.
If the code is run again, the same results cannot even be replicated for sure.
On another run of this code a Top 5 accuracy of 96\% was recorded.
While this is still a very good accuracy, it does not compare to the first run of the script.