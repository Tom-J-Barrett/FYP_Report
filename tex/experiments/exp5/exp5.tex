\subsection*{Overview}
As the Food-101 dataset mostly consisted of meals \textcite{food101}, it was decided to extend the
dataset slightly by including some single foods such as:
\begin{itemize}
    \item{cheese}
    \item{grapes}
    \item{banana}
    \item{apple}
    \item{orange}
    \item{spaghetti}
    \item{roll}
\end{itemize}

In order to collect these images, the ImageNet repository was utilised to search for
these foods individually and then download the subset of images to be included
with the Food 101 dataset \textcite{imagenet}. The retrain.py script was run on the extended dataset as in
\ref{inception}.

\subsection*{Network Architecture}
The Inception V3 Model was used for this experiment.

\subsection*{Dataset}
In this experiment the extended Food-101 dataset was used.

\subsection*{Libraries}
Tensorflow and numpy are used in the retrain.py script.

\subsection*{Script}
As seen in \ref{extended}.

\subsection*{Results}
For this model, an accuracy of 55.3\% was achieved.

For example, an image of a banana (Figure \ref{fig:banana}) was fed into the model with
the followng results:
\begin{itemize}
    \item{banana 0.9962}
    \item{orange 0.0009}
    \item{cheese 0.0003}
    \item{frozen yoghurt carpaccio 0.0002}
    \item{churros 0.0001}
\end{itemize}
 
\begin{figure}
    \includegraphics{banana}
    \caption{Banana}
    \label{fig:banana}
\end{figure}

\subsection*{Empirical Analysis}
The slight increase in accuracy in \ref{inception}, from 54.8\% to 55.3\%, makes
sense. Since the model was pre-trained using the ImageNet dataset and all the
new images used were from ImageNet, we would expect a higher classification
accuracy on the new additions to the dataset. This would overall increase the
average classification accuracy.
