\subsection*{Overview}
Similar to the previous experiment, this experiment is a Udemy course exercise \textcite{udemy}. In contrast, this does not use a dataset built into Tensorflow so therefore, there is extra configuration to be done on the dataset.

\subsection*{Network Architecture}
Same architecture as in experiment 1.

\subsection*{Dataset}
The CIFAR-10 dataset is used here. 
This has 60000 images split into 10 classes and a test set of 1000 images \textcite{cifar}.

\subsection*{Script}
\begin{lstlisting}
def unpickle(file):
    import pickle
    with open(file, 'rb') as fo:
        cifar_dict = pickle.load(fo, encoding='bytes')
    return cifar_dict

dirs = ['batches.meta','data_batch_1','data_batch_2','data_batch_3','data_batch_4','data_batch_5','test_batch']

all_data = [0,1,2,3,4,5,6]

for i,direc in zip(all_data,dirs):
    all_data[i] = unpickle(CIFAR_DIR+direc)

batch_meta = all_data[0]
data_batch1 = all_data[1]
data_batch2 = all_data[2]
data_batch3 = all_data[3]
data_batch4 = all_data[4]
data_batch5 = all_data[5]
test_batch = all_data[6]
\end{lstlisting}

\begin{lstlisting}
def set_up_images(self):
        
        print("Setting Up Training Images and Labels")
        
        # Vertically stacks the training images
        self.training_images = np.vstack([d[b"data"] for d in self.all_train_batches])
        train_len = len(self.training_images)
        
        # Reshapes and normalizes training images
        self.training_images = self.training_images.reshape(train_len,3,32,32).transpose(0,2,3,1)/255
        # One hot Encodes the training labels (e.g. [0,0,0,1,0,0,0,0,0,0])
        self.training_labels = one_hot_encode(np.hstack([d[b"labels"] for d in self.all_train_batches]), 10)
        
        print("Setting Up Test Images and Labels")
        
        # Vertically stacks the test images
        self.test_images = np.vstack([d[b"data"] for d in self.test_batch])
        test_len = len(self.test_images)
        
        # Reshapes and normalizes test images
        self.test_images = self.test_images.reshape(test_len,3,32,32).transpose(0,2,3,1)/255
        # One hot Encodes the test labels (e.g. [0,0,0,1,0,0,0,0,0,0])
        self.test_labels = one_hot_encode(np.hstack([d[b"labels"] for d in self.test_batch]), 10)

        
    def next_batch(self, batch_size):
        # Note that the 100 dimension in the reshape call is set by an assumed batch size of 100
        x = self.training_images[self.i:self.i+batch_size].reshape(100,32,32,3)
        y = self.training_labels[self.i:self.i+batch_size]
        self.i = (self.i + batch_size) % len(self.training_images)
return x, y
\end{lstlisting}

\begin{lstlisting}
cross_entropy = tf.reduce_mean(tf.nn.softmax_cross_entropy_with_logits(labels = y_true, logits = y_pred))
optimizer = tf.train.AdamOptimizer(learning_rate = 0.001)
train = optimizer.minimize(cross_entropy)
init = tf.global_variables_initializer()
with tf.Session() as sess:
    sess.run(tf.global_variables_initializer())

    for i in range(10000):
        batch = ch.next_batch(100)
        sess.run(train, feed_dict={x: batch[0], y_true: batch[1], hold_prob: 0.5})
        
        # PRINT OUT A MESSAGE EVERY 100 STEPS
        if i%1000 == 0:
            
            # Test the Train Model
            matches = tf.equal(tf.argmax(y_pred,1),tf.argmax(y_true,1))

            acc = tf.reduce_mean(tf.cast(matches,tf.float32))
            testSet = ch.next_test_batch(100)
            print('Accuracy: ')
            print(sess.run(acc,feed_dict={x:testSet[0] ,y_true:testSet[1],hold_prob:1.0}))
print('\n')
\end{lstlisting}

\subsection*{Results}
A Final Accuracy of 71\% was reached.

\subsection*{Empirical Analysis}
