\subsection*{Overview}
In the preious experiments, I have looked at the one-shot approach to food image
classification. That is, the model will give a prediction of the most likely
food item in that image. This is a problem when there are multiple food in an
image, see \ref{fig:fruit}. There are a few options to combat this problem. Firstly, I could detect
objects in the image, segment the image according to these objects and then run
each segment through the model. A simple approach to this would be to segment
the image into a number of sections and then run each section through the model.
In order to follow the latter approach, I used a sliding window approach. This
sliding window would move across the image and classify the segment of the image
in the window. I had three options for window sliding shape as defined by a
command line argument.

\begin{lstlisting}
python sliding_window.py --image=~/image_dir --window_shape grid
\end{lstlisting}

There are three options for window shape:
\begin{itemize}
	\item{Grid based window as per \ref{fig:fruitGrid}}
	\item{Row based window as per \ref{fig:fruitRow}}
	\item{Column based window as per \ref{fig:fruitColumn}}
\end{itemize}

\begin{figure}
    \includegraphics[scale=0.5]{fruit}
    \caption{Bowl of fruit}
    \label{fig:fruit}
\end{figure}

\begin{figure}
    \includegraphics[scale=0.5]{fruitGrid}
	\caption{Grid based window}
    \label{fig:fruitGrid}
\end{figure}

\begin{figure}
    \includegraphics[scale=0.5]{fruitRow}
    \caption{Row based window}
    \label{fig:fruitRow}
\end{figure}

\begin{figure}
    \includegraphics[scale=0.5]{fruitColumn}
    \caption{Column Based Window}
    \label{fig:fruitColumn}
\end{figure}

\begin{figure}
    \includegraphics[scale=0.5]{fruitCO}
    \caption{Fruit with Color Overlay}
    \label{fig:fruitOverlay}
\end{figure}

\subsection*{Libraries}
For this experiment Tensorflow provided the classification of each segment while
also helping with resizing along with Numpy. OpenCv was used to implement the
sliding window as per \textcite{slidingWindowTut}

\subsection*{Script}
There were four main elements to the script. Firstly, extracting a window to be
classified. Secondly, resizing the image to be compatible with the Tensorflow
model. Thirdly, running the window through the Tensorflow model and finally
saving a new image with a coloured overlay of classifications.

\subsubsection*{Extracting the window from the image}
\begin{lstlisting}
# loop over the image pyramid
for resized in pyramid(image, scale=1.5):
		# loop over the sliding window for each layer of the pyramid
		for (x, y, window) in sliding_window(resized, stepSize=32, windowSize=(winW, winH)):
			# if the window does not meet our desired window size, ignore it
				if window.shape[0] != winH or window.shape[1] !=winW:
					continue
\end{lstlisting}


\begin{lstlisting}
def sliding_window(image, stepSize, windowSize):
	# slide a window across the image
	for y in xrange(0, image.shape[0], stepSize):
		for x in xrange(0, image.shape[1], stepSize):
			# yield the current window
			yield (x, y, image[y:y + windowSize[1], x:x + windowSize[0]])
\end{lstlisting}

\subsubsection*{Resizing the window}
\begin{lstlisting}
window = cv2.resize(window, (299, 299))
\end{lstlisting}

\begin{lstlisting}
resized_image = tf.reshape(image, [1, input_height, input_width, 3])
resized = tf.image.resize_area(resized_image, [input_height, input_width])
normalized = tf.divide(tf.subtract(resized, [input_mean]), [input_std])
\end{lstlisting}

\subsubsection*{Running the Tensorflow model}
\begin{lstlisting}
with tf.Session() as sess:
	numpy_image = sess.run(normalized)

with tf.Session(graph=graph) as sess:
    results = sess.run(output_operation.outputs[0],
                      {input_operation.outputs[0]: numpy_image})
	probabilities = np.squeeze(results)
\end{lstlisting}

\subsubsection*{Saving the image with colour overlay}
As seen in Figure \ref{fig:fruitOverlay}, each square represents a window and
each colour is for a different classification. Blue is for an apple, yellow for
banana, green for grape, white for orange and black if an unexpected prediction
is made.

\begin{lstlisting}
cv2.rectangle(display_image, (x, y), (x + winW, y + winH), colour_dict.get(top1,
(0,0,0)), 4)
\end{lstlisting}

\subsection*{Results}
\subsubsection*{Grid based window}
The grid based window resulted in fifteen separate classification. As seen in
\ref{fig:fruit}, there are multiple fruits in the image. Of these fruits, our
model is trained on four, apple, banana, orange and grapes. This method
classified all four to Top-1 accuracy at least once each. This method took 42.8
seconds to run.

\begin{table}[]
	\centering
	\caption{My caption}
	\label{my-label}
	\begin{tabular}{ll}
		Food type & No. of Top-1 Classifications \\
		Apple     & 5                      \\
		Banana    & 1                      \\
		Grape     & 4                      \\
		Orange    & 5                     
	\end{tabular}
\end{table}

\begin{table}[]
	\centering
	\caption{My caption}
	\label{rowWindowTable}
	\begin{tabular}{ll}
		Food type & No. of Top-1 Classifications \\
		Apple     & 1                      \\
		Banana    & 0                      \\
		Grape     & 0                      \\
		Orange    & 0                      \\
		Other     & 3                     
	\end{tabular}
\end{table}

\begin{table}[]
	\centering
	\caption{My caption}
	\label{colWindowTable}
	\begin{tabular}{ll}
		Food type & No. of Top-1 Classifications \\
		Apple     & 3                      \\
		Banana    & 1                      \\
		Grape     & 0                      \\
		Orange    & 0                      \\
		Other     & 1                     
	\end{tabular}
\end{table}

\subsubsection*{Row based window}
The row based method resulted in four predictions as follows in
\ref{rowWindowTable}. Out of these four predictions, only one classified a known
fruit at Top-1 accuracy, an apple. An apple was also predicted to Top-5 accuracy
in another instance. The runtime of this method was 16.1 seconds.

\subsubsection*{Column based window}
The column based window approach had five total predictions and ran for a total
of 13 seconds. As seen in \ref{colWindowTable}, two out of four known fruits
were classified to a Top-1 accuracy with all other fruits predicted to Top-5
accuracy. Only one Top-1 prediction did not contain a correct fruit.

\subsection*{Analysis}
These results are very interesting because while a banana was only predicted to
Top-1 accuracy once in grid based, once in column based and zero times in row
based, if the whole image is ran through the model, a banana is at the Top-1 accuracy.
























