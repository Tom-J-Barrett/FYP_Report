\subsection*{Overview}
Due to the fact that the end goal for this project is to have a smartphone
application that a user can use to keep track of their calorie measurement,
there are a couple of options in how to achieve this. Firstly, an image can be
taken on the phone and sent to a server to run a classification algorithm.
Secondly, a model can be stored on the phone for computation. I decided to train
the model, using transfer learning as before, but on a different architecture,
MobileNet.

\subsection*{Network Architecture}
The network architecture used for this experiment is MobileNet
\textcite{retrainInception}. This architecture is designed to be smaller so that
it can be used on smartphones which have less powerful resources available.

\subsection*{Dataset}
The Food 101 dataset \textcite{food101} with added classes was used for this experiment.

\subsection*{Libraries}
Tensorflow and numpy.

\subsection*{Script}
The retrain.py script \textcite{retrainInception} was used, with a different
command paramater.

\begin{lstlisting}
python tensorflow/examples/image_retraining/retrain.py \ --image_dir
~/dataset_directory \ --architecture mobilenet_1.0_224 \
--how_many_training_steps 10000 \ --learning_rate 0.1
\end{lstlisting}

\subsection*{Results}
The final test accuracy of this model came to 50.2\%.

\subsection*{Analysis}
There was a decrease of 16.1\% in this model to the highest accuracy from
experiment 7. This is due to the smaller architecture which is aimed to be faster
and smaller with the expected decrease in accuracy.
