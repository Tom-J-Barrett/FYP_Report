\tocless\subsection{Objective}
Since the end goal for this project is to have a smartphone
application that a user can use to keep track of their calorie measurement,
there are a couple of options in how to achieve this. Firstly, an image can be
taken on the phone and sent to a server to run a classification algorithm.
Secondly, a model can be stored on the phone for computation. Transfer learning was once again used but on a different, smaller architecture called MobileNet \parencite{mobilenet}.

\begin{table}[h]
\centering
\caption{MobileNet}
\label{my-label}
\begin{tabular}{|l|p{8cm}|}
\hline
\textbf{Network Architecture} & MobileNet           \\ \hline
\textbf{Dataset}              & Food-101+ dataset \\ \hline
\textbf{APIs and Libraries}   & TensorFlow and NumPy                                                       \\ \hline
\end{tabular}
\end{table}

\tocless\subsection{Script}
The retrain.py script \parencite{retrainInception} was used with a different
command parameter as in Figure \ref{lst:mobilenetCommand}.

\begin{figure}[h]
\caption{Train Model using MobileNet Architecture Command}
\label{lst:mobilenetCommand}
\begin{lstlisting}[style=Command]
python TensorFlow/examples/image_retraining/retrain.py \ --image_dir
~/dataset_directory \ --architecture mobilenet_1.0_224 \
--how_many_training_steps 10000 \ --learning_rate 0.1
\end{lstlisting}
\end{figure}

\tocless\subsection{Results}
The final Top-1 accuracy of this model came to 50.2\%.
The 95\% confidence interval of the Top-1 accuracy of this model is between 40.8\% and 59.6\%.

\tocless\subsection{Analysis}
There was a decrease of 16.1\% in this model to the highest accuracy from
\ref{mobilenet}. This is due to the smaller architecture which is aimed to be faster
and smaller with an expected decrease in accuracy.
