\subsection*{Overview}
For the next experiment, it was decided to use the Food-101 dataset.
An on line tutorial was used to create a dataset in tensorflow using image directories on disk \textcite{file_dir_code}.
Tutorials used previously were also utilized for this experiment \textcite{udemy} and \textcite{cifar}.

\subsection*{Network Architecture}
A similar network architecture to the previous two experiments (\ref{udemy1} and \ref{udemy2}) was used here.

\subsection*{Dataset}
The Food-101 datset was used for this experiment \textcite{food101}.

\subsection*{Script}
\begin{lstlisting}
# Reading the dataset
# 2 modes: 'file' or 'folder'
def read_images(dataset_path, mode, batch_size):
    imagepaths, labels = list(), list()
    if mode == 'file':
        # Read dataset file
        data = open(dataset_path, 'r').read().splitlines()
        for d in data:
            imagepaths.append(d.split(' ')[0])
            labels.append(int(d.split(' ')[1]))
    elif mode == 'folder':
        # An ID will be affected to each sub-folders by alphabetical order
        label = 0
        # List the directory
        try:  # Python 2
            classes = sorted(os.walk(dataset_path).next()[1])
        except Exception:  # Python 3
            classes = sorted(os.walk(dataset_path).__next__()[1])
        # List each sub-directory (the classes)
        for c in classes:
            c_dir = os.path.join(dataset_path, c)
            try:  # Python 2
                walk = os.walk(c_dir).next()
            except Exception:  # Python 3
                walk = os.walk(c_dir).__next__()
            # Add each image to the training set
            for sample in walk[2]:
                # Only keeps jpeg images
                if sample.endswith('.jpg') or sample.endswith('.jpeg'):
                    imagepaths.append(os.path.join(c_dir, sample))
                    labels.append(label)
            label += 1
    else:
raise Exception("Unknown mode.")
\end{lstlisting}

\begin{lstlisting}
# Convert to Tensor
    imagepaths = tf.convert_to_tensor(imagepaths, dtype=tf.string)
    labels = tf.convert_to_tensor(labels, dtype=tf.int32)
    # Build a TF Queue, shuffle data
    image, label = tf.train.slice_input_producer([imagepaths, labels],
                                                 shuffle=True)

    # Read images from disk
    image = tf.read_file(image)
    image = tf.image.decode_jpeg(image, channels=CHANNELS)

    # Resize images to a common size
    image = tf.image.resize_images(image, [IMG_HEIGHT, IMG_WIDTH])

    # Normalize
    image = image * 1.0/127.5 - 1.0

    # Create batches
    X, Y = tf.train.batch([image, label], batch_size=batch_size,
                          capacity=batch_size * 8,
                          num_threads=4)

    return X, Y

\end{lstlisting}

\begin{lstlisting}
# Start training
with tf.Session() as sess:

    # Run the initializer
    sess.run(init)

    # Start the data queue
    tf.train.start_queue_runners()

    # Training cycle
    for step in range(1, num_steps+1):

        if step % display_step == 0:
            # Run optimization and calculate batch loss and accuracy
            _, loss, acc = sess.run([train_op, loss_op, accuracy])
            print("Step " + str(step) + ", Minibatch Loss= " + \
                  "{:.4f}".format(loss) + ", Training Accuracy= " + \
                  "{:.3f}".format(acc))
        else:
            # Only run the optimization op (backprop)
            sess.run(train_op)

print("Optimization Finished!")
\end{lstlisting}

\subsection*{Results}
The final accuracy for this model was 18.8\% after 5000 steps.

\subsection*{Empirical Analysis}
The poor accuracy of this model is to be expected due to the simple architecture of the network and the fact that 101 classes is a lot to process.