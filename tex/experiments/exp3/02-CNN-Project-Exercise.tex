
% Default to the notebook output style

    


% Inherit from the specified cell style.




    
\documentclass[11pt]{article}

    
    
    \usepackage[T1]{fontenc}
    % Nicer default font (+ math font) than Computer Modern for most use cases
    \usepackage{mathpazo}

    % Basic figure setup, for now with no caption control since it's done
    % automatically by Pandoc (which extracts ![](path) syntax from Markdown).
    \usepackage{graphicx}
    % We will generate all images so they have a width \maxwidth. This means
    % that they will get their normal width if they fit onto the page, but
    % are scaled down if they would overflow the margins.
    \makeatletter
    \def\maxwidth{\ifdim\Gin@nat@width>\linewidth\linewidth
    \else\Gin@nat@width\fi}
    \makeatother
    \let\Oldincludegraphics\includegraphics
    % Set max figure width to be 80% of text width, for now hardcoded.
    \renewcommand{\includegraphics}[1]{\Oldincludegraphics[width=.8\maxwidth]{#1}}
    % Ensure that by default, figures have no caption (until we provide a
    % proper Figure object with a Caption API and a way to capture that
    % in the conversion process - todo).
    \usepackage{caption}
    \DeclareCaptionLabelFormat{nolabel}{}
    \captionsetup{labelformat=nolabel}

    \usepackage{adjustbox} % Used to constrain images to a maximum size 
    \usepackage{xcolor} % Allow colors to be defined
    \usepackage{enumerate} % Needed for markdown enumerations to work
    \usepackage{geometry} % Used to adjust the document margins
    \usepackage{amsmath} % Equations
    \usepackage{amssymb} % Equations
    \usepackage{textcomp} % defines textquotesingle
    % Hack from http://tex.stackexchange.com/a/47451/13684:
    \AtBeginDocument{%
        \def\PYZsq{\textquotesingle}% Upright quotes in Pygmentized code
    }
    \usepackage{upquote} % Upright quotes for verbatim code
    \usepackage{eurosym} % defines \euro
    \usepackage[mathletters]{ucs} % Extended unicode (utf-8) support
    \usepackage[utf8x]{inputenc} % Allow utf-8 characters in the tex document
    \usepackage{fancyvrb} % verbatim replacement that allows latex
    \usepackage{grffile} % extends the file name processing of package graphics 
                         % to support a larger range 
    % The hyperref package gives us a pdf with properly built
    % internal navigation ('pdf bookmarks' for the table of contents,
    % internal cross-reference links, web links for URLs, etc.)
    \usepackage{hyperref}
    \usepackage{longtable} % longtable support required by pandoc >1.10
    \usepackage{booktabs}  % table support for pandoc > 1.12.2
    \usepackage[inline]{enumitem} % IRkernel/repr support (it uses the enumerate* environment)
    \usepackage[normalem]{ulem} % ulem is needed to support strikethroughs (\sout)
                                % normalem makes italics be italics, not underlines
    

    
    
    % Colors for the hyperref package
    \definecolor{urlcolor}{rgb}{0,.145,.698}
    \definecolor{linkcolor}{rgb}{.71,0.21,0.01}
    \definecolor{citecolor}{rgb}{.12,.54,.11}

    % ANSI colors
    \definecolor{ansi-black}{HTML}{3E424D}
    \definecolor{ansi-black-intense}{HTML}{282C36}
    \definecolor{ansi-red}{HTML}{E75C58}
    \definecolor{ansi-red-intense}{HTML}{B22B31}
    \definecolor{ansi-green}{HTML}{00A250}
    \definecolor{ansi-green-intense}{HTML}{007427}
    \definecolor{ansi-yellow}{HTML}{DDB62B}
    \definecolor{ansi-yellow-intense}{HTML}{B27D12}
    \definecolor{ansi-blue}{HTML}{208FFB}
    \definecolor{ansi-blue-intense}{HTML}{0065CA}
    \definecolor{ansi-magenta}{HTML}{D160C4}
    \definecolor{ansi-magenta-intense}{HTML}{A03196}
    \definecolor{ansi-cyan}{HTML}{60C6C8}
    \definecolor{ansi-cyan-intense}{HTML}{258F8F}
    \definecolor{ansi-white}{HTML}{C5C1B4}
    \definecolor{ansi-white-intense}{HTML}{A1A6B2}

    % commands and environments needed by pandoc snippets
    % extracted from the output of `pandoc -s`
    \providecommand{\tightlist}{%
      \setlength{\itemsep}{0pt}\setlength{\parskip}{0pt}}
    \DefineVerbatimEnvironment{Highlighting}{Verbatim}{commandchars=\\\{\}}
    % Add ',fontsize=\small' for more characters per line
    \newenvironment{Shaded}{}{}
    \newcommand{\KeywordTok}[1]{\textcolor[rgb]{0.00,0.44,0.13}{\textbf{{#1}}}}
    \newcommand{\DataTypeTok}[1]{\textcolor[rgb]{0.56,0.13,0.00}{{#1}}}
    \newcommand{\DecValTok}[1]{\textcolor[rgb]{0.25,0.63,0.44}{{#1}}}
    \newcommand{\BaseNTok}[1]{\textcolor[rgb]{0.25,0.63,0.44}{{#1}}}
    \newcommand{\FloatTok}[1]{\textcolor[rgb]{0.25,0.63,0.44}{{#1}}}
    \newcommand{\CharTok}[1]{\textcolor[rgb]{0.25,0.44,0.63}{{#1}}}
    \newcommand{\StringTok}[1]{\textcolor[rgb]{0.25,0.44,0.63}{{#1}}}
    \newcommand{\CommentTok}[1]{\textcolor[rgb]{0.38,0.63,0.69}{\textit{{#1}}}}
    \newcommand{\OtherTok}[1]{\textcolor[rgb]{0.00,0.44,0.13}{{#1}}}
    \newcommand{\AlertTok}[1]{\textcolor[rgb]{1.00,0.00,0.00}{\textbf{{#1}}}}
    \newcommand{\FunctionTok}[1]{\textcolor[rgb]{0.02,0.16,0.49}{{#1}}}
    \newcommand{\RegionMarkerTok}[1]{{#1}}
    \newcommand{\ErrorTok}[1]{\textcolor[rgb]{1.00,0.00,0.00}{\textbf{{#1}}}}
    \newcommand{\NormalTok}[1]{{#1}}
    
    % Additional commands for more recent versions of Pandoc
    \newcommand{\ConstantTok}[1]{\textcolor[rgb]{0.53,0.00,0.00}{{#1}}}
    \newcommand{\SpecialCharTok}[1]{\textcolor[rgb]{0.25,0.44,0.63}{{#1}}}
    \newcommand{\VerbatimStringTok}[1]{\textcolor[rgb]{0.25,0.44,0.63}{{#1}}}
    \newcommand{\SpecialStringTok}[1]{\textcolor[rgb]{0.73,0.40,0.53}{{#1}}}
    \newcommand{\ImportTok}[1]{{#1}}
    \newcommand{\DocumentationTok}[1]{\textcolor[rgb]{0.73,0.13,0.13}{\textit{{#1}}}}
    \newcommand{\AnnotationTok}[1]{\textcolor[rgb]{0.38,0.63,0.69}{\textbf{\textit{{#1}}}}}
    \newcommand{\CommentVarTok}[1]{\textcolor[rgb]{0.38,0.63,0.69}{\textbf{\textit{{#1}}}}}
    \newcommand{\VariableTok}[1]{\textcolor[rgb]{0.10,0.09,0.49}{{#1}}}
    \newcommand{\ControlFlowTok}[1]{\textcolor[rgb]{0.00,0.44,0.13}{\textbf{{#1}}}}
    \newcommand{\OperatorTok}[1]{\textcolor[rgb]{0.40,0.40,0.40}{{#1}}}
    \newcommand{\BuiltInTok}[1]{{#1}}
    \newcommand{\ExtensionTok}[1]{{#1}}
    \newcommand{\PreprocessorTok}[1]{\textcolor[rgb]{0.74,0.48,0.00}{{#1}}}
    \newcommand{\AttributeTok}[1]{\textcolor[rgb]{0.49,0.56,0.16}{{#1}}}
    \newcommand{\InformationTok}[1]{\textcolor[rgb]{0.38,0.63,0.69}{\textbf{\textit{{#1}}}}}
    \newcommand{\WarningTok}[1]{\textcolor[rgb]{0.38,0.63,0.69}{\textbf{\textit{{#1}}}}}
    
    
    % Define a nice break command that doesn't care if a line doesn't already
    % exist.
    \def\br{\hspace*{\fill} \\* }
    % Math Jax compatability definitions
    \def\gt{>}
    \def\lt{<}
    % Document parameters
    \title{}
    \date{}
    
    
    

    % Pygments definitions
    
\makeatletter
\def\PY@reset{\let\PY@it=\relax \let\PY@bf=\relax%
    \let\PY@ul=\relax \let\PY@tc=\relax%
    \let\PY@bc=\relax \let\PY@ff=\relax}
\def\PY@tok#1{\csname PY@tok@#1\endcsname}
\def\PY@toks#1+{\ifx\relax#1\empty\else%
    \PY@tok{#1}\expandafter\PY@toks\fi}
\def\PY@do#1{\PY@bc{\PY@tc{\PY@ul{%
    \PY@it{\PY@bf{\PY@ff{#1}}}}}}}
\def\PY#1#2{\PY@reset\PY@toks#1+\relax+\PY@do{#2}}

\expandafter\def\csname PY@tok@gd\endcsname{\def\PY@tc##1{\textcolor[rgb]{0.63,0.00,0.00}{##1}}}
\expandafter\def\csname PY@tok@gu\endcsname{\let\PY@bf=\textbf\def\PY@tc##1{\textcolor[rgb]{0.50,0.00,0.50}{##1}}}
\expandafter\def\csname PY@tok@gt\endcsname{\def\PY@tc##1{\textcolor[rgb]{0.00,0.27,0.87}{##1}}}
\expandafter\def\csname PY@tok@gs\endcsname{\let\PY@bf=\textbf}
\expandafter\def\csname PY@tok@gr\endcsname{\def\PY@tc##1{\textcolor[rgb]{1.00,0.00,0.00}{##1}}}
\expandafter\def\csname PY@tok@cm\endcsname{\let\PY@it=\textit\def\PY@tc##1{\textcolor[rgb]{0.25,0.50,0.50}{##1}}}
\expandafter\def\csname PY@tok@vg\endcsname{\def\PY@tc##1{\textcolor[rgb]{0.10,0.09,0.49}{##1}}}
\expandafter\def\csname PY@tok@vi\endcsname{\def\PY@tc##1{\textcolor[rgb]{0.10,0.09,0.49}{##1}}}
\expandafter\def\csname PY@tok@vm\endcsname{\def\PY@tc##1{\textcolor[rgb]{0.10,0.09,0.49}{##1}}}
\expandafter\def\csname PY@tok@mh\endcsname{\def\PY@tc##1{\textcolor[rgb]{0.40,0.40,0.40}{##1}}}
\expandafter\def\csname PY@tok@cs\endcsname{\let\PY@it=\textit\def\PY@tc##1{\textcolor[rgb]{0.25,0.50,0.50}{##1}}}
\expandafter\def\csname PY@tok@ge\endcsname{\let\PY@it=\textit}
\expandafter\def\csname PY@tok@vc\endcsname{\def\PY@tc##1{\textcolor[rgb]{0.10,0.09,0.49}{##1}}}
\expandafter\def\csname PY@tok@il\endcsname{\def\PY@tc##1{\textcolor[rgb]{0.40,0.40,0.40}{##1}}}
\expandafter\def\csname PY@tok@go\endcsname{\def\PY@tc##1{\textcolor[rgb]{0.53,0.53,0.53}{##1}}}
\expandafter\def\csname PY@tok@cp\endcsname{\def\PY@tc##1{\textcolor[rgb]{0.74,0.48,0.00}{##1}}}
\expandafter\def\csname PY@tok@gi\endcsname{\def\PY@tc##1{\textcolor[rgb]{0.00,0.63,0.00}{##1}}}
\expandafter\def\csname PY@tok@gh\endcsname{\let\PY@bf=\textbf\def\PY@tc##1{\textcolor[rgb]{0.00,0.00,0.50}{##1}}}
\expandafter\def\csname PY@tok@ni\endcsname{\let\PY@bf=\textbf\def\PY@tc##1{\textcolor[rgb]{0.60,0.60,0.60}{##1}}}
\expandafter\def\csname PY@tok@nl\endcsname{\def\PY@tc##1{\textcolor[rgb]{0.63,0.63,0.00}{##1}}}
\expandafter\def\csname PY@tok@nn\endcsname{\let\PY@bf=\textbf\def\PY@tc##1{\textcolor[rgb]{0.00,0.00,1.00}{##1}}}
\expandafter\def\csname PY@tok@no\endcsname{\def\PY@tc##1{\textcolor[rgb]{0.53,0.00,0.00}{##1}}}
\expandafter\def\csname PY@tok@na\endcsname{\def\PY@tc##1{\textcolor[rgb]{0.49,0.56,0.16}{##1}}}
\expandafter\def\csname PY@tok@nb\endcsname{\def\PY@tc##1{\textcolor[rgb]{0.00,0.50,0.00}{##1}}}
\expandafter\def\csname PY@tok@nc\endcsname{\let\PY@bf=\textbf\def\PY@tc##1{\textcolor[rgb]{0.00,0.00,1.00}{##1}}}
\expandafter\def\csname PY@tok@nd\endcsname{\def\PY@tc##1{\textcolor[rgb]{0.67,0.13,1.00}{##1}}}
\expandafter\def\csname PY@tok@ne\endcsname{\let\PY@bf=\textbf\def\PY@tc##1{\textcolor[rgb]{0.82,0.25,0.23}{##1}}}
\expandafter\def\csname PY@tok@nf\endcsname{\def\PY@tc##1{\textcolor[rgb]{0.00,0.00,1.00}{##1}}}
\expandafter\def\csname PY@tok@si\endcsname{\let\PY@bf=\textbf\def\PY@tc##1{\textcolor[rgb]{0.73,0.40,0.53}{##1}}}
\expandafter\def\csname PY@tok@s2\endcsname{\def\PY@tc##1{\textcolor[rgb]{0.73,0.13,0.13}{##1}}}
\expandafter\def\csname PY@tok@nt\endcsname{\let\PY@bf=\textbf\def\PY@tc##1{\textcolor[rgb]{0.00,0.50,0.00}{##1}}}
\expandafter\def\csname PY@tok@nv\endcsname{\def\PY@tc##1{\textcolor[rgb]{0.10,0.09,0.49}{##1}}}
\expandafter\def\csname PY@tok@s1\endcsname{\def\PY@tc##1{\textcolor[rgb]{0.73,0.13,0.13}{##1}}}
\expandafter\def\csname PY@tok@dl\endcsname{\def\PY@tc##1{\textcolor[rgb]{0.73,0.13,0.13}{##1}}}
\expandafter\def\csname PY@tok@ch\endcsname{\let\PY@it=\textit\def\PY@tc##1{\textcolor[rgb]{0.25,0.50,0.50}{##1}}}
\expandafter\def\csname PY@tok@m\endcsname{\def\PY@tc##1{\textcolor[rgb]{0.40,0.40,0.40}{##1}}}
\expandafter\def\csname PY@tok@gp\endcsname{\let\PY@bf=\textbf\def\PY@tc##1{\textcolor[rgb]{0.00,0.00,0.50}{##1}}}
\expandafter\def\csname PY@tok@sh\endcsname{\def\PY@tc##1{\textcolor[rgb]{0.73,0.13,0.13}{##1}}}
\expandafter\def\csname PY@tok@ow\endcsname{\let\PY@bf=\textbf\def\PY@tc##1{\textcolor[rgb]{0.67,0.13,1.00}{##1}}}
\expandafter\def\csname PY@tok@sx\endcsname{\def\PY@tc##1{\textcolor[rgb]{0.00,0.50,0.00}{##1}}}
\expandafter\def\csname PY@tok@bp\endcsname{\def\PY@tc##1{\textcolor[rgb]{0.00,0.50,0.00}{##1}}}
\expandafter\def\csname PY@tok@c1\endcsname{\let\PY@it=\textit\def\PY@tc##1{\textcolor[rgb]{0.25,0.50,0.50}{##1}}}
\expandafter\def\csname PY@tok@fm\endcsname{\def\PY@tc##1{\textcolor[rgb]{0.00,0.00,1.00}{##1}}}
\expandafter\def\csname PY@tok@o\endcsname{\def\PY@tc##1{\textcolor[rgb]{0.40,0.40,0.40}{##1}}}
\expandafter\def\csname PY@tok@kc\endcsname{\let\PY@bf=\textbf\def\PY@tc##1{\textcolor[rgb]{0.00,0.50,0.00}{##1}}}
\expandafter\def\csname PY@tok@c\endcsname{\let\PY@it=\textit\def\PY@tc##1{\textcolor[rgb]{0.25,0.50,0.50}{##1}}}
\expandafter\def\csname PY@tok@mf\endcsname{\def\PY@tc##1{\textcolor[rgb]{0.40,0.40,0.40}{##1}}}
\expandafter\def\csname PY@tok@err\endcsname{\def\PY@bc##1{\setlength{\fboxsep}{0pt}\fcolorbox[rgb]{1.00,0.00,0.00}{1,1,1}{\strut ##1}}}
\expandafter\def\csname PY@tok@mb\endcsname{\def\PY@tc##1{\textcolor[rgb]{0.40,0.40,0.40}{##1}}}
\expandafter\def\csname PY@tok@ss\endcsname{\def\PY@tc##1{\textcolor[rgb]{0.10,0.09,0.49}{##1}}}
\expandafter\def\csname PY@tok@sr\endcsname{\def\PY@tc##1{\textcolor[rgb]{0.73,0.40,0.53}{##1}}}
\expandafter\def\csname PY@tok@mo\endcsname{\def\PY@tc##1{\textcolor[rgb]{0.40,0.40,0.40}{##1}}}
\expandafter\def\csname PY@tok@kd\endcsname{\let\PY@bf=\textbf\def\PY@tc##1{\textcolor[rgb]{0.00,0.50,0.00}{##1}}}
\expandafter\def\csname PY@tok@mi\endcsname{\def\PY@tc##1{\textcolor[rgb]{0.40,0.40,0.40}{##1}}}
\expandafter\def\csname PY@tok@kn\endcsname{\let\PY@bf=\textbf\def\PY@tc##1{\textcolor[rgb]{0.00,0.50,0.00}{##1}}}
\expandafter\def\csname PY@tok@cpf\endcsname{\let\PY@it=\textit\def\PY@tc##1{\textcolor[rgb]{0.25,0.50,0.50}{##1}}}
\expandafter\def\csname PY@tok@kr\endcsname{\let\PY@bf=\textbf\def\PY@tc##1{\textcolor[rgb]{0.00,0.50,0.00}{##1}}}
\expandafter\def\csname PY@tok@s\endcsname{\def\PY@tc##1{\textcolor[rgb]{0.73,0.13,0.13}{##1}}}
\expandafter\def\csname PY@tok@kp\endcsname{\def\PY@tc##1{\textcolor[rgb]{0.00,0.50,0.00}{##1}}}
\expandafter\def\csname PY@tok@w\endcsname{\def\PY@tc##1{\textcolor[rgb]{0.73,0.73,0.73}{##1}}}
\expandafter\def\csname PY@tok@kt\endcsname{\def\PY@tc##1{\textcolor[rgb]{0.69,0.00,0.25}{##1}}}
\expandafter\def\csname PY@tok@sc\endcsname{\def\PY@tc##1{\textcolor[rgb]{0.73,0.13,0.13}{##1}}}
\expandafter\def\csname PY@tok@sb\endcsname{\def\PY@tc##1{\textcolor[rgb]{0.73,0.13,0.13}{##1}}}
\expandafter\def\csname PY@tok@sa\endcsname{\def\PY@tc##1{\textcolor[rgb]{0.73,0.13,0.13}{##1}}}
\expandafter\def\csname PY@tok@k\endcsname{\let\PY@bf=\textbf\def\PY@tc##1{\textcolor[rgb]{0.00,0.50,0.00}{##1}}}
\expandafter\def\csname PY@tok@se\endcsname{\let\PY@bf=\textbf\def\PY@tc##1{\textcolor[rgb]{0.73,0.40,0.13}{##1}}}
\expandafter\def\csname PY@tok@sd\endcsname{\let\PY@it=\textit\def\PY@tc##1{\textcolor[rgb]{0.73,0.13,0.13}{##1}}}

\def\PYZbs{\char`\\}
\def\PYZus{\char`\_}
\def\PYZob{\char`\{}
\def\PYZcb{\char`\}}
\def\PYZca{\char`\^}
\def\PYZam{\char`\&}
\def\PYZlt{\char`\<}
\def\PYZgt{\char`\>}
\def\PYZsh{\char`\#}
\def\PYZpc{\char`\%}
\def\PYZdl{\char`\$}
\def\PYZhy{\char`\-}
\def\PYZsq{\char`\'}
\def\PYZdq{\char`\"}
\def\PYZti{\char`\~}
% for compatibility with earlier versions
\def\PYZat{@}
\def\PYZlb{[}
\def\PYZrb{]}
\makeatother


    % Exact colors from NB
    \definecolor{incolor}{rgb}{0.0, 0.0, 0.5}
    \definecolor{outcolor}{rgb}{0.545, 0.0, 0.0}



    
    % Prevent overflowing lines due to hard-to-break entities
    \sloppy 
    % Setup hyperref package
    \hypersetup{
      breaklinks=true,  % so long urls are correctly broken across lines
      colorlinks=true,
      urlcolor=urlcolor,
      linkcolor=linkcolor,
      citecolor=citecolor,
      }
    % Slightly bigger margins than the latex defaults
    
    \geometry{verbose,tmargin=1in,bmargin=1in,lmargin=1in,rmargin=1in}
    
    

    \begin{document}
    
    
    \maketitle
    
    

    
    \subsection{CNN-Project-Exercise}\label{cnn-project-exercise}

We'll be using the CIFAR-10 dataset, which is very famous dataset for
image recognition!

The CIFAR-10 dataset consists of 60000 32x32 colour images in 10
classes, with 6000 images per class. There are 50000 training images and
10000 test images.

The dataset is divided into five training batches and one test batch,
each with 10000 images. The test batch contains exactly 1000
randomly-selected images from each class. The training batches contain
the remaining images in random order, but some training batches may
contain more images from one class than another. Between them, the
training batches contain exactly 5000 images from each class.

\subsubsection{Follow the Instructions in Bold, if you get stuck
somewhere, view the solutions video! Most of the challenge with this
project is actually dealing with the data and its dimensions, not from
setting up the CNN
itself!}\label{follow-the-instructions-in-bold-if-you-get-stuck-somewhere-view-the-solutions-video-most-of-the-challenge-with-this-project-is-actually-dealing-with-the-data-and-its-dimensions-not-from-setting-up-the-cnn-itself}

    \subsection{Step 0: Get the Data}\label{step-0-get-the-data}

** \emph{Note: If you have trouble with this just watch the solutions
video. This doesn't really have anything to do with the exercise, its
more about setting up your data. Please make sure to watch the solutions
video before posting any QA questions.} **

    ** Download the data for CIFAR from here:
https://www.cs.toronto.edu/\textasciitilde{}kriz/cifar.html **

\textbf{Specifically the CIFAR-10 python version link:
https://www.cs.toronto.edu/\textasciitilde{}kriz/cifar-10-python.tar.gz
}

** Remember the directory you save the file in! **

    \begin{Verbatim}[commandchars=\\\{\}]
{\color{incolor}In [{\color{incolor}1}]:} \PY{c+c1}{\PYZsh{} Put file path as a string here}
        \PY{n}{CIFAR\PYZus{}DIR} \PY{o}{=} \PY{l+s+s1}{\PYZsq{}}\PY{l+s+s1}{C:/Users/tom13/jupyter/FYP/CIFAR\PYZhy{}10/}\PY{l+s+s1}{\PYZsq{}}
\end{Verbatim}


    The archive contains the files data\_batch\_1, data\_batch\_2, ...,
data\_batch\_5, as well as test\_batch. Each of these files is a Python
"pickled" object produced with cPickle.

** Load the Data. Use the Code Below to load the data: **

    \begin{Verbatim}[commandchars=\\\{\}]
{\color{incolor}In [{\color{incolor}2}]:} \PY{k}{def} \PY{n+nf}{unpickle}\PY{p}{(}\PY{n}{file}\PY{p}{)}\PY{p}{:}
            \PY{k+kn}{import} \PY{n+nn}{pickle}
            \PY{k}{with} \PY{n+nb}{open}\PY{p}{(}\PY{n}{file}\PY{p}{,} \PY{l+s+s1}{\PYZsq{}}\PY{l+s+s1}{rb}\PY{l+s+s1}{\PYZsq{}}\PY{p}{)} \PY{k}{as} \PY{n}{fo}\PY{p}{:}
                \PY{n}{cifar\PYZus{}dict} \PY{o}{=} \PY{n}{pickle}\PY{o}{.}\PY{n}{load}\PY{p}{(}\PY{n}{fo}\PY{p}{,} \PY{n}{encoding}\PY{o}{=}\PY{l+s+s1}{\PYZsq{}}\PY{l+s+s1}{bytes}\PY{l+s+s1}{\PYZsq{}}\PY{p}{)}
            \PY{k}{return} \PY{n}{cifar\PYZus{}dict}
\end{Verbatim}


    \begin{Verbatim}[commandchars=\\\{\}]
{\color{incolor}In [{\color{incolor}3}]:} \PY{n}{dirs} \PY{o}{=} \PY{p}{[}\PY{l+s+s1}{\PYZsq{}}\PY{l+s+s1}{batches.meta}\PY{l+s+s1}{\PYZsq{}}\PY{p}{,}\PY{l+s+s1}{\PYZsq{}}\PY{l+s+s1}{data\PYZus{}batch\PYZus{}1}\PY{l+s+s1}{\PYZsq{}}\PY{p}{,}\PY{l+s+s1}{\PYZsq{}}\PY{l+s+s1}{data\PYZus{}batch\PYZus{}2}\PY{l+s+s1}{\PYZsq{}}\PY{p}{,}\PY{l+s+s1}{\PYZsq{}}\PY{l+s+s1}{data\PYZus{}batch\PYZus{}3}\PY{l+s+s1}{\PYZsq{}}\PY{p}{,}\PY{l+s+s1}{\PYZsq{}}\PY{l+s+s1}{data\PYZus{}batch\PYZus{}4}\PY{l+s+s1}{\PYZsq{}}\PY{p}{,}\PY{l+s+s1}{\PYZsq{}}\PY{l+s+s1}{data\PYZus{}batch\PYZus{}5}\PY{l+s+s1}{\PYZsq{}}\PY{p}{,}\PY{l+s+s1}{\PYZsq{}}\PY{l+s+s1}{test\PYZus{}batch}\PY{l+s+s1}{\PYZsq{}}\PY{p}{]}
\end{Verbatim}


    \begin{Verbatim}[commandchars=\\\{\}]
{\color{incolor}In [{\color{incolor}4}]:} \PY{n}{all\PYZus{}data} \PY{o}{=} \PY{p}{[}\PY{l+m+mi}{0}\PY{p}{,}\PY{l+m+mi}{1}\PY{p}{,}\PY{l+m+mi}{2}\PY{p}{,}\PY{l+m+mi}{3}\PY{p}{,}\PY{l+m+mi}{4}\PY{p}{,}\PY{l+m+mi}{5}\PY{p}{,}\PY{l+m+mi}{6}\PY{p}{]}
\end{Verbatim}


    \begin{Verbatim}[commandchars=\\\{\}]
{\color{incolor}In [{\color{incolor}5}]:} \PY{k}{for} \PY{n}{i}\PY{p}{,}\PY{n}{direc} \PY{o+ow}{in} \PY{n+nb}{zip}\PY{p}{(}\PY{n}{all\PYZus{}data}\PY{p}{,}\PY{n}{dirs}\PY{p}{)}\PY{p}{:}
            \PY{n}{all\PYZus{}data}\PY{p}{[}\PY{n}{i}\PY{p}{]} \PY{o}{=} \PY{n}{unpickle}\PY{p}{(}\PY{n}{CIFAR\PYZus{}DIR}\PY{o}{+}\PY{n}{direc}\PY{p}{)}
\end{Verbatim}


    \begin{Verbatim}[commandchars=\\\{\}]
{\color{incolor}In [{\color{incolor}6}]:} \PY{n}{batch\PYZus{}meta} \PY{o}{=} \PY{n}{all\PYZus{}data}\PY{p}{[}\PY{l+m+mi}{0}\PY{p}{]}
        \PY{n}{data\PYZus{}batch1} \PY{o}{=} \PY{n}{all\PYZus{}data}\PY{p}{[}\PY{l+m+mi}{1}\PY{p}{]}
        \PY{n}{data\PYZus{}batch2} \PY{o}{=} \PY{n}{all\PYZus{}data}\PY{p}{[}\PY{l+m+mi}{2}\PY{p}{]}
        \PY{n}{data\PYZus{}batch3} \PY{o}{=} \PY{n}{all\PYZus{}data}\PY{p}{[}\PY{l+m+mi}{3}\PY{p}{]}
        \PY{n}{data\PYZus{}batch4} \PY{o}{=} \PY{n}{all\PYZus{}data}\PY{p}{[}\PY{l+m+mi}{4}\PY{p}{]}
        \PY{n}{data\PYZus{}batch5} \PY{o}{=} \PY{n}{all\PYZus{}data}\PY{p}{[}\PY{l+m+mi}{5}\PY{p}{]}
        \PY{n}{test\PYZus{}batch} \PY{o}{=} \PY{n}{all\PYZus{}data}\PY{p}{[}\PY{l+m+mi}{6}\PY{p}{]}
\end{Verbatim}


    \begin{Verbatim}[commandchars=\\\{\}]
{\color{incolor}In [{\color{incolor}7}]:} \PY{n}{batch\PYZus{}meta}
\end{Verbatim}


\begin{Verbatim}[commandchars=\\\{\}]
{\color{outcolor}Out[{\color{outcolor}7}]:} \{b'label\_names': [b'airplane',
          b'automobile',
          b'bird',
          b'cat',
          b'deer',
          b'dog',
          b'frog',
          b'horse',
          b'ship',
          b'truck'],
         b'num\_cases\_per\_batch': 10000,
         b'num\_vis': 3072\}
\end{Verbatim}
            
    ** Why the 'b's in front of the string? ** Bytes literals are always
prefixed with 'b' or 'B'; they produce an instance of the bytes type
instead of the str type. They may only contain ASCII characters; bytes
with a numeric value of 128 or greater must be expressed with escapes.

https://stackoverflow.com/questions/6269765/what-does-the-b-character-do-in-front-of-a-string-literal

    \begin{Verbatim}[commandchars=\\\{\}]
{\color{incolor}In [{\color{incolor}8}]:} \PY{n}{data\PYZus{}batch1}\PY{o}{.}\PY{n}{keys}\PY{p}{(}\PY{p}{)}
\end{Verbatim}


\begin{Verbatim}[commandchars=\\\{\}]
{\color{outcolor}Out[{\color{outcolor}8}]:} dict\_keys([b'data', b'batch\_label', b'labels', b'filenames'])
\end{Verbatim}
            
    Loaded in this way, each of the batch files contains a dictionary with
the following elements: * data -\/- a 10000x3072 numpy array of uint8s.
Each row of the array stores a 32x32 colour image. The first 1024
entries contain the red channel values, the next 1024 the green, and the
final 1024 the blue. The image is stored in row-major order, so that the
first 32 entries of the array are the red channel values of the first
row of the image. * labels -\/- a list of 10000 numbers in the range
0-9. The number at index i indicates the label of the ith image in the
array data.

The dataset contains another file, called batches.meta. It too contains
a Python dictionary object. It has the following entries:

\begin{itemize}
\tightlist
\item
  label\_names -\/- a 10-element list which gives meaningful names to
  the numeric labels in the labels array described above. For example,
  label\_names{[}0{]} == "airplane", label\_names{[}1{]} ==
  "automobile", etc.
\end{itemize}

    \subsubsection{Display a single image using
matplotlib.}\label{display-a-single-image-using-matplotlib.}

** Grab a single image from data\_batch1 and display it with
plt.imshow(). You'll need to reshape and transpose the numpy array
inside the X = data\_batch{[}b'data'{]} dictionary entry.**

** It should end up looking like this: **

\begin{verbatim}
# Array of all images reshaped and formatted for viewing
X = X.reshape(10000, 3, 32, 32).transpose(0,2,3,1).astype("uint8")
\end{verbatim}

    \begin{Verbatim}[commandchars=\\\{\}]
{\color{incolor}In [{\color{incolor}9}]:} \PY{k+kn}{import} \PY{n+nn}{matplotlib}\PY{n+nn}{.}\PY{n+nn}{pyplot} \PY{k}{as} \PY{n+nn}{plt}
        \PY{o}{\PYZpc{}}\PY{k}{matplotlib} inline
        \PY{k+kn}{import} \PY{n+nn}{numpy} \PY{k}{as} \PY{n+nn}{np}
\end{Verbatim}


    \begin{Verbatim}[commandchars=\\\{\}]
{\color{incolor}In [{\color{incolor}10}]:} \PY{n}{X} \PY{o}{=} \PY{n}{data\PYZus{}batch1}\PY{p}{[}\PY{l+s+sa}{b}\PY{l+s+s2}{\PYZdq{}}\PY{l+s+s2}{data}\PY{l+s+s2}{\PYZdq{}}\PY{p}{]}
\end{Verbatim}


    \begin{Verbatim}[commandchars=\\\{\}]
{\color{incolor}In [{\color{incolor}11}]:} \PY{n}{X}\PY{o}{.}\PY{n}{shape}
\end{Verbatim}


\begin{Verbatim}[commandchars=\\\{\}]
{\color{outcolor}Out[{\color{outcolor}11}]:} (10000, 3072)
\end{Verbatim}
            
    \begin{Verbatim}[commandchars=\\\{\}]
{\color{incolor}In [{\color{incolor}12}]:} \PY{n}{X} \PY{o}{=} \PY{n}{X}\PY{o}{.}\PY{n}{reshape}\PY{p}{(}\PY{l+m+mi}{10000}\PY{p}{,} \PY{l+m+mi}{3}\PY{p}{,} \PY{l+m+mi}{32}\PY{p}{,} \PY{l+m+mi}{32}\PY{p}{)}\PY{o}{.}\PY{n}{transpose}\PY{p}{(}\PY{l+m+mi}{0}\PY{p}{,}\PY{l+m+mi}{2}\PY{p}{,}\PY{l+m+mi}{3}\PY{p}{,}\PY{l+m+mi}{1}\PY{p}{)}\PY{o}{.}\PY{n}{astype}\PY{p}{(}\PY{l+s+s2}{\PYZdq{}}\PY{l+s+s2}{uint8}\PY{l+s+s2}{\PYZdq{}}\PY{p}{)}
\end{Verbatim}


    \begin{Verbatim}[commandchars=\\\{\}]
{\color{incolor}In [{\color{incolor}13}]:} \PY{n}{plt}\PY{o}{.}\PY{n}{imshow}\PY{p}{(}\PY{n}{X}\PY{p}{[}\PY{l+m+mi}{12}\PY{p}{]}\PY{p}{)}
\end{Verbatim}


\begin{Verbatim}[commandchars=\\\{\}]
{\color{outcolor}Out[{\color{outcolor}13}]:} <matplotlib.image.AxesImage at 0x287f0aa3780>
\end{Verbatim}
            
    \begin{center}
    \adjustimage{max size={0.9\linewidth}{0.9\paperheight}}{02-CNN-Project-Exercise_files/02-CNN-Project-Exercise_19_1.png}
    \end{center}
    { \hspace*{\fill} \\}
    
    \begin{Verbatim}[commandchars=\\\{\}]
{\color{incolor}In [{\color{incolor}14}]:} \PY{n}{plt}\PY{o}{.}\PY{n}{imshow}\PY{p}{(}\PY{n}{X}\PY{p}{[}\PY{l+m+mi}{1}\PY{p}{]}\PY{p}{)}
\end{Verbatim}


\begin{Verbatim}[commandchars=\\\{\}]
{\color{outcolor}Out[{\color{outcolor}14}]:} <matplotlib.image.AxesImage at 0x287f0b0e390>
\end{Verbatim}
            
    \begin{center}
    \adjustimage{max size={0.9\linewidth}{0.9\paperheight}}{02-CNN-Project-Exercise_files/02-CNN-Project-Exercise_20_1.png}
    \end{center}
    { \hspace*{\fill} \\}
    
    \begin{Verbatim}[commandchars=\\\{\}]
{\color{incolor}In [{\color{incolor}15}]:} \PY{n}{plt}\PY{o}{.}\PY{n}{imshow}\PY{p}{(}\PY{n}{X}\PY{p}{[}\PY{l+m+mi}{4}\PY{p}{]}\PY{p}{)}
\end{Verbatim}


\begin{Verbatim}[commandchars=\\\{\}]
{\color{outcolor}Out[{\color{outcolor}15}]:} <matplotlib.image.AxesImage at 0x287f0b53390>
\end{Verbatim}
            
    \begin{center}
    \adjustimage{max size={0.9\linewidth}{0.9\paperheight}}{02-CNN-Project-Exercise_files/02-CNN-Project-Exercise_21_1.png}
    \end{center}
    { \hspace*{\fill} \\}
    
    \section{Helper Functions for Dealing With
Data.}\label{helper-functions-for-dealing-with-data.}

** Use the provided code below to help with dealing with grabbing the
next batch once you've gotten ready to create the Graph Session. Can you
break down how it works? **

    \begin{Verbatim}[commandchars=\\\{\}]
{\color{incolor}In [{\color{incolor}55}]:} \PY{k}{def} \PY{n+nf}{one\PYZus{}hot\PYZus{}encode}\PY{p}{(}\PY{n}{vec}\PY{p}{,} \PY{n}{vals}\PY{o}{=}\PY{l+m+mi}{10}\PY{p}{)}\PY{p}{:}
             \PY{l+s+sd}{\PYZsq{}\PYZsq{}\PYZsq{}}
         \PY{l+s+sd}{    For use to one\PYZhy{}hot encode the 10\PYZhy{} possible labels}
         \PY{l+s+sd}{    \PYZsq{}\PYZsq{}\PYZsq{}}
             \PY{n}{n} \PY{o}{=} \PY{n+nb}{len}\PY{p}{(}\PY{n}{vec}\PY{p}{)}
             \PY{n}{out} \PY{o}{=} \PY{n}{np}\PY{o}{.}\PY{n}{zeros}\PY{p}{(}\PY{p}{(}\PY{n}{n}\PY{p}{,} \PY{n}{vals}\PY{p}{)}\PY{p}{)}
             \PY{n}{out}\PY{p}{[}\PY{n+nb}{range}\PY{p}{(}\PY{n}{n}\PY{p}{)}\PY{p}{,} \PY{n}{vec}\PY{p}{]} \PY{o}{=} \PY{l+m+mi}{1}
             \PY{k}{return} \PY{n}{out}
\end{Verbatim}


    \begin{Verbatim}[commandchars=\\\{\}]
{\color{incolor}In [{\color{incolor}56}]:} \PY{k}{class} \PY{n+nc}{CifarHelper}\PY{p}{(}\PY{p}{)}\PY{p}{:}
             
             \PY{k}{def} \PY{n+nf}{\PYZus{}\PYZus{}init\PYZus{}\PYZus{}}\PY{p}{(}\PY{n+nb+bp}{self}\PY{p}{)}\PY{p}{:}
                 \PY{n+nb+bp}{self}\PY{o}{.}\PY{n}{i} \PY{o}{=} \PY{l+m+mi}{0}
                 
                 \PY{c+c1}{\PYZsh{} Grabs a list of all the data batches for training}
                 \PY{n+nb+bp}{self}\PY{o}{.}\PY{n}{all\PYZus{}train\PYZus{}batches} \PY{o}{=} \PY{p}{[}\PY{n}{data\PYZus{}batch1}\PY{p}{,}\PY{n}{data\PYZus{}batch2}\PY{p}{,}\PY{n}{data\PYZus{}batch3}\PY{p}{,}\PY{n}{data\PYZus{}batch4}\PY{p}{,}\PY{n}{data\PYZus{}batch5}\PY{p}{]}
                 \PY{c+c1}{\PYZsh{} Grabs a list of all the test batches (really just one batch)}
                 \PY{n+nb+bp}{self}\PY{o}{.}\PY{n}{test\PYZus{}batch} \PY{o}{=} \PY{p}{[}\PY{n}{test\PYZus{}batch}\PY{p}{]}
                 
                 \PY{c+c1}{\PYZsh{} Intialize some empty variables for later on}
                 \PY{n+nb+bp}{self}\PY{o}{.}\PY{n}{training\PYZus{}images} \PY{o}{=} \PY{k+kc}{None}
                 \PY{n+nb+bp}{self}\PY{o}{.}\PY{n}{training\PYZus{}labels} \PY{o}{=} \PY{k+kc}{None}
                 
                 \PY{n+nb+bp}{self}\PY{o}{.}\PY{n}{test\PYZus{}images} \PY{o}{=} \PY{k+kc}{None}
                 \PY{n+nb+bp}{self}\PY{o}{.}\PY{n}{test\PYZus{}labels} \PY{o}{=} \PY{k+kc}{None}
             
             \PY{k}{def} \PY{n+nf}{set\PYZus{}up\PYZus{}images}\PY{p}{(}\PY{n+nb+bp}{self}\PY{p}{)}\PY{p}{:}
                 
                 \PY{n+nb}{print}\PY{p}{(}\PY{l+s+s2}{\PYZdq{}}\PY{l+s+s2}{Setting Up Training Images and Labels}\PY{l+s+s2}{\PYZdq{}}\PY{p}{)}
                 
                 \PY{c+c1}{\PYZsh{} Vertically stacks the training images}
                 \PY{n+nb+bp}{self}\PY{o}{.}\PY{n}{training\PYZus{}images} \PY{o}{=} \PY{n}{np}\PY{o}{.}\PY{n}{vstack}\PY{p}{(}\PY{p}{[}\PY{n}{d}\PY{p}{[}\PY{l+s+sa}{b}\PY{l+s+s2}{\PYZdq{}}\PY{l+s+s2}{data}\PY{l+s+s2}{\PYZdq{}}\PY{p}{]} \PY{k}{for} \PY{n}{d} \PY{o+ow}{in} \PY{n+nb+bp}{self}\PY{o}{.}\PY{n}{all\PYZus{}train\PYZus{}batches}\PY{p}{]}\PY{p}{)}
                 \PY{n}{train\PYZus{}len} \PY{o}{=} \PY{n+nb}{len}\PY{p}{(}\PY{n+nb+bp}{self}\PY{o}{.}\PY{n}{training\PYZus{}images}\PY{p}{)}
                 
                 \PY{c+c1}{\PYZsh{} Reshapes and normalizes training images}
                 \PY{n+nb+bp}{self}\PY{o}{.}\PY{n}{training\PYZus{}images} \PY{o}{=} \PY{n+nb+bp}{self}\PY{o}{.}\PY{n}{training\PYZus{}images}\PY{o}{.}\PY{n}{reshape}\PY{p}{(}\PY{n}{train\PYZus{}len}\PY{p}{,}\PY{l+m+mi}{3}\PY{p}{,}\PY{l+m+mi}{32}\PY{p}{,}\PY{l+m+mi}{32}\PY{p}{)}\PY{o}{.}\PY{n}{transpose}\PY{p}{(}\PY{l+m+mi}{0}\PY{p}{,}\PY{l+m+mi}{2}\PY{p}{,}\PY{l+m+mi}{3}\PY{p}{,}\PY{l+m+mi}{1}\PY{p}{)}\PY{o}{/}\PY{l+m+mi}{255}
                 \PY{c+c1}{\PYZsh{} One hot Encodes the training labels (e.g. [0,0,0,1,0,0,0,0,0,0])}
                 \PY{n+nb+bp}{self}\PY{o}{.}\PY{n}{training\PYZus{}labels} \PY{o}{=} \PY{n}{one\PYZus{}hot\PYZus{}encode}\PY{p}{(}\PY{n}{np}\PY{o}{.}\PY{n}{hstack}\PY{p}{(}\PY{p}{[}\PY{n}{d}\PY{p}{[}\PY{l+s+sa}{b}\PY{l+s+s2}{\PYZdq{}}\PY{l+s+s2}{labels}\PY{l+s+s2}{\PYZdq{}}\PY{p}{]} \PY{k}{for} \PY{n}{d} \PY{o+ow}{in} \PY{n+nb+bp}{self}\PY{o}{.}\PY{n}{all\PYZus{}train\PYZus{}batches}\PY{p}{]}\PY{p}{)}\PY{p}{,} \PY{l+m+mi}{10}\PY{p}{)}
                 
                 \PY{n+nb}{print}\PY{p}{(}\PY{l+s+s2}{\PYZdq{}}\PY{l+s+s2}{Setting Up Test Images and Labels}\PY{l+s+s2}{\PYZdq{}}\PY{p}{)}
                 
                 \PY{c+c1}{\PYZsh{} Vertically stacks the test images}
                 \PY{n+nb+bp}{self}\PY{o}{.}\PY{n}{test\PYZus{}images} \PY{o}{=} \PY{n}{np}\PY{o}{.}\PY{n}{vstack}\PY{p}{(}\PY{p}{[}\PY{n}{d}\PY{p}{[}\PY{l+s+sa}{b}\PY{l+s+s2}{\PYZdq{}}\PY{l+s+s2}{data}\PY{l+s+s2}{\PYZdq{}}\PY{p}{]} \PY{k}{for} \PY{n}{d} \PY{o+ow}{in} \PY{n+nb+bp}{self}\PY{o}{.}\PY{n}{test\PYZus{}batch}\PY{p}{]}\PY{p}{)}
                 \PY{n}{test\PYZus{}len} \PY{o}{=} \PY{n+nb}{len}\PY{p}{(}\PY{n+nb+bp}{self}\PY{o}{.}\PY{n}{test\PYZus{}images}\PY{p}{)}
                 
                 \PY{c+c1}{\PYZsh{} Reshapes and normalizes test images}
                 \PY{n+nb+bp}{self}\PY{o}{.}\PY{n}{test\PYZus{}images} \PY{o}{=} \PY{n+nb+bp}{self}\PY{o}{.}\PY{n}{test\PYZus{}images}\PY{o}{.}\PY{n}{reshape}\PY{p}{(}\PY{n}{test\PYZus{}len}\PY{p}{,}\PY{l+m+mi}{3}\PY{p}{,}\PY{l+m+mi}{32}\PY{p}{,}\PY{l+m+mi}{32}\PY{p}{)}\PY{o}{.}\PY{n}{transpose}\PY{p}{(}\PY{l+m+mi}{0}\PY{p}{,}\PY{l+m+mi}{2}\PY{p}{,}\PY{l+m+mi}{3}\PY{p}{,}\PY{l+m+mi}{1}\PY{p}{)}\PY{o}{/}\PY{l+m+mi}{255}
                 \PY{c+c1}{\PYZsh{} One hot Encodes the test labels (e.g. [0,0,0,1,0,0,0,0,0,0])}
                 \PY{n+nb+bp}{self}\PY{o}{.}\PY{n}{test\PYZus{}labels} \PY{o}{=} \PY{n}{one\PYZus{}hot\PYZus{}encode}\PY{p}{(}\PY{n}{np}\PY{o}{.}\PY{n}{hstack}\PY{p}{(}\PY{p}{[}\PY{n}{d}\PY{p}{[}\PY{l+s+sa}{b}\PY{l+s+s2}{\PYZdq{}}\PY{l+s+s2}{labels}\PY{l+s+s2}{\PYZdq{}}\PY{p}{]} \PY{k}{for} \PY{n}{d} \PY{o+ow}{in} \PY{n+nb+bp}{self}\PY{o}{.}\PY{n}{test\PYZus{}batch}\PY{p}{]}\PY{p}{)}\PY{p}{,} \PY{l+m+mi}{10}\PY{p}{)}
         
                 
             \PY{k}{def} \PY{n+nf}{next\PYZus{}batch}\PY{p}{(}\PY{n+nb+bp}{self}\PY{p}{,} \PY{n}{batch\PYZus{}size}\PY{p}{)}\PY{p}{:}
                 \PY{c+c1}{\PYZsh{} Note that the 100 dimension in the reshape call is set by an assumed batch size of 100}
                 \PY{n}{x} \PY{o}{=} \PY{n+nb+bp}{self}\PY{o}{.}\PY{n}{training\PYZus{}images}\PY{p}{[}\PY{n+nb+bp}{self}\PY{o}{.}\PY{n}{i}\PY{p}{:}\PY{n+nb+bp}{self}\PY{o}{.}\PY{n}{i}\PY{o}{+}\PY{n}{batch\PYZus{}size}\PY{p}{]}\PY{o}{.}\PY{n}{reshape}\PY{p}{(}\PY{l+m+mi}{100}\PY{p}{,}\PY{l+m+mi}{32}\PY{p}{,}\PY{l+m+mi}{32}\PY{p}{,}\PY{l+m+mi}{3}\PY{p}{)}
                 \PY{n}{y} \PY{o}{=} \PY{n+nb+bp}{self}\PY{o}{.}\PY{n}{training\PYZus{}labels}\PY{p}{[}\PY{n+nb+bp}{self}\PY{o}{.}\PY{n}{i}\PY{p}{:}\PY{n+nb+bp}{self}\PY{o}{.}\PY{n}{i}\PY{o}{+}\PY{n}{batch\PYZus{}size}\PY{p}{]}
                 \PY{n+nb+bp}{self}\PY{o}{.}\PY{n}{i} \PY{o}{=} \PY{p}{(}\PY{n+nb+bp}{self}\PY{o}{.}\PY{n}{i} \PY{o}{+} \PY{n}{batch\PYZus{}size}\PY{p}{)} \PY{o}{\PYZpc{}} \PY{n+nb}{len}\PY{p}{(}\PY{n+nb+bp}{self}\PY{o}{.}\PY{n}{training\PYZus{}images}\PY{p}{)}
                 \PY{k}{return} \PY{n}{x}\PY{p}{,} \PY{n}{y}
             
             \PY{k}{def} \PY{n+nf}{next\PYZus{}test\PYZus{}batch}\PY{p}{(}\PY{n+nb+bp}{self}\PY{p}{,} \PY{n}{batch\PYZus{}size}\PY{p}{)}\PY{p}{:}
                 \PY{c+c1}{\PYZsh{} Note that the 100 dimension in the reshape call is set by an assumed batch size of 100}
                 \PY{n}{x} \PY{o}{=} \PY{n+nb+bp}{self}\PY{o}{.}\PY{n}{test\PYZus{}images}\PY{p}{[}\PY{n+nb+bp}{self}\PY{o}{.}\PY{n}{i}\PY{p}{:}\PY{n+nb+bp}{self}\PY{o}{.}\PY{n}{i}\PY{o}{+}\PY{n}{batch\PYZus{}size}\PY{p}{]}\PY{o}{.}\PY{n}{reshape}\PY{p}{(}\PY{l+m+mi}{100}\PY{p}{,}\PY{l+m+mi}{32}\PY{p}{,}\PY{l+m+mi}{32}\PY{p}{,}\PY{l+m+mi}{3}\PY{p}{)}
                 \PY{n}{y} \PY{o}{=} \PY{n+nb+bp}{self}\PY{o}{.}\PY{n}{test\PYZus{}labels}\PY{p}{[}\PY{n+nb+bp}{self}\PY{o}{.}\PY{n}{i}\PY{p}{:}\PY{n+nb+bp}{self}\PY{o}{.}\PY{n}{i}\PY{o}{+}\PY{n}{batch\PYZus{}size}\PY{p}{]}
                 \PY{n+nb+bp}{self}\PY{o}{.}\PY{n}{i} \PY{o}{=} \PY{p}{(}\PY{n+nb+bp}{self}\PY{o}{.}\PY{n}{i} \PY{o}{+} \PY{n}{batch\PYZus{}size}\PY{p}{)} \PY{o}{\PYZpc{}} \PY{n+nb}{len}\PY{p}{(}\PY{n+nb+bp}{self}\PY{o}{.}\PY{n}{test\PYZus{}images}\PY{p}{)}
                 \PY{k}{return} \PY{n}{x}\PY{p}{,} \PY{n}{y}
\end{Verbatim}


    ** How to use the above code: **

    \begin{Verbatim}[commandchars=\\\{\}]
{\color{incolor}In [{\color{incolor}57}]:} \PY{c+c1}{\PYZsh{} Before Your tf.Session run these two lines}
         \PY{n}{ch} \PY{o}{=} \PY{n}{CifarHelper}\PY{p}{(}\PY{p}{)}
         \PY{n}{ch}\PY{o}{.}\PY{n}{set\PYZus{}up\PYZus{}images}\PY{p}{(}\PY{p}{)}
         
         \PY{c+c1}{\PYZsh{} During your session to grab the next batch use this line}
         \PY{c+c1}{\PYZsh{}batch = ch.next\PYZus{}batch(100)}
\end{Verbatim}


    \begin{Verbatim}[commandchars=\\\{\}]
Setting Up Training Images and Labels
Setting Up Test Images and Labels

    \end{Verbatim}

    \subsection{Creating the Model}\label{creating-the-model}

** Import tensorflow **

    \begin{Verbatim}[commandchars=\\\{\}]
{\color{incolor}In [{\color{incolor}58}]:} \PY{k+kn}{import} \PY{n+nn}{tensorflow} \PY{k}{as} \PY{n+nn}{tf}
\end{Verbatim}


    \begin{Verbatim}[commandchars=\\\{\}]
{\color{incolor}In [{\color{incolor}59}]:} \PY{n}{x} \PY{o}{=} \PY{n}{tf}\PY{o}{.}\PY{n}{placeholder}\PY{p}{(}\PY{n}{tf}\PY{o}{.}\PY{n}{float32}\PY{p}{,} \PY{n}{shape} \PY{o}{=} \PY{p}{[}\PY{k+kc}{None}\PY{p}{,} \PY{l+m+mi}{32}\PY{p}{,} \PY{l+m+mi}{32}\PY{p}{,} \PY{l+m+mi}{3}\PY{p}{]}\PY{p}{)}
         \PY{n}{y\PYZus{}true} \PY{o}{=} \PY{n}{tf}\PY{o}{.}\PY{n}{placeholder}\PY{p}{(}\PY{n}{tf}\PY{o}{.}\PY{n}{float32}\PY{p}{,} \PY{n}{shape} \PY{o}{=} \PY{p}{[}\PY{k+kc}{None}\PY{p}{,} \PY{l+m+mi}{10}\PY{p}{]}\PY{p}{)}
         \PY{n}{hold\PYZus{}prob} \PY{o}{=} \PY{n}{tf}\PY{o}{.}\PY{n}{placeholder}\PY{p}{(}\PY{n}{tf}\PY{o}{.}\PY{n}{float32}\PY{p}{)}
\end{Verbatim}


    \subsubsection{Helper Functions}\label{helper-functions}

** Grab the helper functions from MNIST with CNN (or recreate them here
yourself for a hard challenge!). You'll need: **

\begin{itemize}
\tightlist
\item
  init\_weights
\item
  init\_bias
\item
  conv2d
\item
  max\_pool\_2by2
\item
  convolutional\_layer
\item
  normal\_full\_layer
\end{itemize}

    \begin{Verbatim}[commandchars=\\\{\}]
{\color{incolor}In [{\color{incolor}60}]:} \PY{k}{def} \PY{n+nf}{init\PYZus{}weights}\PY{p}{(}\PY{n}{shape}\PY{p}{)}\PY{p}{:}
             \PY{n}{init\PYZus{}random\PYZus{}dist} \PY{o}{=} \PY{n}{tf}\PY{o}{.}\PY{n}{truncated\PYZus{}normal}\PY{p}{(}\PY{n}{shape}\PY{p}{,} \PY{n}{stddev}\PY{o}{=}\PY{l+m+mf}{0.1}\PY{p}{)}
             \PY{k}{return} \PY{n}{tf}\PY{o}{.}\PY{n}{Variable}\PY{p}{(}\PY{n}{init\PYZus{}random\PYZus{}dist}\PY{p}{)}
         
         \PY{k}{def} \PY{n+nf}{init\PYZus{}bias}\PY{p}{(}\PY{n}{shape}\PY{p}{)}\PY{p}{:}
             \PY{n}{init\PYZus{}bias\PYZus{}vals} \PY{o}{=} \PY{n}{tf}\PY{o}{.}\PY{n}{constant}\PY{p}{(}\PY{l+m+mf}{0.1}\PY{p}{,} \PY{n}{shape}\PY{o}{=}\PY{n}{shape}\PY{p}{)}
             \PY{k}{return} \PY{n}{tf}\PY{o}{.}\PY{n}{Variable}\PY{p}{(}\PY{n}{init\PYZus{}bias\PYZus{}vals}\PY{p}{)}
         
         \PY{k}{def} \PY{n+nf}{conv2d}\PY{p}{(}\PY{n}{x}\PY{p}{,} \PY{n}{W}\PY{p}{)}\PY{p}{:}
             \PY{k}{return} \PY{n}{tf}\PY{o}{.}\PY{n}{nn}\PY{o}{.}\PY{n}{conv2d}\PY{p}{(}\PY{n}{x}\PY{p}{,} \PY{n}{W}\PY{p}{,} \PY{n}{strides}\PY{o}{=}\PY{p}{[}\PY{l+m+mi}{1}\PY{p}{,} \PY{l+m+mi}{1}\PY{p}{,} \PY{l+m+mi}{1}\PY{p}{,} \PY{l+m+mi}{1}\PY{p}{]}\PY{p}{,} \PY{n}{padding}\PY{o}{=}\PY{l+s+s1}{\PYZsq{}}\PY{l+s+s1}{SAME}\PY{l+s+s1}{\PYZsq{}}\PY{p}{)}
         
         \PY{k}{def} \PY{n+nf}{max\PYZus{}pool\PYZus{}2by2}\PY{p}{(}\PY{n}{x}\PY{p}{)}\PY{p}{:}
             \PY{k}{return} \PY{n}{tf}\PY{o}{.}\PY{n}{nn}\PY{o}{.}\PY{n}{max\PYZus{}pool}\PY{p}{(}\PY{n}{x}\PY{p}{,} \PY{n}{ksize}\PY{o}{=}\PY{p}{[}\PY{l+m+mi}{1}\PY{p}{,} \PY{l+m+mi}{2}\PY{p}{,} \PY{l+m+mi}{2}\PY{p}{,} \PY{l+m+mi}{1}\PY{p}{]}\PY{p}{,}
                                   \PY{n}{strides}\PY{o}{=}\PY{p}{[}\PY{l+m+mi}{1}\PY{p}{,} \PY{l+m+mi}{2}\PY{p}{,} \PY{l+m+mi}{2}\PY{p}{,} \PY{l+m+mi}{1}\PY{p}{]}\PY{p}{,} \PY{n}{padding}\PY{o}{=}\PY{l+s+s1}{\PYZsq{}}\PY{l+s+s1}{SAME}\PY{l+s+s1}{\PYZsq{}}\PY{p}{)}
         
         \PY{k}{def} \PY{n+nf}{convolutional\PYZus{}layer}\PY{p}{(}\PY{n}{input\PYZus{}x}\PY{p}{,} \PY{n}{shape}\PY{p}{)}\PY{p}{:}
             \PY{n}{W} \PY{o}{=} \PY{n}{init\PYZus{}weights}\PY{p}{(}\PY{n}{shape}\PY{p}{)}
             \PY{n}{b} \PY{o}{=} \PY{n}{init\PYZus{}bias}\PY{p}{(}\PY{p}{[}\PY{n}{shape}\PY{p}{[}\PY{l+m+mi}{3}\PY{p}{]}\PY{p}{]}\PY{p}{)}
             \PY{k}{return} \PY{n}{tf}\PY{o}{.}\PY{n}{nn}\PY{o}{.}\PY{n}{relu}\PY{p}{(}\PY{n}{conv2d}\PY{p}{(}\PY{n}{input\PYZus{}x}\PY{p}{,} \PY{n}{W}\PY{p}{)} \PY{o}{+} \PY{n}{b}\PY{p}{)}
         
         \PY{k}{def} \PY{n+nf}{normal\PYZus{}full\PYZus{}layer}\PY{p}{(}\PY{n}{input\PYZus{}layer}\PY{p}{,} \PY{n}{size}\PY{p}{)}\PY{p}{:}
             \PY{n}{input\PYZus{}size} \PY{o}{=} \PY{n+nb}{int}\PY{p}{(}\PY{n}{input\PYZus{}layer}\PY{o}{.}\PY{n}{get\PYZus{}shape}\PY{p}{(}\PY{p}{)}\PY{p}{[}\PY{l+m+mi}{1}\PY{p}{]}\PY{p}{)}
             \PY{n}{W} \PY{o}{=} \PY{n}{init\PYZus{}weights}\PY{p}{(}\PY{p}{[}\PY{n}{input\PYZus{}size}\PY{p}{,} \PY{n}{size}\PY{p}{]}\PY{p}{)}
             \PY{n}{b} \PY{o}{=} \PY{n}{init\PYZus{}bias}\PY{p}{(}\PY{p}{[}\PY{n}{size}\PY{p}{]}\PY{p}{)}
             \PY{k}{return} \PY{n}{tf}\PY{o}{.}\PY{n}{matmul}\PY{p}{(}\PY{n}{input\PYZus{}layer}\PY{p}{,} \PY{n}{W}\PY{p}{)} \PY{o}{+} \PY{n}{b}
\end{Verbatim}


    \subsubsection{Create the Layers}\label{create-the-layers}

** Create a convolutional layer and a pooling layer as we did for MNIST.
\textbf{ } Its up to you what the 2d size of the convolution should be,
but the last two digits need to be 3 and 32 because of the 3 color
channels and 32 pixels. So for example you could use:**

\begin{verbatim}
    convo_1 = convolutional_layer(x,shape=[4,4,3,32])
\end{verbatim}

    \begin{Verbatim}[commandchars=\\\{\}]
{\color{incolor}In [{\color{incolor}61}]:} \PY{n}{convo\PYZus{}1} \PY{o}{=} \PY{n}{convolutional\PYZus{}layer}\PY{p}{(}\PY{n}{x}\PY{p}{,} \PY{n}{shape} \PY{o}{=} \PY{p}{[}\PY{l+m+mi}{4}\PY{p}{,} \PY{l+m+mi}{4}\PY{p}{,} \PY{l+m+mi}{3}\PY{p}{,} \PY{l+m+mi}{32}\PY{p}{]}\PY{p}{)}
         \PY{n}{convo\PYZus{}1\PYZus{}pooling} \PY{o}{=} \PY{n}{max\PYZus{}pool\PYZus{}2by2}\PY{p}{(}\PY{n}{convo\PYZus{}1}\PY{p}{)}
\end{Verbatim}


    ** Create the next convolutional and pooling layers. The last two
dimensions of the convo\_2 layer should be 32,64 **

    \begin{Verbatim}[commandchars=\\\{\}]
{\color{incolor}In [{\color{incolor}62}]:} \PY{n}{convo\PYZus{}2} \PY{o}{=} \PY{n}{convolutional\PYZus{}layer}\PY{p}{(}\PY{n}{convo\PYZus{}1\PYZus{}pooling}\PY{p}{,} \PY{n}{shape} \PY{o}{=} \PY{p}{[}\PY{l+m+mi}{4}\PY{p}{,} \PY{l+m+mi}{4}\PY{p}{,} \PY{l+m+mi}{32}\PY{p}{,} \PY{l+m+mi}{64}\PY{p}{]}\PY{p}{)}
         \PY{n}{convo\PYZus{}2\PYZus{}pooling} \PY{o}{=} \PY{n}{max\PYZus{}pool\PYZus{}2by2}\PY{p}{(}\PY{n}{convo\PYZus{}2}\PY{p}{)}
\end{Verbatim}


    ** Now create a flattened layer by reshaping the pooling layer into
{[}-1,8 * 8 * 64{]} or {[}-1,4096{]} **

    \begin{Verbatim}[commandchars=\\\{\}]
{\color{incolor}In [{\color{incolor}63}]:} \PY{n}{convo\PYZus{}2\PYZus{}flat} \PY{o}{=} \PY{n}{tf}\PY{o}{.}\PY{n}{reshape}\PY{p}{(}\PY{n}{convo\PYZus{}2\PYZus{}pooling}\PY{p}{,} \PY{p}{[}\PY{o}{\PYZhy{}}\PY{l+m+mi}{1}\PY{p}{,} \PY{l+m+mi}{4096}\PY{p}{]}\PY{p}{)}
\end{Verbatim}


    ** Create a new full layer using the normal\_full\_layer function and
passing in your flattend convolutional 2 layer with size=1024. (You
could also choose to reduce this to something like 512)**

    \begin{Verbatim}[commandchars=\\\{\}]
{\color{incolor}In [{\color{incolor}64}]:} \PY{n}{full\PYZus{}layer\PYZus{}one} \PY{o}{=} \PY{n}{tf}\PY{o}{.}\PY{n}{nn}\PY{o}{.}\PY{n}{relu}\PY{p}{(}\PY{n}{normal\PYZus{}full\PYZus{}layer}\PY{p}{(}\PY{n}{convo\PYZus{}2\PYZus{}flat}\PY{p}{,} \PY{l+m+mi}{1024}\PY{p}{)}\PY{p}{)}
\end{Verbatim}


    ** Now create the dropout layer with tf.nn.dropout, remember to pass in
your hold\_prob placeholder. **

    \begin{Verbatim}[commandchars=\\\{\}]
{\color{incolor}In [{\color{incolor}65}]:} \PY{n}{full\PYZus{}one\PYZus{}dropout} \PY{o}{=} \PY{n}{tf}\PY{o}{.}\PY{n}{nn}\PY{o}{.}\PY{n}{dropout}\PY{p}{(}\PY{n}{full\PYZus{}layer\PYZus{}one}\PY{p}{,} \PY{n}{keep\PYZus{}prob} \PY{o}{=} \PY{n}{hold\PYZus{}prob}\PY{p}{)}
\end{Verbatim}


    ** Finally set the output to y\_pred by passing in the dropout layer
into the normal\_full\_layer function. The size should be 10 because of
the 10 possible labels**

    \begin{Verbatim}[commandchars=\\\{\}]
{\color{incolor}In [{\color{incolor}66}]:} \PY{n}{y\PYZus{}pred} \PY{o}{=} \PY{n}{normal\PYZus{}full\PYZus{}layer}\PY{p}{(}\PY{n}{full\PYZus{}one\PYZus{}dropout}\PY{p}{,} \PY{l+m+mi}{10}\PY{p}{)}
\end{Verbatim}


    \subsubsection{Loss Function}\label{loss-function}

** Create a cross\_entropy loss function **

    \begin{Verbatim}[commandchars=\\\{\}]
{\color{incolor}In [{\color{incolor}67}]:} \PY{n}{cross\PYZus{}entropy} \PY{o}{=} \PY{n}{tf}\PY{o}{.}\PY{n}{reduce\PYZus{}mean}\PY{p}{(}\PY{n}{tf}\PY{o}{.}\PY{n}{nn}\PY{o}{.}\PY{n}{softmax\PYZus{}cross\PYZus{}entropy\PYZus{}with\PYZus{}logits}\PY{p}{(}\PY{n}{labels} \PY{o}{=} \PY{n}{y\PYZus{}true}\PY{p}{,} \PY{n}{logits} \PY{o}{=} \PY{n}{y\PYZus{}pred}\PY{p}{)}\PY{p}{)}
\end{Verbatim}


    \subsubsection{Optimizer}\label{optimizer}

** Create the optimizer using an Adam Optimizer. **

    \begin{Verbatim}[commandchars=\\\{\}]
{\color{incolor}In [{\color{incolor}68}]:} \PY{n}{optimizer} \PY{o}{=} \PY{n}{tf}\PY{o}{.}\PY{n}{train}\PY{o}{.}\PY{n}{AdamOptimizer}\PY{p}{(}\PY{n}{learning\PYZus{}rate} \PY{o}{=} \PY{l+m+mf}{0.001}\PY{p}{)}
         \PY{n}{train} \PY{o}{=} \PY{n}{optimizer}\PY{o}{.}\PY{n}{minimize}\PY{p}{(}\PY{n}{cross\PYZus{}entropy}\PY{p}{)}
\end{Verbatim}


    ** Create a variable to intialize all the global tf variables. **

    \begin{Verbatim}[commandchars=\\\{\}]
{\color{incolor}In [{\color{incolor}69}]:} \PY{n}{init} \PY{o}{=} \PY{n}{tf}\PY{o}{.}\PY{n}{global\PYZus{}variables\PYZus{}initializer}\PY{p}{(}\PY{p}{)}
\end{Verbatim}


    \subsection{Graph Session}\label{graph-session}

** Perform the training and test print outs in a Tf session and run your
model! **

    \begin{Verbatim}[commandchars=\\\{\}]
{\color{incolor}In [{\color{incolor}73}]:} \PY{k}{with} \PY{n}{tf}\PY{o}{.}\PY{n}{Session}\PY{p}{(}\PY{p}{)} \PY{k}{as} \PY{n}{sess}\PY{p}{:}
             \PY{n}{sess}\PY{o}{.}\PY{n}{run}\PY{p}{(}\PY{n}{tf}\PY{o}{.}\PY{n}{global\PYZus{}variables\PYZus{}initializer}\PY{p}{(}\PY{p}{)}\PY{p}{)}
         
             \PY{k}{for} \PY{n}{i} \PY{o+ow}{in} \PY{n+nb}{range}\PY{p}{(}\PY{l+m+mi}{10000}\PY{p}{)}\PY{p}{:}
                 \PY{n}{batch} \PY{o}{=} \PY{n}{ch}\PY{o}{.}\PY{n}{next\PYZus{}batch}\PY{p}{(}\PY{l+m+mi}{100}\PY{p}{)}
                 \PY{n}{sess}\PY{o}{.}\PY{n}{run}\PY{p}{(}\PY{n}{train}\PY{p}{,} \PY{n}{feed\PYZus{}dict}\PY{o}{=}\PY{p}{\PYZob{}}\PY{n}{x}\PY{p}{:} \PY{n}{batch}\PY{p}{[}\PY{l+m+mi}{0}\PY{p}{]}\PY{p}{,} \PY{n}{y\PYZus{}true}\PY{p}{:} \PY{n}{batch}\PY{p}{[}\PY{l+m+mi}{1}\PY{p}{]}\PY{p}{,} \PY{n}{hold\PYZus{}prob}\PY{p}{:} \PY{l+m+mf}{0.5}\PY{p}{\PYZcb{}}\PY{p}{)}
                 
                 \PY{c+c1}{\PYZsh{} PRINT OUT A MESSAGE EVERY 100 STEPS}
                 \PY{k}{if} \PY{n}{i}\PY{o}{\PYZpc{}}\PY{k}{1000} == 0:
                     
                     \PY{c+c1}{\PYZsh{} Test the Train Model}
                     \PY{n}{matches} \PY{o}{=} \PY{n}{tf}\PY{o}{.}\PY{n}{equal}\PY{p}{(}\PY{n}{tf}\PY{o}{.}\PY{n}{argmax}\PY{p}{(}\PY{n}{y\PYZus{}pred}\PY{p}{,}\PY{l+m+mi}{1}\PY{p}{)}\PY{p}{,}\PY{n}{tf}\PY{o}{.}\PY{n}{argmax}\PY{p}{(}\PY{n}{y\PYZus{}true}\PY{p}{,}\PY{l+m+mi}{1}\PY{p}{)}\PY{p}{)}
         
                     \PY{n}{acc} \PY{o}{=} \PY{n}{tf}\PY{o}{.}\PY{n}{reduce\PYZus{}mean}\PY{p}{(}\PY{n}{tf}\PY{o}{.}\PY{n}{cast}\PY{p}{(}\PY{n}{matches}\PY{p}{,}\PY{n}{tf}\PY{o}{.}\PY{n}{float32}\PY{p}{)}\PY{p}{)}
                     \PY{n}{testSet} \PY{o}{=} \PY{n}{ch}\PY{o}{.}\PY{n}{next\PYZus{}test\PYZus{}batch}\PY{p}{(}\PY{l+m+mi}{100}\PY{p}{)}
                     \PY{n+nb}{print}\PY{p}{(}\PY{l+s+s1}{\PYZsq{}}\PY{l+s+s1}{Accuracy: }\PY{l+s+s1}{\PYZsq{}}\PY{p}{)}
                     \PY{n+nb}{print}\PY{p}{(}\PY{n}{sess}\PY{o}{.}\PY{n}{run}\PY{p}{(}\PY{n}{acc}\PY{p}{,}\PY{n}{feed\PYZus{}dict}\PY{o}{=}\PY{p}{\PYZob{}}\PY{n}{x}\PY{p}{:}\PY{n}{testSet}\PY{p}{[}\PY{l+m+mi}{0}\PY{p}{]} \PY{p}{,}\PY{n}{y\PYZus{}true}\PY{p}{:}\PY{n}{testSet}\PY{p}{[}\PY{l+m+mi}{1}\PY{p}{]}\PY{p}{,}\PY{n}{hold\PYZus{}prob}\PY{p}{:}\PY{l+m+mf}{1.0}\PY{p}{\PYZcb{}}\PY{p}{)}\PY{p}{)}
                     \PY{n+nb}{print}\PY{p}{(}\PY{l+s+s1}{\PYZsq{}}\PY{l+s+se}{\PYZbs{}n}\PY{l+s+s1}{\PYZsq{}}\PY{p}{)}
\end{Verbatim}


    \begin{Verbatim}[commandchars=\\\{\}]
Accuracy: 
0.08


Accuracy: 
0.58


Accuracy: 
0.66


Accuracy: 
0.65


Accuracy: 
0.69


Accuracy: 
0.7


Accuracy: 
0.7


Accuracy: 
0.63


Accuracy: 
0.67


Accuracy: 
0.71



    \end{Verbatim}


    % Add a bibliography block to the postdoc
    
    
    
    \end{document}
