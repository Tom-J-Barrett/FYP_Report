\subsection*{Overview}
In the preious experiments, I have looked at the one-shot approach to food image
classification. That is, the model will give a prediction of the most likely
food item in that image. This is a problem when there are multiple food in an
image, see \ref{figure:fruit}. There are a few options to combat this problem. Firstly, I could detect
objects in the image, segment the image according to these objects and then run
each segment through the model. A simple approach to this would be to segment
the image into a number of sections and then run each section through the model.
In order to follow the latter approach, I used a sliding window approach. This
sliding window would move across the image and classify the segment of the image
in the window. I had three options for window sliding shape as defined by a
command line argument.

\begin{lstlisting}
python sliding_window.py --image=~/image_dir --window_shape grid
\end{lstlisting}

\subsection*{Libraries}
For this experiment Tensorflow provided the classification of each segment while
also helping with resizing along with Numpy. OpenCv was used to implement the
sliding window as per \parencite{slidingWindowTut}.

\subsection*{Results}
\subsubsection*{Grid based window}
The grid based window resulted in fifteen separate classification. As seen in
\ref{figure:fruit}, there are multiple fruits in the image. Of these fruits, our
model is trained on four, apple, banana, orange and grapes. This method
classified all four to Top-1 accuracy at least once each. This method took 42.8
seconds to run.

\begin{table}[]
	\centering
	\caption{My caption}
	\label{my-label}
	\begin{tabular}{ll}
		Food type & No. of Top-1 Classifications \\
		Apple     & 5                      \\
		Banana    & 1                      \\
		Grape     & 4                      \\
		Orange    & 5                     
	\end{tabular}
\end{table}

subsubsection*{Row based window}
The row based method resulted in four predictions as follows. The runtime of
this method was 16.1 seconds.

\begin{table}[]
	\centering
	\caption{My caption}
	\label{my-label}
	\begin{tabular}{ll}
		Food type & No. of Top-1 Classifications \\
		Apple     & 1                      \\
		Banana    & 0                      \\
		Grape     & 0                      \\
		Orange    & 0                      \\
		Other     & 3                     
	\end{tabular}
\end{table}

\subsubsection*{Column based window}
The column based window approach had five total predictions and ran for a total
of 13 seconds.

\begin{table}[]
	\centering
	\caption{My caption}
	\label{my-label}
	\begin{tabular}{ll}
		Food type & No. of Top-1 Classifications \\
		Apple     & 3                      \\
		Banana    & 1                      \\
		Grape     & 0                      \\
		Orange    & 0                      \\
		Other     & 1                     
	\end{tabular}
\end{table}
\subsection*{Analysis}
These results are very interesting because while a banana was only predicted to
Top-1 accuracy once in grid based, once in column based and zero times in row
based, if the whole image is ran through the model, a banana is at the Top-1 accuracy.
























