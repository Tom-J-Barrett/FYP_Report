Many reserachers have used convolutional neural networks for image
classification with various network architectures and many have used a food image dataset. I will be looking at
some of these papers below.

\subsubsection*{Deep Learning Based Food Recognition}
One paper focused on a deep learning approach to food image recognition based
their neural network architecture on Inception-ResNet and Inception V3. They
also used the Food-101 dataset \textcite{deepLearning}. For this system, Googles
Tensorflow was used for image preprocessing. Preprocessing was needed as the
environmental background is different in many food images. Because of these
"Grey World method and Histogram equalization" \textcite{deepLearning} were
used.

AWS GPU instances were used for training and the results on completion were
quite impressive with a Top-1 Accuracy of 72.55\% and a Top-5 Accuracy of
91.31\% \textcite{deepLearning}

\subsubsection*{Food Image Recognition Using Deep Convolutional Neural Network
With Pre-Training and Fine-Tuning}
Another research team in Japan researched this topic. They were aware of how
difficult the problem was and therefore employed many techniques to solve the
problem such as "pre-training with the large-scale ImageNet data, fine-tuning
and activateion features extracted from the pre-trained DCNN"
\textcite{yanaiFood}. 

In conclusion, they found that the "fine-tuned DCNN which was pre-trained
with 2000 categories" \textcite{yanaiFood} from ImageNet was the best method. A
DCNN is a Deep Convolution Neural Network. The achieved results of 78.77\% for
Top-1 Accuracy in the UECFOOD100 dataset.

\subsubsection*{Food Detection and Recognition Using Convolutional Neural Network}
\textcite{kagayaFood} also employed the use of convolutional neural networks for
image detection. They used a CNN for the "tasks of food detection and recognition
through parameter optimization" \textcite{kagayaFood}.

They found that a CNN is much better suited to the task than a Support Vector
Machine (SVM). They achieved an overall classification accuracy of 93.8\%
against their baseline accuracy of 89.7\% \textcite{kagayaFood}. This accuracy
was calculated using a dataset that they created specifically for this task.
When they had completed the task they analysed the trained convolutional kernels
and came to an interesting conclusion. They found that "color features are
essential to food image recognition" \textcite{kagayaFood}.

\subsubsection*{DeepFood: Deep Learning-based Food Image Recognition for
Computer-aided Dietary Assessment}
The last paper that I will look at, oriented around using a convolutional neural
network for food image recognition, focused on developing a dietary assessment
application for use on a smartphone. They used the UEC-256 and Food-101 datset
for their experiments and achieved impressive results.

They used a convolutional neural network but "with a few major optimizations,
such as optimized model and an optimized convolution technique"
\textcite{deepFood}. They used the Inception module for their CNN. After the
inception module was complete, they made the GoogleNet by combining modules. In
total, the network had 22 layers.

They achieved the results shown in \ref{resultsDeepFood}.

\begin{table}[]
	\centering
	\caption{DeepFood Results}
	\label{resultsDeepFood}
	\begin{tabular}{lll}
		                          & Top-1  & Top-5  \\
								  UEC-256                   & 54.7\% & 81.5\% \\
								  UEC-100                   & 76.3\% & 94.6\% \\
								  Food-101                  & 77.4\% & 93.7\% \\
								  UEC-256 With Bounding Box & 63.8\% & 87.2\% \\
								  UEC-100 With Bounding Box & 77.2\% & 94.8\%
	\end{tabular}
\end{table}

