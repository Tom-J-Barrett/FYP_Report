\subsubsection*{A Food Image Recognition System with Multiple Kernel Learning}
In this paper, \textcite{kernelLearning}, a practical use for food image recognition in the form of a mobile phone application was proposed. In order to classify the images, multiple kernel learning(MKL) was used. MKL is similar to am SVM expect that instead of a single kernel during training, MKL "treats with a combined kernel which is a weighted linear combination of several single kernels" \textcite{kernelLearning}. The idea behind this is that different food types are distinguishable by different factors and using this method, the best of these factors can be used for classification of that food type. In the experiment carried out by \textcite{kernelLearning}, three different factors were used for learning:
\begin{itemize}
	\item{Color Histograms}
	\item{Gabor Texture Features}
	\item{Bag-of-Features using Scale Invariant Feature Transformation (SIFT)}
\end{itemize}

50 different classifiers were created in a SVM using MKL with "one category as a positive set and other 49 categories as a negative set" \textcite{kernelLearning}. For each of these categories, a web scrape was carried out and then the best 100 images for each scrape was manually selected. Five-fold cross validation was utilised in the paper. MKL proceeded to yield results of 61.34\% on the 50 food types and a Top-3 accuracy of 80.05\% \textcite{kernelLearning}. The prototype mobile phone application resulted in a 37.55\% user accuracy.

\subsubsection*{A Novel SVM Based Food Recognition Method for Calorie Measurement Applications}
Another, quite successful, study was carried out using a SVM. \textcite{novelSVM} had established that both colour and texture are very important but they also decided that shape and size are vital features to analyse. The proposed system has two main parts, segmentation followed by classification. In order to create a 'robust' system, a 'Robust Handling of Different Lighting Conditions' module is added to the system \textcite{novelSVM}. This is so that various lighting conditions don't cause color data to be distorted. 

Since this paper calls for calorie estimation, the first step of the system calculates the size of the food portion. In order to do this, a coin or the users thumb is included in the image taken so that the pixel count of the thumb and the food can be compared to estimate the size. Following this the image is segmented into various portions. The following step classifies each segment of the image by extracting color, texture and shape features and inputing these into a SVM \textcite{novelSVM}.

12 different food types were trained for this SVM with an average classification accuracy of 92.6\%.

\subsubsection*{Measuring Calorie and Nutrition From Food Image}
Another study that employed both a SVM and an emphasis on colour, texture and shape features, was carried out by \textcite{pouladzadeh2014measuring}. Size was also a factor in the calorie measurement module of the system. It was found that using all four of these features increases the overall accuracy.

In order to segment the image successfully, Gabor filters were applied to separate texture features while color was also utilised. For each segment established, size, shape, color and texture features were extracted and using a SVM, a classification was made. The SVM used the radial basis function kernel \textcite{pouladzadeh2014measuring}.

Calorie estimation was also a large part of this paper, and the users thumb was taken with the food in order to calculate food size.

In the prototype application, once the classification had been made, the user can confirm or change the prediction. Another feature of the application was in regards to "Partially Eaten Food" \textcite{pouladzadeh2014measuring}. This was done by taking a picture before and after consumption and therefore only calculated the size of the food eaten and therefore more accurate calorie counts can be produced.

15 food types were trained using the SVM with 3000 images. The accuracy for the classifier averaged at 90.41\% using 10 fold cross-validation \textcite{pouladzadeh2014measuring}. There was also a calorie count accuracy of 86\%. The best classification results were on single foods followed by non mixed and finally mixed foods produced the worst results.

% \subsubsection*{Segmentation Assisted Food Classification for Dietary Assessment}
% \textcite{zhu2011segmentation}

% \subsubsection*{Large Scale Leaning for Food Image Classification}
% \textcite{LSL_2015}

% \subsubsection*{Promising Approaches of Computer-supported Dietary Assessment and Management: Current Research Status and Available Applications}
% \textcite{arens2015promising}

% \subsubsection*{A Personal Assistive System for Nutrient Intake Monitoring}
% \textcite{personalAssistive}

% \subsubsection*{Toward Dietary Assessment via Mobile Phone Video Camera}
% \textcite{chen2010toward}

% \subsubsection*{Novel Technologies for Assessing Dietary Intake: Evaluating the
% Usability of a Mobile Telephone Food Record Among Adults and Adolescents}
% \textcite{novelTech}

% \subsubsection*{Vitamin D Intake Among Young Canadian Adults: Validation of a Mobile Vitamin D Calculator App}
% \textcite{goodman2015vitamin}

% \subsubsection*{Automatic Chinese Food Identification and Quantity Estimation}
% \textcite{chen2012automatic}

% \subsubsection*{An Overview of the Technology Assisted Dietary Assessment Project at Purdue University}
% \textcite{khanna2010overview}

% \subsubsection*{An Image Processing Approach for Calorie Intake Measurement}
% \textcite{villalobos2012image}

% \subsubsection*{“Snap-n-Eat” Food Recognition and Nutrition Estimation on a Smartphone}
% \textcite{snap}

% \subsubsection*{Merging dietary assessment with the adolescent lifestyle}
% \textcite{schap2014merging}