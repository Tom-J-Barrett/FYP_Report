\section{Support Vector Machine}
A Support Vector Machine (SVM) is a machine learning algorithm that has been
very popular before the use of a CNN was mainstream.
Many of the texts that will be analysed later use SVMs for their classification.

A SVM works by creating an n-dimensional space, with n as the number of
inputs you have \parencite{svm}. The SVM algorithm finds the hyperplane that splits this space.
This hyperplane can then be used for classification. 
An SVM casts the problem to a higher dimensional space and this can solve a problem when the classes are not linearly separable.
This can be seen in Figure \ref{fig:svm}, where on the left there is no clear way to separate the classes but once the problem is cast to another dimensional, the separation is clear.

\begin{figure}[h]
	\centering
    \includegraphics[scale=0.35]{svm}
    \caption[Support Vector Machine Applicability]{Support Vector Machine Applicability sourced from https://stackoverflow.com/questions/9480605/}
    \label{fig:svm}
\end{figure}

A kernel is used to implement a SVM.
There are a few types of kernels that can be used in the SVM algorithm such as:
\begin{itemize}
	\item{Linear Kernel}
	\item{Polynomial Kernel}
	\item{Radial Kernel}
\end{itemize}
The kernel is a vector of how similar two vectors are.
In terms of a linear kernel, the dot product of two vectors is the kernel.

There are benefits and liabilities to using a SVM.
They can be very accurate and can work efficiently with small datasets but unfortunately it can take a large amount of time to train. 