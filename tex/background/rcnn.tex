Due to the issues with composite images (images with multiple items) in food image recognition, an object detection approach may prove to be successful.
If the objects (different food items) in an image could be classified seperately then the overall effiency of a nutrional assessment system, aided by deep learning, could be improved.
Ross Girshik and other contributors had some very positive results in the area
of object detection using region based convolutional neural networks. There were
four iterations of papers based on this work by Ross and groups in UC Berkley,
Microsoft and Facebook. A PHD student at the time of Ross's first paper also
completed his dissertation on the subject. These papers, their
results (Table \ref{rcnnResults}) and the changes made through each iteration will be analysed thoroughly in the proceeding sections.

In the first paper written by Ross Girshik, while researching at UC Berkeley,
focused on two main insights. These were that " one can apply high-capacity convolutional neural networks (CNNs) to bottom-up region proposals in order to localize and segment objects" and that
"when training data is scarce, supervised pre-training for n auxiliary task,
followed by domain-specific fine-tuning, yields a significant performance boost"
\parencite{rcnn}.

The system that they developed followed these steps:
\begin{itemize}
    \item{Take image as input}
    \item{Extract approximately 2,000 region proposals from the image}
    \item{Compute fixed length vectors of features for the regions using a convolutional
        neural network}
    \item{Use a Support Vector Machine (SVM) to classify these regions}
    \item{Bounding box regression for final region proposals}
\end{itemize}

This system utilised selective search to gather these region proposals, but they
mention that a sliding-window detector is also an option. Ross Girshik and his
team used the open source Caffe CNN library for this system. The system is quite
efficient and scalable. It is scalable because of the fixed length vector of
features which will remain constant regardless of inputs and additional outputs.
The team evaluated their results on a few metrics and test sets as seen in Table
\ref{rcnnResults}. Explanation of the datasets used can be seen in Table \ref{datasets}.

Ross Girshik's next iteration of work on region based convolution neural
networks took place in Microsoft Research. This paper was titled "Fast R-CNN" as
its aim was to decrease training and testing time "while also increasing
detection accuracy" \parencite{fastRcnn}.

This paper analyses why RCNN \parencite{rcnn} was slow and therefore how it could be improved.
RCNN was classified to be slow because of three main factors:
\begin{itemize}
	\item{There are multiple stages to training as both a CNN and a SVM need to
		be trained.}
	\item{In training of the SVM, each region proposal must be written to disk
		and is therefore expensive.}
	\item{Object detection takes 47 seconds per image}
\end{itemize}

Due to these problems with RCNN, a new algorithm, titled Fast RCNN was proposed.
The architecture is as follows. An image is taken as input along with a
proposal for regions. The image is pushed through convolutional and pooling
layers (using max pooling). A fixed-length vector of features is then extracted
from each region proposal. These vectors are inputted to fully connected
layers for bounding box location prediction.
At detection time, a pass through of the net is all that is needed so this
runtime is significantly less than RCNN.

Due to the success of RCNN and Fast RCNN, Faster RCNN was introduced to combat
the problem of region proposal computation \parencite{fasterRcnn}.
The architecture for this system comprises of two modules. These consist of a
convolutional neural network for region proposals (RPN) which the feeds into a Fast
RCNN detector. These combine to produce a single neural network for object
detection.

Instead of training these networks separately, the team had to look at how to
share layers between the two networks. There were three options available:
\begin{itemize}
    \item{Alternating training whereby RPN is trained, and then used to train
        Fast RCNN. The Fast RCNN network is then used to initialise RPN and the
		process is iterated \parencite{fasterRcnn}. This paper follows this approach.}
    \item{Approximate joint training.}
    \item{Non- approximate joint training.}
\end{itemize}

\begin{table}[h]
    \centering
    \caption{Results from Region Based CNN Research}
    \label{rcnnResults}
    \begin{tabular}{|p{1.5cm}|l|l|l|l|l|l|}
    \hline
                    & \textbf{VOC07} & \textbf{VOC10} & \textbf{VOC11} & \textbf{VOC12} & \textbf{COCO15} &
                    \textbf{COCO16} \\ \hline
                    RCNN        & 58.5\%  & 53.7\%  & 47.9\%  & N/A     & N/A
                    & N/A      \\ \hline
                    Fast RCNN   & 70.0\%  & 68.8\%  & N/A     & 68.4\%  & N/A
                    & N/A      \\ \hline
                    Faster RCNN & 78.8\%  & N/A     & N/A     & 75.9\%  & 42.7\%
                    & N/A      \\ \hline
                    Mask RCNN   & N/A     & N/A     & N/A     & N/A     & N/A
                    & 63.1\%  \\ \hline
    \end{tabular}
\end{table}

\begin{table}[h]
\centering
\caption{Datasets}
\label{datasets}
\begin{tabular}{|p{1.65cm}|p{10.5cm}|}
\hline
\textbf{Table} & \textbf{Explanation}                                                                                                                                                                                               \\ \hline
VOC07          & The PASCAL VOC dataset is used for the PASCAL (Pattern Analysis, Statistical Modelling and Computational Learning) Visual Object Classes Challenge. \parencite{pascal-voc-2007} was used for the 2007 challenge. \\ \hline
VOC10          & The PASCAL VOC 2010 dataset was used for the 2010 challenge \parencite{pascal-voc-2010}.                                                                                                                         \\ \hline
VOC11          & The PASCAL VOC 2011 dataset was used for the 2011 challenge \parencite{pascal-voc-2011}.                                                                                                                        \\ \hline
VOC12          & The PASCAL VOC 2012 dataset was used for the 2012 challenge \parencite{pascal-voc-2012}.                                                                                                                        \\ \hline
COCO15         & The COCO (Common Objects in Context) dataset created by Microsoft (\parencite{coco}) is used to measure the efficacy of object detection algorithms. The COCO 2015 dataset was released in 2015 for training.    \\ \hline
COCO16         & An updated version of the COCO dataset was released in 2015.                                                                                                                                                       \\ \hline
\end{tabular}
\end{table}

The most recent paper on this topic was also written by Ross Girshik while
working with Facebook AI Research \parencite{maskRcnn}. Mask RCNN " extends Faster
RCNN by adding a branch for predicting an object mask in parallel with the
existing branch for bounding box regression" \parencite{maskRcnn}.
Mask RCNN has two modules, like Faster RCNN, where the first module is the
Region Proposal Network. In the second module, in parallel to classification, a
binary mask is outputted for each region. Bounding box regression and
classification are done in parallel.

