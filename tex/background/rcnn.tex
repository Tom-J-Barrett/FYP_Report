Ross Girshik and other contributers had some very positive results in the area
of object detection using region based convolutional neural networks. There were
four iterations of papers based on this work by Ross and groups in UC Berkley,
Mircosoft and Facebook. A PHD student at the time of Ross's first paper also
completed his dissertation on the subject. I will analyse this papers, their
results (\ref{rcnnResults}) and the changes made through each iteration.

\subsubsection*{RCNN}
In the first paper written by Ross Girshik, while researching at UC Berkeley,
focused on two main insights. These were that "one can apply high-capacity convolutional neural networks (CNNs) to bottom-up region proposals in order to localize and segment objects" and that
"when training data is scarce, supervised pre-training for n auxilary task,
followed by domain-specific fine-tuning, yields a significant performance boost"
\textcite{rcnn}.

The system that they developed followed these steps:
\begin{itemize}
    \item{Take image as input}
    \item{Extract approximately 2000 region proposals from the image}
    \item{Compute fixed length vectors of features for the regions using a convolutional
        neural network}
    \item{Use a Support Vector Machine (SVM) to classify these regions}
    \item{Bounding box regression for final region proposals}
\end{itemize}

This system utilised selective search to gather these region proposals but they
mention that a sliding-window detector is also an option. Ross Girshik and his
team used the open source Caffe CNN library for this system. The system is quite
efficient and scalable. It is scalable because of the fixed length vector of
features which will remain constant regardless of inputs and additional outputs.

The team evaluated their results on a few metrics and test sets as seen in
\ref{rcnnResults}.

\subsubsection*{Fast RCNN}
Ross Girshik's next iteration of work on region based convolution neural
networks took place in Microsoft Research. This paper was titled "Fast R-CNN" as
it's aim was to decrease training and testing time "while also increasing
detection accuracy" \textcite{fastRcnn}.

This paper analyses why RCNN \textcite{rcnn} was slow and therefore how it could be improved.
RCNN was classified to be slow because of three main factors:
\begin{itemize}
	\item{There are multiple stages to training as both a CNN and a SVM need to
		be trained.}
	\item{In training of the SVM, each region proposal must be written to disk
		and is therefore expensive.}
	\item{Object detection takes 47s per image \textcite{fastRcnn}.}
\end{itemize}

Due to these problems with RCNN, a new algorithm, titled Fast RCNN was proposed.
The architecture is as follows. An image is taken as input along with a
proposals for regions. The image is pushed through convolutional and pooling
layers (using max pooling). A fixed-length vector of features is then extracted
from each region proposal. These vectors are inputted to fully connected
layers for bounding box location prediction \textcite{fastRcnn}.

At detection time, a pass through of the net is all that is needed so this
runtime is significantly less than RCNN.

\subsubsection*{Faster RCNN}
Due to the success of RCNN and Fast RCNN, Faster RCNN was introduced to combat
the problem of region proposal computation \textcite{fasterRcnn}.

The architecture for this system comprises of two modules. These consist of a
convolutional neural network for region proposals (RPN) which the feeds into a Fast
RCNN detector. These combine to produce a single neural network for object
detection.

Instead of training these networks seperately, the team had to look at how to
share layers between the two networks. There were three option available:
\begin{itemize}
    \item{Alternating training whereby RPN is trained, and then used to train
        Fast RCNN. The Fast RCNN network is then used to initialise RPN and the
		process is iterated \textcite{fasterRcnn}. This paper follows this approach.}
    \item{Approximate joint training.}
    \item{Non- approximate joint training.}
\end{itemize}

\subsubsection*{Mask RCNN}
The most recent paper on this topic was also written by Ross Girshik while
working with Facebook AI Research \textcite{maskRcnn}. Mask RCNN "extends Faster
RCNN by adding a branch for predicting an object mask in parallel with the
existing branch for bounding box regression" \textcite{maskRcnn}.

Mask RCNN has two modules, similar to Faster RCNN, where the first module is the
Region Proposal Network. In the second module, in parallel to classification, a
binary mask is outputted for each region. Bounding box regression and
classification are done in parallel.

\begin{table}[]
    \centering
    \caption{Results from Region Based CNN Research}
    \label{rcnnResults}
    \begin{tabular}{lllllll}
                    & VOC07 & VOC10 & VOC11 & VOC12 & COCO15 &
                    COCO16 \\
                    RCNN        & 58.5\%  & 53.7\%  & 47.9\%  & N/A     & N/A
                    & N/A      \\
                    Fast RCNN   & 70.0\%  & 68.8\%  & N/A     & 68.4\%  & N/A
                    & N/A      \\
                    Faster RCNN & 78.8\%  & N/A     & N/A     & 75.9\%  & 42.7\%
                    & N/A      \\
                    Mask RCNN   & N/A     & N/A     & N/A     & N/A     & N/A
                    & 63.1\%  
    \end{tabular}
\end{table}
