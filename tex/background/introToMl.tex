\section{Introduction to Machine Learning}
In \parencite{MLANN}, machine learning is defined as " the question of how to
construct computer programs that automatically improve with experience".
Machine learning has blossomed in recent years with applications across multiple
domains using vastly different paradigms and technologies. 
Some of the different approaches used in machine learning are: Artificial Neural Networks, Genetic Algorithms, Decision Tree Learning and Bayesian Learning \parencite{MLANN}.

There are many ways in which machine learning can be used in the modern world,
many of which are being utilised to great effect.
Some of these applications are image recognition, natural language
processing and many more.
These applications can be applied to many different domains such as security (face detection in airports) or object detection (autonomous driving).
\parencite{medical} carried out research into the area of using machine learning to aid diagnosis of medical conditions.
Machine learning algorithms could suggest possible diagnosis for medical professionals to interpret and therefore reduce the time spent diagnosing a patient's condition.
There may be fear that machine learning will start to take away many jobs
from
humans but this may not be the case as in the example above as computers will only be aiding professionals.

One of the most exciting avenues in machine learning is computer
vision.
This is due to the application domains mentioned above. 
Computer vision is the process of extracting high-dimensional data from an image to produce useful information, which in terms of classification usually results in labelling. It can be used in many areas to improve our lives. As
mentioned earlier, autonomous cars are only possible when a machine can
determine what objects are around it. Computer Vision can allow a machine to
recognise medical conditions in an image such as breast cancer using mammography images \parencite{medical}. The applications are nearly limitless.