\section{Introduction to Machine Learning}
In \textcite{MLANN}, machine learning is defined as "the question of how to
construct computer programs that automatically improve with experience".
Machine learning has blossomed in recent years with applications across multiple
domains using vastly different paradigms and technologies. 
Some of the different approaches used in machine learning are: Artificial Neural Networks, Genetic Algorithms, Decision Tree Learning and Bayesian Learning \textcite{MLANN}.

There are many ways in which machine learning can be used in the modern world,
many of which are being utilised to great affect.
Some of these applications are image recognition, natural language
processing,
medical diagnosis and many more.
There may be fear that machine learning will start to take away many jobs
from
humans but this may not be the case. There are many practical uses such as security and safety that could be leveraged such as face detection for security in airports or autonomous cars.

One of the most exciting avenues in machine learning is
computer
vision. Computer vision is the process of extracting high-dimensional data from an image to produce useful information, which in terms of classification usually results in labeling. It can be used in many areas to improve our lives. As
mentioned earlier, autonomous cars are only possible when a machine can
determine what objects are around it. Computer Vision can allow a machine to
recognise skin diseases in an image. The applications are nearly limitless.

