While Convolutional Neural Networks have proven very successful in recent years, there are many other methods of food image identification and classification that have been employed by food image recognition researchers, such as SVMs.
A summary of the results of these methods can be seen in Table \ref{other_dietary_summary}.

\subsubsection*{A Food Image Recognition System with Multiple Kernel Learning}
In this paper, \parencite{kernelLearning}, a practical use for food image recognition in the form of a mobile phone application was proposed.
In order to classify the images, multiple kernel learning(MKL) was used.
MKL is similar to a SVM expect that instead of a single kernel during training, MKL " treats with a combined kernel which is a weighted linear combination of several single kernels" \parencite{kernelLearning}.
The idea behind this is that different food types are distinguishable by different factors and using this method, the best of these factors can be used for classification of that food type.
In the experiments carried out, three different factors were used for learning:
\begin{itemize}
	\item{Colour Histograms}
	\item{Gabor Texture Features}
	\item{Bag-of-Features using Scale Invariant Feature Transformation (SIFT)}
\end{itemize}

50 different classifiers were created in a SVM using MKL with " one category as a positive set and other 49 categories as a negative set" \parencite{kernelLearning}.
For each of these categories, a web scrape was carried out and then the best 100 images for each scrape was manually selected. Five-fold cross validation was utilised in the paper.

MKL proceeded to yield results of 61.34\% on the 50 food types and a Top-3 accuracy of 80.05\%.
The prototype mobile phone application resulted in a 37.55\% user accuracy.

While the Top 1 and Top 5 accuracies for this model show promising results, the user accuracy is very poor.
This would suggest that the classifier does not work well with real life images and is possibly overfitting to the training and testing dataset.
MKL classifiers may not be promising for generalisation.

\subsubsection*{A Novel SVM Based Food Recognition Method for Calorie Measurement Applications}
Another quite successful study was carried out using a SVM.
\parencite{novelSVM} had established that both colour and texture are very important, but they also decided that shape and size are vital features to analyse.
The proposed system has two main parts, segmentation followed by classification.
To create a 'robust' system, a 'Robust Handling of Different Lighting Conditions' module is added to the system \parencite{novelSVM}.
This is so that various lighting conditions don't cause colour data to be distorted. 

Since this paper calls for calorie estimation, the first step of the system calculates the size of the food portion. In order to do this, a coin or the users thumb is included in the image taken so that the pixel count of the thumb and the food can be compared to estimate the size. Following this the image is segmented into various portions. The following step classifies each segment of the image by extracting colour, texture and shape features and inputting these into a SVM.

12 different food types were trained for this SVM with an average classification accuracy of 92.6\%.
\parencite{novelSVM} concluded that it would be difficult to use their algorithm with real data and no evidence of real-world testing was recorded.
This is unfortunate as it is difficult to get an idea of the success of the system.
Also, only 12 food types were trained and it would be interesting to see how the system performed with more classes. 

\subsubsection*{Measuring Calorie and Nutrition From Food Image}
Another study that employed both a SVM and an emphasis on colour, texture and shape features, was carried out by \parencite{pouladzadeh2014measuring}.
Size was also a factor in the calorie measurement module of the system. It was found that using all four of these features increases the overall accuracy.

In order to segment the image successfully, Gabor filters were applied to separate texture features while colour was also utilised. For each segment established, size, shape, colour and texture features were extracted and using a SVM, a classification was made. The SVM used the radial basis function kernel.

Calorie estimation was also a large part of this paper, and the users thumb was taken with the food to calculate food size.

In the prototype application, once the classification had been made, the user can confirm or change the prediction. Another feature of the application was in regards to " Partially Eaten Food" \parencite{pouladzadeh2014measuring}. This was accomplished by taking a picture before and after consumption and as a result only calculated the size of the food eaten and therefore more accurate calorie counts can be produced.

15 food types were trained using the SVM with 3000 images. The accuracy for the classifier averaged at 90.41\% using 10-fold cross-validation.
There was also a calorie count accuracy of 86\%. The best classification results were on single foods followed by non-mixed and finally mixed foods produced the worst results.

Even though \parencite{pouladzadeh2014measuring} segmented the image before classification, they still had poor results on mixed foods.
The low number of classes the classifier was trained on makes it difficult to see how effective the classifier is.
It would also be beneficial to know how long classification took on a new image as the classifier used features of colour, texture, size and shape which would be quite time consuming.

\subsubsection*{Segmentation Assisted Food Classification for Dietary Assessment}
\parencite{zhu2011segmentation} had a strong focus on the segmentation aspect of a dietary assessment system.
The segmentation of the food images was achieved " using Normalized Cuts based on intensity and colour" \parencite{zhu2011segmentation}.
Normalized Cuts is a graph-based segmentation method.
To aid the segmentation aspect of this study, a common background colour was introduced to the images.
Segmentation refinement was also an important module in the experiment.
This is the process by which neighbouring segments with  the same classification label are merged together.
This also helps calculate a more accurate size estimation.

The classification of the segmented image was processed by using a SVM calculating colour and texture features.
Gabor filters were used for the texture feature extraction.

In the experimental results for this study, it was found that segmentation was not always successful " when the region of interest is camouflaged by making its boundary faint" \parencite{zhu2011segmentation}. In their case, it was a can of coke that wasn't segmented correctly.
The classification accuracy was of 56.2\% and 95.5\% with ground truth segmentation data.
19 classes of food were used in this study with approximately 60 images per class.

Very little information was given in this paper on the classifier used.
The low classification results do not yield promising results for this method.

% % \subsubsection*{Promising Approaches of Computer-supported Dietary Assessment and Management: 
% % Current Research Status and Available Applications}
% % \parencite{arens2015promising}

% \subsubsection*{Novel Technologies for Assessing Dietary Intake: Evaluating the
% Usability of a Mobile Telephone Food Record Among Adults and Adolescents}
% \parencite{novelTech}

\subsubsection*{Automatic Chinese Food Identification and Quantity Estimation}
There was a study carried out on food identification through a smart phone application \parencite{chen2012automatic}.
This study resulted in an application that allows a user to send an image of their food to a server which can give them an automatic response in 12 seconds \parencite{chen2012automatic}.
This back-end service can have 34 threads working concurrently as stated at the time the paper was published \parencite{chen2012automatic}.

A SVM is used to classify the image across 50 categories trained on around 100 images each.
The SVM uses SIFT and Local Binary pattern feature extractors \parencite{chen2012automatic}.
A separate SVM was trained for each of these extractors and was merged together using a " Multi-class AdaBoost algorithm" \parencite{chen2012automatic}.

The study produced a top-1 accuracy  of 68.3\%. Accuracy of 80.6\%, 84.8\% and 90.9\% were recorded using top-2, top-3 and top-5 accuracy respectively \parencite{chen2012automatic}.
Unfortunately, real-world image classifcation results were not recorded from the smartphone applications.
Due to this, it is difficult to measure the efficacy of the classifier, even more so because they only had 50 classes.  

% % \subsubsection*{An Overview of the Technology Assisted Dietary Assessment Project at Purdue University}
% % \parencite{khanna2010overview}

\subsubsection*{An Image Processing Approach for Calorie Intake Measurement}
\parencite{villalobos2012image} researched the question of using a computer vision approach to this topic. They focus mostly on the segmentation and region of interest calculation of the system in their study.

The system in question requires two images of the food, one from above and one from the side. This helps with size estimation. The users thumb is required to be in the image for accurate size estimation. The application also requires an image after consumption as to not calculate calories for uneaten food.

The system segments the image and then extracts colour, size and shape information from each segment. This data is then used by a SVM for classification along with a nutritional database for calorie information. 

Multiple segmentation methods were tested such as:
\begin{itemize}
	\item{Semi-automatic contour definition}
	\item{Watershed transformation}
	\item{Colour rasterization}
	\item{Edge accentication}
\end{itemize}
The first two were dismissed due to poor results but the second two were used in conjunction for the segmentation aspect of the system.

\parencite{villalobos2012image} did not provide extension information on the classifier used in this study and would be therfore very difficult to replicate.
However, impressive segmentation results were obtained.

\subsubsection*{Food Recognition and Nutrition Estimation on a Smartphone}
An application called "Snap-n-Eat" was proposed by \parencite{snap}.

When an image is taken using this application, the system finds saliency regions to remove the background of the image.
If the image has multiple food types present, hierarchical segmentation takes place before proceeding to a SVM. Similar segments are merged together.
These are found by using colour, texture and size.

SIFT and Histogram of Oriented Gradients (HOG) feature extractors are used on the image and these features are used by the SVM for classification.
The SVM is trained using Scholastic Gradient Descent.
\parencite{snap} also uses a Bog of Visual Words model along with k-means clustering.

An accuracy of 85\% on 15 classes was recorded using this method \parencite{snap}.

\begin{table}[]
\centering
\caption{Summary of accuracy in dietary assessment methods}
\label{other_dietary_summary}
\begin{tabular}{|l|l|l|}
\hline
\textbf{Title}                                                         & \textbf{Classes} & \textbf{Accuracy}   \\ \hline
Food Image Recognition with Multiple Kernel Learning          & 50             & 61.3\%    \\ \hline
A Novel SVM Based Food Recognition Method                     & 12             & 92.6\%     \\ \hline
Measuring Calorie and Nutrition from Food Image               & 15             & 90.4\%    \\ \hline
Segmentation Assisted Food Classification                     & 19            & 56.2\%     \\ \hline
Large Scale Leaning for Food Image Classification             & 11             & 78.0\%       \\ \hline
Toward Dietary Assessment via Mobile Phone Video Camera       & 20             & 92.0\% \\ \hline
Automatic Chinese Food Identification and Quantity Estimation & 50             & 68.3\%     \\ \hline
Food Recognition and Nutrition Estimation on a Smartphone      & 15             & 85.0\% \\ \hline      
\end{tabular}
\end{table}