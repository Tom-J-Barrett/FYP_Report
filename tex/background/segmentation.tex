\subsubsection*{Graph Based Segmentation}
Graph cut segmentation has been used extensively  in image segmentation. OpenCV
has an implementation of a graph cut algorithm called grabcut which has been
used to segment food on occasion \parencite{graphCut}. 

According to \parencite{graphCut}, "Graph cut based method is well-known to be
efficient, robust, and capable of finding the best contour of objects in n
image, suggesting it to be a good method for separating food portions in a food
image for calorie measurement". Along with the graph cut segmentation algorithm,
this research team also used colour and texture segmentation. Gabor filters were
used to measure texture features \parencite{graphCut}. When colour and texture
segmentation was applied, the method came into difficulty with mixed foods but
by applying graph cut segmentation, clearer object boundaries were shown.

In conclusion, the accuracy of the classification increased when using graph
based segmentation rather than colour and texture as seen in \ref{graphCT}.

\begin{table}[]
	\centering
	\caption{Results}
	\label{graphCT}
	\begin{tabular}{llll}
		                  & Single Food Portion & Non-mixed Food & Mixed Food
						  \\
						  colour and Texture & 92.21               & N/A
						  & N/A           \\
						  Graph Based       & 95 (3\% increase)   & 5\% increase
						  & 15\% increase
	\end{tabular}
\end{table}

\subsubsection*{Local Variation Framework}
Another paper was published in which the research team attempted to create a
food calorie extimation system \parencite{foodImageAnalysis}. This system would
compromise of three steps, image segmentation, image classification and weight
estimation. For the segmentation module, a local variation approach to
segmentation was performed. Local variation is by which intensity differences
between neighbouring pixels is measured. This is a type of graph based
segmentation.

The team also carried out some segmentation refinement when the segmentation
algorithm had been performed. This consisted of removed small segments (defined
as less than 50 pixels) and trying to prevent over and under segmentation. After
classification was performed on each segment, segments with low confidence
values were removed \parencite{foodImageAnalysis}.

\subsubsection*{Conclusion}
Both of the above papers of \parencite{graphCut} and \parencite{foodImageAnalysis}
used a graph based segmentation. The first paper used a more generic
implementation while the second used a local variation framework. Both methods
provided successful results in the image segmentation process.
