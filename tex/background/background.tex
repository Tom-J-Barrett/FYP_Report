\chapter{Background}
\label{background}

\section{Introduction to Machine Learning}
In \textcite{MLANN}, Machine Learning is defined as "the question of how to
construct computer programs that automatically improve with experience".
Machine Learning has blossomed in recent years with applications across multiple
domains using vastly different paradigms and technologies.

There are many ways in which Machine Learning can be used in the modern world,
many of which are being utilised to great affect.
Some of these applications, are image recognition, natural language processing,
medical diagnosis and many more.
There may be fear that Machine Learning will start to take away many jobs from
humans but this may not be the case. Imagine a doctor, having to diagnose a
patient, a machine can offer suggestions based on very large datasets of what
the diagnosis is. This is not to say that a Machine would be
perscribing patients, but merely act as an assistant to the doctor.

Machine Ethics is a large problem that comes hand in hand with Machine Learning.
There is a very important question of who takes the blame when things go wrong,
that is why I think it is important that we only use Machine Learning to advise
and not to determine but this can be very difficult in a world where, for
example, an autonomous car has to decide between crashing into a vehicle beside
them with an unknown number of people inside or the two children playing in the
street.

One of the most exciting avenues in Machine Learning, in my opinion, is Computer
Vision. Computer Vision can be used in many areas to improve our lives. As
mentioned earlier, autonomous cars are only possible when a machine can
determine what objects are around it. Computer Vision can allow a machine to
recognise skin diseases in an image. The applications are nearly limitless and
that is without taking into account other uses.

\section{Neural Computing}
The main area of my focus for this project is in
Artificial Neural Networks (ANN). This is because I have researched extensively into Convolutional Neural Networks which are based on ANN's.
\subsection{Artificial Neural Networks}
An Artificial Neural Network is a bio-inspired system that is used to model the human brain in how it learns from experience.
The ANN uses this model to build a very complex web of connected units called
artificial neurons.
These nuerons are connected by certain weights which determines the processing
capacity of the network and these weights are created by learning a
dataset.(Malachy)
An ANN has a set of inputs that take in a value, sometimes from network outputs
and produce a single result or classification.
While an ANN is bio-inspired from the human brain, there are many elements of
the brain that are not present in ANN and many new elements in ANN that are not
modelled from the human brain.

Before I can talk about Convolution Neural Networks which are vital the image
processing, I will have to talk about the perceptron learning algorithm, the multi
layer percepton, and backpropogation.

\section{Perceptron Learning - Artificial Neuron}
In our Artificial Neural Network a Perecptron is an Artificial Neuron.
It is called an Artifical Neuron because it is a bio-inspired neuron which models
a neuron in the human brain in terms of inputs and output.

In Perceptron learning, we can take two inputs which are put towards an
activation function with a bias attached as seen in \ref{fig:perceptron}.
These inputs are multiplied by the weights that connect the input to the
activation function and depending on the result, the activation function may
fire an output. These inputs are either 1 or -1.

\begin{figure}
     \includegraphics{Perceptron}
     \caption{Perceptron}
     \label{fig:perceptron}
\end{figure}

This Perceptron Learning Rule assumes that there are two sets of instances, a
positive and negative set, and that they are linearly seperable.

A perceptron is trained using supervised learning. When the perceptron
classifies a results, it is told if it is correct or not. If the result is
incorrect, weights are changed in value so that this error can be reduced
\textcite{AI}. 

The one major problem with perceptron learning and that is that it can't solve
the problem if there is not a clear linear separation between the classes. There
is a way in which we can attempt to solve this, through the delta rule. The
delta rule utilizes gradient descent to find the best weight for the training
samples \textcite{MLANN}. We will discuss gradient descent in the next section.

\section{Multi Layered Perceptron}
Multi Layer Perceptrons (MLP) are made up of multiple layers of perceprons connected
together.
Firstly, we have an input layer, followed by one of more hidden layers and then
finally an output layer.
Any Neural Network with more than three hidden layers is categorised as a deep
layer.

The input layer of your network consists of the data you feed into the network
in order to classify it. The input layer passes this data to a hidden layer
whose purpose is to transform this data into something that the output layer can
understand. The output layer normally consists of a class prediction.

Multi Layer Perceptrons are a class of feed forward Artificial Neural Networks.
These means that the output of each perceptron feeds into an input in the next
layer of the network.

There is one large problem with MLP's and this is why Convolutional Neural
Networks (CNN) were created. If you attempting to classify images with an MLP then
each pixel in that image would have to be a separate input. This created a
massive amount of neurons through all the layers and this isn't feasible. CNN's
solve this problem which we will discuss later.

\begin{figure}
    \includegraphics[width=150mm,scale=0.5]{mlp}
     \caption{Multi Layer Perceptron}
     \label{fig:mlp}
\end{figure}

\subsubsection*{Gradient Descent and backpropagation}
Gradient Descent is an algorithm used to find the optimal weights to produce the
smallest prediction error. It is used to overcome problems of non linearly
separable classes. Gradient descent search selects a random weight value and
then modifies it gradually to minimize the error. "At each step, the weight
vector is altered in the direction that produces the steepest descent along the
surface" \parencite{MLANN}. This step is iterated until the lowest value is met.

There is an error function used for the perceptron which finds the lowest error for that neuron, but it can't be used here because, since we have many neurons, there could be an error in multiple neurons.
Gradient Descent is mathematically based on the derivative of a function.
The gradient of a function can be calculated by differentiating it.
As the weights are what is being controlled, " they are what we differentiate in respect to" \parencite{MLAlgorithm}.
The negative gradient of this function is followed to find the lowest possible point, hence the name gradient descent \parencite{MLAlgorithm}.

One problem with Gradient Descent is that if we look at Figure \ref{fig:GD}, we may
never get to the optimal point, point B. This is because we will find point A
without too many problems but when the weights change we will get too high a
slope of error and therefore will never reach point B.

Another variation of Gradient Descent is Stochastic Gradient Descent (SGD). SGD
is different because it updates " weights incrementally, following the
calculation of the error of each individual example" \parencite{MLANN}. 

\begin{figure}[h]
      \includegraphics{GradientDescent}
      \caption{Gradient Descent}
      \label{fig:GD}
 \end{figure}

"The Backpropagation algorithm learns the weights of a multilayer network,
given a network with a fixed set of units and interconnections" \parencite{MLANN}.
Backpropagation attempts to minimise the mean squared error between the target
output and the output of a network.

Backpropagation works by starting at the output layer of the network and going
back through previous hidden layers, updating weights as it goes i.e. it propagates back through the network, updating the weights to try and reduce the error.

\parencite{MLANN} defined a walk through of the backpropagation algorithm.
For every value of \[\vec{x}, \vec{t}\], in the training set where x is a vector of inputs and t is a vector of output values to act a target:
\begin{itemize}
	\item{Run x through the network and output \[o_{u}\]}
	\item{For each output k, calculate the error by: \[\delta_{k} \leftarrow o_{k}(1 - o_{k})(t_{k} - o_{k}) \]}
	\item{For every hidden unit, calculate the error by: \[\delta_{h} \leftarrow o_{h}(1 - o_{h}) \sum_{k \in outputs}   w_{kh}\delta_{k}\]}
	\item{Updates weights by: \[w_{ji} \leftarrow w_{ji} + \Delta w_{ji}\] where \[\Delta w_{ji} = \n \delta_{j} x_{ji}\]}
\end{itemize}


\section{Support Vector Machine}

\section{Convolutional Neural Networks}
Convolutional Neural Networks (CNN's) are essentially a Multi Layered Percetron with a
special structure. CNN's have one major difference from a MLP, they have extra
layer of convolution and pooling.

Figure \ref{fig:XtoRec} show an image that we want to compare against
Figure \ref{fig:XtoComp}.
For humans, it is quite easy to determine that these images are very similar but
for a computer this task is surprisingly difficult.

So what a CNN does, to combat this problem, is to take a small feature from
Figure \ref{fig:XtoRec} and compare it to a subsection of Figure \ref{fig:XtoComp}.
The CNN multiplies the feature and a section of Figure \ref{fig:XtoComp}, adds
up the results and divides by 9. This then gives a decimal value of how likely
it is that the feature is in the part of the image, as seen in Figure
\ref{fig:convoluted}.
This is called filtering. The Convolutional layer is composed of carrying out
this filtering for every single possible location in Figure \ref{fig:XtoComp}.
\begin{figure}
	\caption{Image filtering}
    \label{fig:filter}
      \begin{subfigure}[b]{0.4\textwidth}
          \includegraphics[width=\textwidth]{XtoRec}
          \caption{Image to Classify}
          \label{fig:XtoRec}
      \end{subfigure}
    %
      \begin{subfigure}[b]{0.4\textwidth}
      \includegraphics[width=\textwidth]{XtoComp}
      \caption{Image to Compare}
      \label{fig:XtoComp}
      \end{subfigure}
\end{figure}

\begin{figure}
    \caption{Image Convolution}  
    \begin{subfigure}[b]{0.4\textwidth}
          \includegraphics[width=\textwidth]{ImageFeature}
          \caption{Image Feature to Search}
          \label{fig:feature}
      \end{subfigure}
     %
      \begin{subfigure}[b]{0.4\textwidth}
           \includegraphics[width=\textwidth]{ConvImage}x
           \caption{Convoluted Image}
           \label{fig:convoluted}
      \end{subfigure}
\end{figure}
\begin{figure}
    \includegraphics[width=50mm,scale=0.5]{PooledImage}
    \caption{Pooled Image}
    \label{fig:pooled}
\end{figure}
Next is the Pooling Layer, what this layer does, is it takes the convoluted
layer output, you can use Figure \ref{fig:convoluted} as reference, and from a
user defined size ie. 2x2, gets either the highest decimal value (Max pooling)
or the average value (Mean pooling) and records that as the new value for the
section. This is then applied to the entire image. As we can see in Figure
\ref{fig:pooled} we know have a much smaller image stack in which to classify,
thus making the computation easier.

In between the Convolution and Pooling layer, there is sometimes a Normalization
layer. This Normalization layer creates Rectified Linear Units (RLU's). In other
words, if we take Figure \ref{fig:convoluted}, it changes all minus values to
zero.

\subsubsection{Fully Convolutional Networks}
A Fully Convolutional Network is one that does not have a fully connected layer
and in a fully connected layers place is another convolution layer.

\section{Overview of Machine Vision Approaches to Identification and Classification}
There have been many attempts of identification and classification by many
different researchers over the last number of years. There some approaches that
decouple the two tasks of identification and classification from one another but
mostly, researchers have attempted the two together. Sometimes, by semantic
segmentation and in others simply building a classifier for the image without
taking into account, the need for object identification.

\subsection{Region Based Convolutional Neural Networks}
Ross Girshik and other contributers had some very positive results in the area
of object detection using region based convolutional neural networks. There were
four iterations of papers based on this work by Ross and groups in UC Berkley,
Mircosoft and Facebook. A PHD student at the time of Ross's first paper also
completed his dissertation on the subject. I will analyse this papers, their
results and the changes made through each iteration.

\subsubsection{RCNN}
In the first paper written by Ross Girshik focused on two main insights. These
were that "one can apply high-capacity concolutional neural networks (CNNs) to
bottom-up region proposals in order to localize and segment objects" and that
"when training data is scarce, supervised pre-training for n auxilary task,
followed by domain-specific fine-tuning, yields a significant performance boost"
\textcite{rcnn}.
\subsubsection{Fast RCNN}
\textcite{fastRcnn}
\subsubsection{Faster RCNN}
\textcite{fatserRcnn}
\subsubsection{Mask RCNN}
\textcite{maskRcnn}
\subsubsection{Transferable Representations for Visual Recognition}

\subsection{Fully Convolutional Neural Networks for Semantic Segmentation}
There has been a very interesting paper from UC Berkeley focused on using Full Convolutional
Networks for semantic segmentation \textcite{fcn}. Fully Convolutional Networks
do not have any fully connected layers. They are replaced with more filtering
layers. Nvidia Digits have a semantic segmentation implementation based off the
work of this paper.

They took this approach because "feedforward computation and backpropogation are
much more efficient when computer layer-by-layer over an entire image instead of
independently patch-by-patch" \textcite{fcn}.

\subsection{Image Segmentation}
\subsubsection*{Graph Based Segmentation}
Graph cut segmentation has been used extensively  in image segmentation. OpenCV
has an implementation of a graph cut algorithm called grabcut which has been
used to segment food on occasion \parencite{graphCut}. 

According to \parencite{graphCut}, "Graph cut based method is well-known to be
efficient, robust, and capable of finding the best contour of objects in n
image, suggesting it to be a good method for separating food portions in a food
image for calorie measurement". Along with the graph cut segmentation algorithm,
this research team also used colour and texture segmentation. Gabor filters were
used to measure texture features \parencite{graphCut}. When colour and texture
segmentation was applied, the method came into difficulty with mixed foods but
by applying graph cut segmentation, clearer object boundaries were shown.

In conclusion, the accuracy of the classification increased when using graph
based segmentation rather than colour and texture as seen in \ref{graphCT}.

\begin{table}[]
	\centering
	\caption{Results}
	\label{graphCT}
	\begin{tabular}{llll}
		                  & Single Food Portion & Non-mixed Food & Mixed Food
						  \\
						  colour and Texture & 92.21               & N/A
						  & N/A           \\
						  Graph Based       & 95 (3\% increase)   & 5\% increase
						  & 15\% increase
	\end{tabular}
\end{table}

\subsubsection*{Local Variation Framework}
Another paper was published in which the research team attempted to create a
food calorie extimation system \parencite{foodImageAnalysis}. This system would
compromise of three steps, image segmentation, image classification and weight
estimation. For the segmentation module, a local variation approach to
segmentation was performed. Local variation is by which intensity differences
between neighbouring pixels is measured. This is a type of graph based
segmentation.

The team also carried out some segmentation refinement when the segmentation
algorithm had been performed. This consisted of removed small segments (defined
as less than 50 pixels) and trying to prevent over and under segmentation. After
classification was performed on each segment, segments with low confidence
values were removed \parencite{foodImageAnalysis}.

\subsubsection*{Conclusion}
Both of the above papers of \parencite{graphCut} and \parencite{foodImageAnalysis}
used a graph based segmentation. The first paper used a more generic
implementation while the second used a local variation framework. Both methods
provided successful results in the image segmentation process.

\subsection{Convolutional Neural Networks for Classification}
Many researchers have used convolutional neural networks for image
classification with various network architectures and many have used a food image dataset.
Some of these papers will be looked at below.
A summary of their results can be seen in Table \ref{cnn_summary}.

\parencite{deepLearning} focused on a deep learning approach to food image recognition based
their neural network architecture on Inception-ResNet and Inception V3.
Deep learning is a term usually given to algorithms based on neural networks.
They also used the Food-101 dataset. For this system, Google's
Tensorflow was used for image preprocessing. Preprocessing was needed as the
environmental background is different in many food images. Because of these
"Grey World method and Histogram equalization" \parencite{deepLearning} were
used.
Amazon Web Services (AWS) Graphics Processing Units (GPU) instances were used for training.
AWS instances are cloud servers.
The results on completion were quite impressive with a Top-1 Accuracy of 72.55\% and a Top-5 Accuracy of 91.31\%.

Another research team in Japan, \parencite{yanaiFood} researched this topic. This was built off previous research they had carried out in the field \parencite{kawano2014food}.
They were aware of how
difficult the problem was and therefore employed many techniques to solve the
problem such as "pre-training with the large-scale ImageNet data, fine-tuning
and activation features extracted from the pre-trained DCNN". 
In conclusion, they found that the "fine-tuned DCNN which was pre-trained
with 2000 categories" from ImageNet was the best method. A
DCNN is a Deep Convolution Neural Network. 
A network can become a DCNN when the number of hidden layers is larger than three.
While many of the CNNs discussed have been DCNN, most are just labeled as CNNs.
The achieved results of 78.77\% for Top-1 Accuracy in the UECFOOD100 dataset.


\parencite{kagayaFood} also employed the use of convolutional neural networks for
image detection. They used a CNN for the "tasks of food detection and recognition
through parameter optimization".
They found that a CNN is much better suited to the task than a Support Vector
Machine (SVM). They achieved an overall classification accuracy of 93.8\%
against their baseline accuracy of 89.7\%. This accuracy
was calculated using a dataset that they created specifically for this task.
When they had completed the task they analysed the trained convolutional kernels
and came to an interesting conclusion. They found that "color features are
essential to food image recognition".


The last paper that will be analysed, \parencite{deepFood}, oriented around using a Convolutional Neural
Network for food image recognition, focuses on developing a dietary assessment
application for use on a smart phone. They used the UEC-256 and Food-101 dataset
for their experiments and achieved impressive results.
They used a Convolutional Neural Network but "with a few major optimizations,
such as optimized model and an optimized convolution technique". 
They used the Inception module for their CNN. After the
inception module was complete, they made the GoogleNet architecture by combining modules. In
total, the network had 22 layers.
They achieved the results shown in Table \ref{resultsDeepFood}.

\begin{table}[h]
	\centering
	\caption{DeepFood Results}
	\label{resultsDeepFood}
	\begin{tabular}{|l|l|l|}
	\hline
		\textbf{Dataset} & \textbf{Top-1}  & \textbf{Top-5}  \\  \hline
		UEC-256                   & 54.7\% & 81.5\% \\ \hline
		UEC-100                   & 76.3\% & 94.6\% \\ \hline
		Food-101                  & 77.4\% & 93.7\% \\ \hline
		UEC-256 With Bounding Box & 63.8\% & 87.2\% \\ \hline
		  UEC-100 With Bounding Box & 77.2\% & 94.8\% \\ \hline
	\end{tabular}
\end{table}

\parencite{nutrinet} developed a new neural architecture specifically for detecting food and drink images using deep convolutional neural networks called NutriNet.
The trained network was to be used to aid patients with Parkinson's disease in monitoring their diet.
NutriNet was created based off of the AlexNet architecture.
The dataset used for this study consisted of approximately 500 images for each of over 500 classes.
Through this dataset a Top-1 accuracy of 86.72\& and a Top-5 accuracy of 94.47\% was recorded.
A smart phone application was used for real world testing which brought a Top-5 accuracy of 55\%.
The application also saved these real world images from the smart phone to increase their dataset size.
In conclusion, the team found that there were modifications that could be made to the NutriNet architecture as real world images didn't perform incredibly well due to occlusion and background noise in the images.
The detection and recognition steps were separated in this architecture.
The team also acknowledges that joining these steps into a single DCNN may be successful and should be explored.

\begin{table}[h]
	\centering
	\caption{Summary of results in CNN based methods}
	\label{cnn_summary}
	\begin{tabular}{|l|l|l|}
	\hline
		\textbf{Title}                                & \textbf{Dataset}     & \textbf{Top-1 Accuracy} \\ \hline
		\parencite{deepLearning} 			 & Food 101    & 72.6\%  \\ \hline
		\parencite{yanaiFood}               	 & UECFood101  & 78.8\%  \\ \hline
		\parencite{kagayaFood}       		 & Own dataset & 93.8\%   \\ \hline
		\parencite{deepFood}                  & Food 101    & 77.4\%  	\\ \hline
		\parencite{nutrinet}                  & Own dataset & 86.7\% \\ \hline
	\end{tabular}
\end{table}



\section{Technologies}
\subsection{Tensorflow}
TensorFlow is a deep learning software library for various machine learning paradigms. TensorFlow will be used to create neural networks.
TensorFlow uses a data structure called a tensor which is basically an array of n dimensions.
TensorFlow has two utilisations, through a Graphics Processing Unit (GPU) and through a Central Processing Unit (CPU).
GPU computation is recommended for CNN training. 

\tocless\subsubsection{Central Processing Unit Computation}
It is quite easy to get TensorFlow up and running if you are only using a CPU to
train. TensorFlow CPU has been successfully installed on both Windows and Ubuntu, for this project.
For Windows you can download and install using the TensorFlow website and on ubuntu you can
using apt-get. Once installed, TensorFlow can be imported into any python shell
or script for use. TensorFlow can also be used in C++. There will be various
python implementations of neural networks in Chapter 3.

\tocless\subsubsection{Graphics Processing Unit Computation}
For use with a GPU, the set up for TensorFlow is a bit more complicated. Firstly
you must check that the GPU in your machine is compatible for CUDA 8.0 using the
NVIDIA website. If your GPU is compatible, you must install CUDA after signing
up as an NVIDIA developer. CUDA 8.0 is compatible with TensorFlow.
You also need to install cudnn6.
The NVIDIA website contains tutorials to install these.
Once these are installed, download and install tensorflow-gpu.
This can be imported into python similar to CPU computation.

\subsection{Jupyter}
\input{tex/tech/jupyter}
\subsection{OpenCV}
OpenCV is an industry wide, open source library for computer vision and machine learning \parencite{opencv}.
It has over 2500 algorithms that are available for use \parencite{opencv}.
OpenCV is supported across multiple languages and platforms such as Python, C++, C, Java, Matlab, running on Windows, Android, Mac OS and Linux \parencite{opencv}.

There is not much of the library utilised in this project due to the nature of Tensorflow but some algorithms for image reading, writing and resizing were used due to the ease of use.

\begin{lstlisting}
image = cv2.imread('image.jpg')
\end{lstlisting}

\begin{lstlisting}
resized = cv2.resize(image, (299, 299))
\end{lstlisting}

\begin{lstlisting}
cv2.imwrite('imageResized.jpg', resized)
\end{lstlisting}


\section{Evaluating the Output}
There are various metrics that can be used for evaluating the output
of a classifier or segmentation algorithm, many of which we have seen in previous
sections. These metrics will be examined below.

There has been some research into the question of how to evaluate object
detectors, one of which be discussed in detail \parencite{diagnosingErrors}.
This paper in question "analyzes the influences of object characteristics on
detection performance and the frequency and impact of different types of false
positives" \parencite{diagnosingErrors}. They found that there were many effects
that had influence on detectors as follows:
\begin{itemize}
    \item{occlusion}
    \item{size}
    \item{aspect ratio}
    \item{visibility of parts}
    \item{viewpoint}
    \item{localization error}
    \item{confusion with semantically similar objects}
    \item{confusion with other labelled objects}
    \item{confusion with background}
\end{itemize}

The research team goes on to analyse false positives in object detectors.
Localization errors were a large factor. This is where bounding boxes overlap to
other objects in the image. Confusion with similar objects had a large influence
on false positives also by which, for example, a dog detector had a high score
for a cat \parencite{diagnosingErrors}. Confusion with dissimilar objects and
confusion with background are the categories of the rest of the false positives
they measured.

In conclusion the team found that "Most false positives are due to misaligned
detection windows or confusion with similar objects"
\parencite{diagnosingErrors}. They had some recommendations towards improves
detectors as follows:
\begin{itemize}
	\item{Smaller objects are less likely to be detected}
	\item{Localization could be improved}
	\item{Reduce confusion with similar categories}
	\item{Robustness to object variation}
	\item{More detailed analysis}
\end{itemize}

% \tocless\subsection{Detection Average Precision}
% The average detection accuracy of a system. See \ref{fig:dap}.

% \tocless\subsection{Mean Average Precision}
% The mean accuracy of a system across all results. See \ref{fig:MAP}.
		
% \tocless\subsection{Distribution of top-ranked false positives}
% This metric is used to tell where most errors occur when false positives are
% evident. These errors are broken down into four categories of localization,
% similar objects, background confusion and others. See \ref{fig:DFP}.

% \tocless\subsection{Segmentation Mean Accuracy}
% This is the mean accuracy of segmentation. See \ref{fig:SMP}.

% \tocless\subsection{Per-category segmentation accuracy}
% This metric measures the accuracy of segmentation at a category level. See \ref{fig:PCATSA}.	

% \tocless\subsection{Per-class segmentation accuracy}
% The accuracy of segmentation at a class level is analysed. See \ref{fig:PCLASSA}.

\tocless\subsection{Top-1 and Top-5 Accuracy}
When a classifier is given an image it normally responds with a list of
predictions along with a decimal representation of the likelihood that it is of
that class. The Top-1 accuracy is the top valued prediction and Top-5 accuracy
is the top 5 predictions.

\tocless\subsection{Cross-Validation}
When cross-validation is utilised, the test set is used for gradient descent while a separate validation set is used to measure error \parencite{MLANN}.
K-fold cross-validation is where k separate validation sets are created.
These k different validation sets are used to test the model and the results are averaged.

\tocless\subsection{Confusion Matrix}
A proven, very successful way to measure how well a classifier is performing is by using a confusion matrix \parencite{handsOnML}.
A confusion matrix simply counts the number of times an object is classifier correctly or incorrectly as in Table \ref{cm}.

\begin{table}[]
\centering
\caption{Confusion Matrix}
\label{cm}
\begin{tabular}{|l|l|l|}
\hline
\textbf{Actual (n=200)} & \textbf{Predicted: NO} & \textbf{Predicted: YES} \\ \hline
NO                      & 55                     & 18                      \\ \hline
YES                     & 7                      & 120                     \\ \hline
\end{tabular}
\end{table}



