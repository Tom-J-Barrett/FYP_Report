\chapter{Introduction to Using Tensorflow}
The purpose of this chapter is to give an overview of what has been completed in order to gain understanding of how to create CNNs and also how to work with datasets. Three experiments are explored in this chapter, the first two of which are tutorials completed as an introduction to Tensorflow and CNNs. The final experiment explores how to feed datasets from file directories into CNNs.

\section{Template for Experiments}
The template followed for all experiments in chapters three, four and five is outlined in Figure \ref{fig:expTemplate}.

\begin{table}[]
\centering
\caption{Experiment Template}
\label{fig:expTemplate}
\begin{tabular}{|p{4cm}|p{11cm}|}
Experiment Section   & Rationale                \\
Overview             & An explanation of the purpose of the experiment along with how it is carried out. \\
Network Architecture & An explanation of the network architecture used in the experiment. If the architecture has been explained in another experiment, the architecture will only be referenced by name.                       \\
Dataset              & The dataset used for the experiment, with its source and an overview of its details. A brief mention will be made in following experiments.                       \\
API's                & Reference to the technologies used as outlined in Chapter 2.                      \\
Script               & Snippets of the script used for the experiment.                       \\
Results              & The results acquired from the experiment, usually in the form of a percentage accuracy.                       \\
Empirical Analysis   & States any information gained from the experiment and speculation for the reasoning of the results.                      
\end{tabular}
\end{table}

\section{Udemy Tutorial}
\label{udemy1}
\subsection*{Overview}
\subsection*{Network Architecture}
\subsection*{Dataset}
\subsection*{API's}
\subsection*{Script}
\includepdf[pages=-]{tex/experiments/exp1/UdemyCNN.pdf}
\subsection*{Results}
\subsection*{Analysis}


\section{Udemy Tutorial 2}
\label{udemy2}
\subsection*{Network Architecture}
\subsection*{Dataset}
\subsection*{API's}
\subsection*{Script}
\includepdf[pages=-]{tex/experiments/exp2/InteractiveCNNModel.pdf}
\subsection*{Results}
\subsection*{Analysis}


\section{Using the Food 101 dataset}
\label{food101}
\tocless\subsection{Objective}
For the next experiment, it was decided to use the Food-101 dataset.
An on line tutorial was used to create a dataset in TensorFlow using image directories on disk \parencite{file_dir_code}.
Tutorials used previously were also utilised for this experiment such as \parencite{udemy}.

\begin{table}[h]
\centering
\caption{Using the Food-101 Dataset}
\label{my-label}
\begin{tabular}{|l|p{8cm}|}
\hline
\textbf{Network Architecture} & Similar to \ref{udemy1}            \\ \hline
\textbf{Dataset}              & Food-101 dataset \\ \hline
\textbf{APIs and Libraries}   & TensorFlow                                                         \\ \hline
\end{tabular}
\end{table}

\tocless\subsection{Script}
Figures \ref{lst:readDataset}, \ref{lst:createBatches} and \ref{lst:trainModel3} document the key aspects of the script used in this experiment.
The activities documented include reading in the dataset (Figure \ref{lst:readDataset}), create batches from the dataset (Figure \ref{lst:createBatches}) and training the model (Figure \ref{lst:trainModel3}).

\begin{figure}[h] 
\caption{Read in the Dataset \parencite{file_dir_code}}
\label{lst:readDataset}
\begin{lstlisting}[style=Python]
# Reading the dataset
def read_images(dataset_path, mode, batch_size):
    imagepaths, labels = list(), list()
    # An ID will be given to each sub-folder alphabetically
    label = 0
    classes = sorted(os.walk(dataset_path).__next__()[1])
    # List each sclass
    for c in classes:
        c_dir = os.path.join(dataset_path, c)
           walk = os.walk(c_dir).__next__()
        # Add each image to the training set
        for sample in walk[2]:
            if sample.endswith('.jpg') or sample.endswith('.jpeg'):
                imagepaths.append(os.path.join(c_dir, sample))
                labels.append(label)
        label += 1
\end{lstlisting}
\end{figure}

\begin{figure}[h]
\caption{Create Batches \parencite{file_dir_code}}
\label{lst:createBatches}
\begin{lstlisting}[style=Python]
# Convert to Tensor data strcuture
imagepaths = tf.convert_to_tensor(imagepaths, dtype=tf.string)
labels = tf.convert_to_tensor(labels, dtype=tf.int32)

# Build a queue
image, label = tf.train.slice_input_producer([imagepaths, labels],
                                             shuffle=True)

# Read images
image = tf.read_file(image)
image = tf.image.decode_jpeg(image, channels=CHANNELS)

# Resize images
image = tf.image.resize_images(image, [IMG_HEIGHT, IMG_WIDTH])

# Normalize the images
image = image * 1.0/127.5 - 1.0

# Create batches
X, Y = tf.train.batch([image, label], batch_size=batch_size,
                      capacity=batch_size * 8,
                      num_threads=4)
return X, Y
\end{lstlisting}
\end{figure}

\begin{figure}[h]
\caption{Train the Model \parencite{file_dir_code}}
\label{lst:trainModel3}
\begin{lstlisting}[style=Python]
with tf.Session() as sess:
    sess.run(init)
    tf.train.start_queue_runners()

    #For each step in range, run batch through the model
    for step in range(1, num_steps+1):
        if step % display_step == 0:
            # Run optimization and calculate batch loss and accuracy
            _, loss, acc = sess.run([train_op, loss_op, accuracy])
            print("Step " + str(step) + ", Minibatch Loss= " + \
                  "{:.4f}".format(loss) + ", Training Accuracy= " + \
                  "{:.3f}".format(acc))
        else:
            sess.run(train_op)
\end{lstlisting}
\end{figure}

\tocless\subsection{Results}
The final accuracy for this model was 18.8\% after 5,000 steps.

\tocless\subsection{Analysis}
The poor accuracy of this model is to be expected due to the simple architecture of the network.

\clearpage

\chapter{Training Using the Inception-V3 Model Architecture}
This chapter consists of experiments which train various Tensorflow models.
There are parameters that can be changed in the code that creates a Tensorflow model based on the Inception-V3 model architecture.
These parameters were changed in order to see how the model accuracy would be affected.
In addition to this, change in dataset size was explored in relation to model accuracy.


\section{Retrain ImageNet Inception V3 Model}
\label{inception}
\tocless\subsection{Objective}
For this experiment, it was decided to take inspiration from
\parencite{yanaiFood}, where pre-training was used training a model for food
classification. In order to achieve this, the final layer of the
Inception V3 model which was trained on the ImageNet dataset had to be retrained. This is called
Transfer Learning.
A tutorial, created by Google, on the TensorFlow website was followed
for direction on this process \parencite{retrainInception}.

Firstly, to retrain the final layer of a model, a dataset must be
prepared in the correct way. The Food-101 dataset \parencite{food101}
was used for this experiment, which will be analysed below.

This dataset contrasts with \parencite{yanaiFood} as they used the UECFOOD100 dataset.
The food-101 dataset was chosen due to larger number of images per class.
The UECFOOD100 dataset has 100 images per class while the food-101 dataset has 1,000 images per class.
The dataset must be structured so that
there is a separated directory for each class with the directory name as the class
name. These directories should contain all the images for this class. 

Once this dataset has been set up correctly, a directory can be found on GitHub
which contains the necessary files for this tutorial. When the directory has
been downloaded, the following command can be executed:
\begin{lstlisting}[style=Command]
python tensorflow/examples/image_retraining/retrain.py \ --image_dir
~/dataset_directory
\end{lstlisting}

The first thing that the script will do is create bottleneck files for the
images. A bottleneck is a term used to define the final layer before the output
layer. This is so that for each image, we do not have to push it through the
entire network during training \parencite{retrainInception}.

After, the bottlenecks are created, the training can be completed. The images
are split into three sub directories of training, testing and validation. By
default, these images are split into percentages of 80\%, 10\% and 10\%
respectively. The model is trained at a default of 4,000 steps. 

At the final stage of the script, the model is run on a batch of test images not
yet seen and a final test accuracy is displayed. This can be seen in the Script
section below.

The command used for using this model once it is trained is:
\begin{lstlisting}[style=Command]
python tensorflow/examples/label_image.py --graph=/tmp/output_graph.pb
--labels=/tmp/output_labels.txt --input_layer=Mul --output_layer=final_result
--input_mean=128 --input_std=128 --image=~/image_directory
\end{lstlisting}

\begin{table}[h]
\centering
\caption{Retrain ImageNet Inception V3 Model}
\label{my-label}
\begin{tabular}{|l|p{8cm}|}
\hline
\textbf{Network Architecture} & Inception-V3 architecture \parencite{inception}            \\ \hline
\textbf{Dataset}              & Food-101 dataset \\ \hline
\textbf{APIs and Libraries}   & TensorFlow and NumPy                                                        \\ \hline
\end{tabular}
\end{table}

\tocless\subsection{Script}
The following snippets of code are from the retrain.py script.

\tocless\subsubsection{Add New Layer}
\begin{lstlisting}[style=Python]
# Add new layer to the network
(train_step, cross_entropy, bottleneck_input, ground_truth_input,
final_tensor) = add_final_training_ops(
            len(image_lists.keys()), FLAGS.final_tensor_name,
            bottleneck_tensor,
            model_info['bottleneck_tensor_size'],
            model_info['quantize_layer'])
 
# Create operations to evaluate the accuracy of the model
evaluation_step, prediction = add_evaluation_step(
final_tensor, ground_truth_input)
 
# Set up weights to initial default values.
init = tf.global_variables_initializer()
sess.run(init)
\end{lstlisting}

\tocless\subsubsection{Evaluate Model}
\begin{lstlisting}[style=Python]
# Run final test evaluation
test_bottlenecks, test_ground_truth, test_filenames = (
    get_random_cached_bottlenecks(
        sess, image_lists, FLAGS.test_batch_size, 'testing',
        FLAGS.bottleneck_dir, FLAGS.image_dir, jpeg_data_tensor,
        decoded_image_tensor, resized_image_tensor, bottleneck_tensor,
        FLAGS.architecture))
test_accuracy, predictions = sess.run(
   [evaluation_step, prediction],
   feed_dict={bottleneck_input: test_bottlenecks,
        ground_truth_input: test_ground_truth})
tf.logging.info('Final test accuracy = %.1f%% (N=%d)' %
                (test_accuracy * 100, len(test_bottlenecks)))
\end{lstlisting}

\tocless\subsection{Results}
The final test Top-1 accuracy for this retrained model was 54.8\%.
These results are calculated by passing an unseen batch of test images through the model.
The 95\% confidence interval of the Top-1 accuracy of this model is between 45.1\% and 64.5\%.
This confidence interval was calculated by Equation \ref{eqn:ci}.

\begin{equation}\label{eqn:ci}
    e \pm 1.96\sqrt{e(1-e) \over n}
\end{equation}

An image of pizza Figure \ref{fig:pizza}, was fed into the model with the following results:
\begin{itemize}
    \item{pizza 0.925}
    \item{pancakes 0.008}
    \item{nachos 0.007}
    \item{beef carpaccio 0.006}
    \item{tiramisu 0.004}
\end{itemize}
In contrast, Figure \ref{fig:pizza_unclassified} was not classified as a pizza.

\begin{figure}[h] 
\centering
  \label{pizzas} 
  \begin{minipage}[h]{0.5\linewidth}
    \centering
    \includegraphics[scale=0.5]{pizza} 
    \caption{Pizza - sourced from https://www.cicis.com/} 
  \label{fig:pizza}
    \vspace{4ex}
  \end{minipage}%%
  \begin{minipage}[h]{0.5\linewidth}
    \centering
    \includegraphics[scale=0.065]{pizza_unclassified} 
    \caption{Pizza not classified correctly by the model} 
  \label{fig:pizza_unclassified}
    \vspace{4ex}
  \end{minipage} 
\end{figure}

\tocless\subsection{Analysis}
These poor results are not that surprising.
This is because we have many classes to train for, 101, and no parameter tuning has been carried out on the running of this code.
Figure \ref{fig:pizza_unclassified} was not classified correctly, this is most likely due to the fact that the pizza does not take up much of the image.



\section{Retrain with Extended Dataset}
\label{extended}
\subsection*{Overview}
For my fifth experiment, I decided to take inspiration from
\textcite{yanaiFood}, where pre-training was used training a model for food
classification. In order to achieve this, I retrained the final layer of the
Inception V3 model which was trained on the ImageNet dataset. This is called
Transfer Learning. I followed the tutorial by Google on the tensorflow website
for direction on this process \textcite{retrainInception}.

Firstly, in order to retrain the final layer of a model, a dataset must be
prepared in the correct way. I used the Food-101 dataset \textcite{food101}
which I will analyse in a later section. The dataset must be structured so that
there is a separated directory for each class with the directory name as the class
name. These directories should contain all the images for this class. 

Once this dataset has been set up correctly, a directory can be found on github
which contains the necessary files for this tutorial. When the directory has
been downloaded, the following command can be ran:
\begin{lstlisting}
python tensorflow/examples/image_retraining/retrain.py \ --image_dir
~/dataset_directory
\end{lstlisting}

The first thing that the script will do is create bottleneck files for the
images. A bottleneck is a term used to define the final layer before the output
layer. This is so that for each image, we do not have to push it through the
entire network during training \textcite{retrainInception}.

After, the bottlenecks are created, the training can be completed. The images
are split into three sub directories of training, testing and validation. By
default, these images are split into percentages of 80\%, 10\% and 10\%
respectively. The model is trained at a default of 4000 steps. 

At the final stage of the script, the model is run on a batch of test images not
yet seen and a final test accuracy is displayed. This can be seen in the Script
section below.

The command used for using this model once it is trained is:
\begin{lstlisting}
python tensorflow/examples/label_image.py --graph=/tmp/output_graph.pb
--labels=/tmp/output_labels.txt --input_layer=Mul --output_layer=final_result
--input_mean=128 --input_std=128 --image=~/image_directory
\end{lstlisting}

\subsection*{Network Architecture}
The Inception V3 model network architecture was used for this experiment. The
Inception V3 architecture was created by building on the existing Inception
model aimed at efficient image classification \textcite{rethinkingInception}.

\subsection*{Dataset}
The dataset used for this experiment is the Food-101 dataset \textcite{Food
101}. This dataset has 101 classes with 1000 images for each class.

\subsection*{Libraries}
Tensorflow and Numpy were used to run this scipt.

\subsection*{Script}
The following snippets of code are from the retrain.py script.

\begin{lstlisting}
# Add the new layer that we'll be training.
(train_step, cross_entropy, bottleneck_input, ground_truth_input,
final_tensor) = add_final_training_ops(
            len(image_lists.keys()), FLAGS.final_tensor_name,
            bottleneck_tensor,
            model_info['bottleneck_tensor_size'],
            model_info['quantize_layer'])
 
# Create the operations we need to evaluate the accuracy of our new layer.
evaluation_step, prediction = add_evaluation_step(
final_tensor, ground_truth_input)
 
# Merge all the summaries and write them out to the summaries_dir
merged = tf.summary.merge_all()
train_writer = tf.summary.FileWriter(FLAGS.summaries_dir + '/train',
                                     sess.graph)
 
validation_writer = tf.summary.FileWriter(
    FLAGS.summaries_dir + '/validation')
 
# Set up all our weights to their initial default values.
init = tf.global_variables_initializer()
sess.run(init)

\end{lstlisting}




\begin{lstlisting}
# We've completed all our training, so run a final test evaluation on
# some new images we haven't used before.
test_bottlenecks, test_ground_truth, test_filenames = (
    get_random_cached_bottlenecks(
        sess, image_lists, FLAGS.test_batch_size, 'testing',
        FLAGS.bottleneck_dir, FLAGS.image_dir, jpeg_data_tensor,
        decoded_image_tensor, resized_image_tensor, bottleneck_tensor,
        FLAGS.architecture))
test_accuracy, predictions = sess.run(
   [evaluation_step, prediction],
   feed_dict={bottleneck_input: test_bottlenecks,
        ground_truth_input: test_ground_truth})
tf.logging.info('Final test accuracy = %.1f%% (N=%d)' %
                (test_accuracy * 100, len(test_bottlenecks)))
\end{lstlisting}

\subsection*{Results}
The final test accuracy for this retrained model was 54.8\%.

For example, an image of pizza, see \ref{fig:pizza} was fed into the model with the followng results:
\begin{itemize}
    \item{pizza 0.925}
    \item{pancakes 0.008}
    \item{nachos 0.007}
    \item{beef carpaccio 0.006}
    \item{tiramisu 0.004}
\end{itemize}

\begin{figure}
     \includegraphics{pizza}
     \caption{Pizza}
     \label{fig:pizza}
\end{figure}

\subsection*{Analysis}


\section{Retrain with Parameter Tuning}
\label{parameterTuning}
\subsection*{Objective}
Within the retrain.py script \parencite{retrainInception}, as mentioned in
previous experiments, there are various parameters that can be set and changed.
Various combinations of these parameters were changed to see if it would increase the
test accuracy of the model.

\subsection*{Network Architecture}
The Inception V3 architecture was used for this model.

\subsection*{Dataset}
The Food-101+ dataset (\parencite{food101} with additional classes, as per
\ref{inception}), was used for this model.

\subsection*{Libraries}
The libraries in use for this experiment are TensorFlow and NumPy.

\subsection*{Script}
Script as seen in \ref{inception} but with some additions to calculate Top 5 accuracy as seen
below:

\begin{lstlisting}
python retrain_top5.py \ --image_dir
~/dataset_directory \ --how_many_training_steps 4000 \ --learning_rate 0.01 \
--testing_percentage 10 \ --validation_percentage 10
\end{lstlisting}

Top 5 accuracy of the model was als calcultaed using the code below.
\begin{lstlisting}[style=Python]
print('Top 5 Evaluation')
#Variables used to store a list of all classes, the amount of images tested,
# the amount of images with a top 5 accuracy and the sum of the highest probabilities
classes = list(image_lists.keys())
image_count = 0
top_5_count=0
total_of_top1_probs = 0
i = 0 
class_counter = 0

#Loops through all test images, selecting 10 images per class 
#and running them through the method in 'label_image.py'.
while(i < len(test_bottlenecks)):
  class_counter += 1
  
  if class_counter > 10:
    i = int(round((i + len(test_bottlenecks)/len(classes)) - 10))
    class_counter = 0
  else:
    image_count += 1
    results_from_classifier = label_image.runModel(test_filenames[i])
    results = results_from_classifier[0]
    probabilities = results_from_classifier[1]

    if classes[test_ground_truth[i]] in results:
      top_5_count += 1
    else:
      print("Expected:  " + classes[test_ground_truth[i]])
      for result in results:
        print("Classes: " + result)
      print("")

    total_of_top1_probs += max(probabilities)
    i = i + 1
print(str(top_5_count))
average_probabilities = total_of_top1_probs/(len(classes*10))

#Prints out the amount of test images used, the top 5 accuracy
# and the average probability of predictions.
print("Amount of test images: " + str(image_count))
print("Top 5 Accuracy: " + str((top_5_count/image_count)*100))
print("Average probability: " + str(average_probabilities))
\end{lstlisting}

Method from label\_image.py.
\begin{lstlisting}[style=Python]
def runModel(file_name):
  model_file = \
    "/tmp/output_graph.pb"
  label_file = "/tmp/output_labels.txt"
  input_height = 299
  input_width = 299
  input_mean = 128
  input_std = 128
  input_layer = "Mul"
  output_layer = "final_result"

  graph = load_graph(model_file)
  t = read_tensor_from_image_file(file_name,
                                  input_height=input_height,
                                  input_width=input_width,
                                  input_mean=input_mean,
                                  input_std=input_std)

  input_name = "import/" + input_layer
  output_name = "import/" + output_layer
  input_operation = graph.get_operation_by_name(input_name)
  output_operation = graph.get_operation_by_name(output_name)

  with tf.Session(graph=graph) as sess:
    results = sess.run(output_operation.outputs[0],
                      {input_operation.outputs[0]: t})
  results = np.squeeze(results)

  top_k = results.argsort()[-5:][::-1]
  labels = load_labels(label_file)
  
  setIndex = False

  top5_results = [None] * 5
  index = 0
  for i in top_k:
    top5_results[index] = labels[i]
    index += 1

  final_results = [top5_results, results]
  return final_results
\end{lstlisting}

Some further parameters could be set such as:
\begin{itemize}
	\item{--flip\_left\_right}
	\item{--random\_crop}
	\item{--random\_scale}
	\item{--random\_brightness}
\end{itemize}

\begin{figure}
    \includegraphics[scale=0.5]{model_test}
     \caption{Graph of accuracy of the test dataset during training}
     \label{fig:model_train_test}
\end{figure}

\begin{figure}
    \includegraphics[scale=0.5]{model_val}
     \caption{Graph of accuracy of the validation dataset during training}
     \label{fig:model_train_val}
\end{figure}

\begin{figure}
    \includegraphics[scale=0.4]{test_val_accuracy}
     \caption{Comparison of accuracy}
     \label{fig:test_val_accuracy}
\end{figure}

\begin{figure}
    \includegraphics[scale=0.4]{test_val_accuracy_refined}
     \caption{Comparison of accuracy}
     \label{fig:test_val_accuracy_refined}
\end{figure}


\begin{table}[]
	\centering
	\caption{Comparison of parameters}
	\label{parameter_tuning_table}
	\begin{tabular}{|l|l|l|l|l|l|}
  \hline
		\textbf{Parameter Tuning} & \textbf{Steps} & \textbf{Learning Rate} & \textbf{Test} \% & \textbf{Validation} \% &
		\textbf{Results} \\ \hline
		Configuration 1  & 8,000  & 0.01       & 10 & 10       &
		59.1\%  \\ \hline
		Configuration 2  & 8,000  & 0.10           & 10 & 10       &
		65.8\%  \\ \hline
		Configuration 3  & 10,000 & 0.10           & 10 & 10       &
		66.3\%  \\ \hline
		Configuration 4  & 12,000 & 0.10           & 10 & 10       &
		66.6\%  \\ \hline
		Configuration 5  & 10,000 & 0.20           & 10 & 10       &
		66.0\%  \\ \hline
		Configuration 6  & 10,000 & 0.10           & 15      & 15            &
		66.3\%  \\ \hline
	\end{tabular}
\end{table}

\subsection*{Results}
The results of each set of parameters can be seen in Table
\ref{parameter_tuning_table}.
Using the parameters of 10,000 steps, and a learning rate of 0.1 and therefore a final test accuracy of 66.3\%, the model achieved a Top 5 accuracy of 85.96\%.
This figure was calculated from 1090 images in the test dataset and the average probability of the predictions were at 0.62.


\subsection*{Analysis}
The set of parameters that seem to be the most
effective are 10000 steps with a 0.1 learning rate. Graphs of this model can be
seen in Figures \ref{fig:model_train_val} and \ref{fig:model_train_test}. These
are based on the validation set and then the test set respectively. A side by
side comparison can also be seen in Figures \ref{fig:test_val_accuracy} and
\ref{fig:test_val_accuracy_refined} where orange is for during training and blue
for the validation set.

There were two separate factors that each increased classification accuracy of
about 5\% each. These were training steps and learning rate.

Training steps are related to the number of images so before, when the training
steps were at 4000, not all of the training images were being used. As the steps were increased
twofold we saw a 3.8\% increase in accuracy.

Another parameter that increased accuracy significantly was learning rate. The
default learning rate is 0.01 which was increased to 0.1. This resulted in an
increase of 6.7\%. This is most likely since we are only
looking at the last layer, we can afford to change the weights more
significantly.

The Top 5 accuracy of the model was quite good but it was not run on the same number of images as the final test accuracy. This is beacuse TensorFlow does not have an API for calculating Top 5 accuracy. As a result, it had to be calculated manually and this is very time intensive. The average probability of 0.62 is interestingly close to the overall Top 1 accuracy.


\section{MobileNet}
\label{mobilenet}
\subsection*{Overview}
In the preious experiments, I have looked at the one-shot approach to food image
classification. That is, the model will give a prediction of the most liekly
food item in that image. This is a problem when there are multiple food in an
image. There are a few options to combat this problem. Firstly, I could detect
objects in the image, segment the image according to these objects and then run
each segment through the model. A simple approach to this would be to segment
the image into a number of sections and then run each section through the model.

\subsection*{Network Architecture}
\subsection*{Dataset}
\subsection*{API's}
\subsection*{Script}
\subsection*{Results}
\subsection*{Analysis}


\section{Food 101 subset}
\label{subset}
\subsection*{Overview}
In many of the papers that have been researched where food image classification was carried out, they attempted to classify a lot less than 108 food types as has been the case for experiments previously shown.
\parencite{novelSVM} used 12 classes, \parencite{pouladzadeh2014measuring} had 15, \parencite{LSL_2015} attempted to classify 11 classes of food, 20 classes were used in \parencite{chen2010toward} and \parencite{snap} predicted 15 classes.
Due to the lower number of classes in these papers, it was decided to retrain inception on a subset of the food-101 extended dataset to benchmark results.
13 classes were selected from food-101 for training.

\subsection*{Network Architecture}
Retrained Inception model.

\subsection*{Dataset}
A subset of the Food 101 dataset was used for this experiment \parencite{food101}.

\subsection*{Results}
A Final test Accuracy of 92.6\% was recorded for this experiment which performs quite high in comparison to the data in Table \ref{classes_accuracy}.
A Top 5 accuracy of 100\% was calculated with an average prediction probability of 0.89 using 130 images.
This was calculated using the script defined in \ref{parameterTuning}.

\begin{table}[]
\centering
\caption{Accuracy of other studies}
\label{classes_accuracy}
\begin{tabular}{|l|l|l|}
\hline
\textbf{Reference}                       & \textbf{Classes} & \textbf{Accuracy}      \\ \hline
Novel SVM                       & 12      & 92.6\%        \\ \hline
Measuring Calorie and Nutrition & 15      & 90.4\%       \\ \hline
Large Scale Learning            & 11      & 78.0\%          \\ \hline
Toward Dietary Assessment       & 20      & 91.7\% \\ \hline
Snap-n-eat                      & 15      & 85.0\%         \\ \hline
\end{tabular}
\end{table}

\subsection*{Empirical Analysis}
It would make sense the accuracy of our model would increase when the number of classes are reduced as the margin of error is decreased.
The Top 5 accuracy of the model was very successful.
There were only 10 images per class tested though, totaling at 130 images so if the number of images increased, this accuracy would probably decrease slightly.
If the code is run again, the same results cannot even be replicated for sure.
On another run of this code a Top 5 accuracy of 96\% was recorded.
While this is still a very good accuracy, it does not compare to the first run of the script.

\chapter{Analysing the Trained Model}
The purpose of this chapter is to analyse the models trained previously to gain understanding and applicability to the problem statement.

\section{Sliding Window}
\label{slidingWindow}
\subsection*{Overview}
Within the retrain.py script \textcite{retrainInception}, as mentioned in
previous experiments, there are various parameters that can be set and changed.
I tested out a few combinations of the parameters to see if I could increase the
test accuracy of the model.

\subsection*{Network Architecture}
The Inception V3 architecture was used for this model.

\subsection*{Dataset}
The Food-101 dataset \textcite{food101} with additional classes, as per
experiment 6, was used for this model.

\subsection*{Libraries}
The libraries in use for this experiment are tensorflow and numpy.

\subsection*{Script}
Script as seen in experiment 5 but with some additions to the command as seen
below:

\begin{lstlisting}
python tensorflow/examples/image_retraining/retrain.py \ --image_dir
~/dataset_directory \ --how_many_training_steps 4000 \ --learning_rate 0.01 \
--testing_percentage 10 \ --validation_percentage 10
\end{lstlisting}

Some further parameters could be set such as:
\begin{itemize}
	\item{--flip_left_right}
	\item{--random_crop}
	\item{--random_scale}
	\item{--random_brightness}
\end{itemize}

\subsection*{Results}
\begin{table}[]
	\centering
	\caption{My caption}
	\label{my-label}
	\begin{tabular}{llllll}
		Parameter Tuning & Steps & Learning Rate & Test \% & Validation \% &
		Results \\
		Configuration 1  & 8000  & default       & default & default       &
		59.1\%  \\
		Configuration 2  & 8000  & 0.1           & default & default       &
		65.8\%  \\
		Configuration 3  & 10000 & 0.1           & default & default       &
		66.3\%  \\
		Configuration 4  & 12000 & 0.1           & default & default       &
		66.6\%  \\
		Configuration 5  & 10000 & 0.2           & default & default       &
		66.0\%  \\
		Configuration 6  & 10000 & 0.1           & 15      & 15            &
		66.3\% 
	\end{tabular}
\end{table}

\subsection*{Analysis}
There were two separate factors that each increased classification accuracy of
about 5\% each. These were training steps and learning rate.

Training steps are related to the number of images so before, when the training
steps were at 4000, not all of our training images were being used. As the steps were increased
twofold we saw a 3.8\% increase in accuracy.

Another parameter that increased accuracy significantly was learning rate. The
default learning rate is 0.01 which I increased to 0.1. This resulted in an
increase of 6.7\%. This is most likely due to the fact that since we are only
looking at the last layer, we can afford to change the weights more
significantly.


\section{Recursive Refinement}
\label{RR}
\tocless\subsection{Objective}
After the sliding window code was run on Figure \ref{fig:fruit} in \ref{slidingWindow},
it was observed that a sliding window was predicting grapes correctly in
regions that contained a bunch of grapes. Since it would make sense that the
model would be able detect an individual grape, it was decided that
recursive refinement would be ran on a window that contained a grape. Due to the model
requiring a 299 x 299 image size, the window could only be refined once as
very small segments could not be resized up to 299 x 299. A window of 70 x 70 size was used.

\tocless\subsection{Script}
As you would think with recursive refinement, a recursive function would used,
but this was unnecessary due to image size restrictions. Instead, a
conditional for loop was added to the existing code.
\begin{lstlisting}[style=Python]
if top1 == "grape" and window_shape == "grid" and rr_grape:
			for (x_grape, y_grape, grape_window) in sliding_window(window_resized, stepSize=64, windowSize=(70, 70)):
				#reshape to square
				grape_window_resized = cv2.resize(grape_window, (299, 299))
				top1_grape = subSample.classify(grape_window_resized, window_shape)
				if top1_grape == "grape":
					cv2.rectangle(display_image, (x_grape + x, y_grape + y), (x_grape + x + 70, y_grape + y + 70), colour_dict.get(top1, (0,0,0)), 4)
					#cv2.imshow("Window", grape_window_resized)
					cv2.waitKey(1)
time.sleep(0.025)
\end{lstlisting}

\begin{figure}
\centering
    \includegraphics[scale=0.5]{fruit_rr}
      \caption{Recursive refinement 1}
      \label{fig:rr1}
\end{figure}

\begin{figure}
\centering
    \includegraphics[scale=0.5]{newfruit_rr_grid}
      \caption{Recursive refinement 2 - sourced from http://www.travispta.org/}
      \label{fig:rr2}
\end{figure}

\begin{figure}
\centering
    \includegraphics[scale=0.5]{processed_image}
      \caption{Recursive refinement 3}
      \label{fig:rr3}
\end{figure}


\tocless\subsection{Results}
Some very similar results were recorded on three separate images. Two new images are seen
here which we will explore in future experiments. In all three images, while the script
is calculating some expected predictions, grapes are being classified in locations
that have nothing resembling a grape. These can be viewed in Figures
\ref{fig:rr1}, \ref{fig:rr2} and \ref{fig:rr3}.

\tocless\subsection{Analysis}
The instances where false positives were recorded may indicate issues with the recursive refinement approach as every window must be classified and it is possible that this just happens to be a grape in some instances.
Further analysis to why this may be occurring is outlined in the section \ref{rrAnalyse}.


\section{Impact of Background}
\label{background}
\tocless\subsection{Objective}
As we can see in section \ref{slidingWindow}, using sliding windows to classify many sections
of an image, there were some cases where some unexpected predictions were made.
Due to this, the decision was made to analyse the effect the background of the
image on its classification. The sliding window code was then ran on a new
image. This new image was the same fruit bowl as used previously but the
background was filled in as white as per Figure \ref{fig:filledFruit}.

\begin{figure}[h]
\centering
    \includegraphics[scale=0.75]{fruitFillBg}
    \caption{Bowl of fruit with background removed}
    \label{fig:filledFruit}
\end{figure}

\begin{table}[h]
    \centering
    \caption{Comparison of fruit image sliding window results with and without
    background}
    \label{comparisionFruitTable}
    \begin{tabular}{|l|l|l|l|p{1.25cm}|p{1.25cm}|p{2cm}|}
    \hline
        \textbf{Food type} & \textbf{Grid} & \textbf{Row} & \textbf{Column} & \textbf{White Grid} & \textbf{White Row} & \textbf{White Column} \\ \hline
        Apple     & 5    & 1   & 3      & 10          & 0          & 4
        \\ \hline
        Banana    & 1    & 0   & 1      & 0           & 0          & 0
        \\ \hline
        Grape     & 4    & 0   & 0      & 1           & 0          & 1
        \\ \hline
        Orange    & 5    & 0   & 0      & 3           & 0          & 0
        \\ \hline
        Other     & 0    & 3   & 1      & 1           & 4          & 0  \\ \hline           
    \end{tabular}
\end{table}

\tocless\subsection{Results}
\tocless\subsubsection{Grid}
For the grid-based sliding window approach, the results turned out to be less
successful than with the background. In this experiment, fourteen out of fifteen Top-1 classifications were of an expected food type rather than fifteen out of
fifteen with the background present. We expected the food types of apple,
orange, grape and banana to appear in this image but while a banana was detected
to a Top-5 accuracy on a few occasions it was never predicted to a Top-1
accuracy. The contrast between the image results can be seen in Table
\ref{comparisionFruitTable}.

\tocless\subsubsection{Row}
The row-based sliding window again had worse results than its counterpart, with
zero out of four correct classifications as opposed to one. In this case, an
orange appeared at Top-5 accuracy once. The most common prediction was ice-cream
which appeared at Top-1 accuracy in three out of four instances.

\tocless\subsubsection{Column}
In contrast to our previous two methods of sliding window, this method
outperformed its counterpart with correct predictions of all five windows while
before we only had four out of five. In this experiment, an apple was predicted
four times and a grape once, with all correct fruits appearing to Top-5
accuracy.

\tocless\subsection{Analysis}
Surprisingly, removing the background of the image reduced the prediction
accuracy.
Many white foods were classified instead which makes sense due
the impact of colour expected.


\section{Alternative Test Image}
\label{alternative}
\tocless\subsection{Objective}
In our previous sliding window-oriented experiments, we had only used a
single image. In order to see whether this image had biases unknown
to us, another fruit bowl image had to be tested. This image was selected as
fruit took up a larger portion of the image as seen in Figure \ref{fig:newFruit}.

\begin{figure}[h]
\centering
    \includegraphics[scale=0.4]{fruitbowl2}
    \caption{Alternative Bowl of fruit}
    \label{fig:newFruit}
\end{figure}

\begin{table}[]
    \centering
    \caption{Comparison of fruit bowl images}
    \label{newFruitTable}
    \begin{tabular}{|l|l|l|l|p{1.25cm}|p{1.25cm}|p{2cm}|}
    \hline
        \textbf{Food type} & \textbf{Grid} & \textbf{Row} & \textbf{Column} & \textbf{New Grid} & \textbf{New Row} & \textbf{New Column} \\ \hline
        Apple     & 5    & 1   & 3      & 4        & 1       & 0          \\ \hline
        Banana    & 1    & 0   & 1      & 5        & 0       & 5          \\ \hline
        Grape     & 4    & 0   & 0      & 2        & 1       & 0          \\ \hline
        Orange    & 5    & 0   & 0      & 0        & 0       & 0          \\ \hline
        Other     & 0    & 3   & 1      & 1        & 2       & 1         \\ \hline
    \end{tabular}
\end{table}

\tocless\subsection{Results}
\tocless\subsubsection{Grid}
The performance of this experiment was slightly worse than with the previously
used image. When the grid based sliding window was executed on Figure
\ref{fig:newFruit}, fourteen out of fifteen predictions had an expected value.
Out of the fourteen predictions orange was not predicted to Top-1 accuracy at
all. This can be seen, in comparison to the previously used image, in Table
\ref{newFruitTable}.

\tocless\subsubsection{Row}
In the row based window for the new fruit image, the results were not very
successful as has been the trend for most row based classification. Two out of
four predictions had an expected value at Top-1 accuracy.

\tocless\subsubsection{Column}
The column based approach had a similar result to its counterpart in that only
one of its predictions was unexpected. Although, due  to the size of the new
image, another column was created and thus has a better overall accuracy.

\tocless\subsection{Analysis}
A possible reason that an orange was not classified in any of these images is
because in Figure \ref{fig:newFruit}, a mandarin is displayed.


\section{Analysing Results of Recursive Refinement Further}
\label{rrAnalyse}
\subsection*{Overview}
In \ref{RR}, some interesting results were obtained when attempting to run recursive refinement on an image with grapes in it.
Originally it was meant to see if individual grapes could be recognised.
On evaluation of the images used for training, it was found that bunches of grapes were used for training, not individual grapes. Therefore the results expected where never feasibly going to be obtained.
Even though this was the case, the script resulted in finding grapes in portions of the background, as seen in Figure \ref{fig:rr1}.
The purpose of this experiment is to look into why this is occurring.
A new image of a fruit bowl (Figure \ref{fig:fruitFromPhone}), taken on a mobile phone was selected for this experiment.
The image was used to run the script as defined in \ref{RR} but on two separate models.
The model with tuned parameters, trained on 108 classes, with a final test accuracy of 66.3\% and the model trained on 13 classes, with a final accuracy of 92.6\%.

\begin{figure}[h]
\centering
    \includegraphics[scale=0.08]{fruitFromPhone}
      \caption{Fruit image taken on a mobile phone}
      \label{fig:fruitFromPhone}
\end{figure}

\subsection*{Results}
The outputs of the script can be seen in Figure \ref{fig:fruitRR13} and Figure \ref{fig:fruitRR108}. The output using the model trained on 13 classes had less false positive predictions. As the script ran (on both models), the segments predicted as grapes were saved to disk and then were each classified using 'label\_image.py'. The probabilities of the predictions were recorded with results as seen in Table \ref{RR_result_table}.

\begin{figure}[h]
\centering
    \includegraphics[scale=0.5]{fruitFromPhoneRR_13Classes}
      \caption{Image after sliding window - 13 classes}
      \label{fig:fruitRR13}
\end{figure}

\begin{figure}[h]
	\centering
    \includegraphics[scale=0.5]{fruitFromPhoneRR_108Classes}
      \caption{Image after sliding window - 108 classes}
      \label{fig:fruitRR108}
\end{figure}

\begin{table}[]
\centering
\caption{Results of recursive refinement segment classifications using two models}
\label{RR_result_table}
\begin{tabular}{|l|l|l|}
\hline
Fruitbowl Segment & 13 class model                                                                                                                                       & 108 class model                                                                                                                                   \\ \hline
False Positive 1                 & \begin{tabular}[c]{@{}l@{}}grape 0.31476045\\ apple 0.25866747\\ orange 0.11228518\\ apple pie 0.10234257\\ chocolate cake 0.09062083\end{tabular}   & \begin{tabular}[c]{@{}l@{}}grape 0.31820437\\ panna cotta 0.091400646\\ macarons 0.08560496\\ roll 0.08188102\\ apple 0.03981942\end{tabular}     \\ \hline
False Positive 2                 & \begin{tabular}[c]{@{}l@{}}grape 0.29391274\\ apple 0.2371384\\ orange 0.16196558\\ chocolate cake 0.10039616\\ apple pie 0.096121624\end{tabular}   & \begin{tabular}[c]{@{}l@{}}panna cotta 0.21783036\\ grape 0.105718106\\ apple 0.087030284\\ macarons 0.07753955\\ orange 0.037592527\end{tabular} \\ \hline
False Positive 3                 & \begin{tabular}[c]{@{}l@{}}grape 0.22402291\\ chocolate cake 0.16412543\\ apple 0.14523567\\ orange 0.12525927\\ apple pie 0.099948086\end{tabular}  & \begin{tabular}[c]{@{}l@{}}grape 0.2835378\\ panna cotta 0.088551655\\ macarons 0.06166248\\ orange 0.041896477\\ roll 0.040348493\end{tabular}   \\ \hline
False Positive 4                 & \begin{tabular}[c]{@{}l@{}}apple 0.37422368\\ grape 0.28592423\\ orange 0.10230055\\ chocolate cake 0.085367255\\ apple pie 0.053731006\end{tabular} & \begin{tabular}[c]{@{}l@{}}grape 0.20476209\\ macarons 0.17529227\\ panna cotta 0.16139281\\ roll 0.04359601\\ orange 0.036991216\end{tabular}    \\ \hline
False Positive 5                 & N/A                                                                                                                                                  & \begin{tabular}[c]{@{}l@{}}grape 0.22574392\\ macarons 0.14840269\\ panna cotta 0.08670294\\ roll 0.08650105\\ orange 0.046900786\end{tabular}    \\ \hline
False Positive 6                 & N/A                                                                                                                                                  & \begin{tabular}[c]{@{}l@{}}grape 0.31369162\\ panna cotta 0.11833602\\ macarons 0.09010604\\ apple 0.057512555\\ roll 0.047116004\end{tabular}    \\ \hline
\end{tabular}
\end{table}

\subsection*{Empirical Analysis}
\subsubsection*{Comparing the two models}
As we can see evidence of in Table \ref{RR_result_table}, along with Figure \ref{fig:fruitRR13} and Figure \ref{fig:fruitRR108}, the model trained on 13 classes predicted only four false positives while the larger model predicted 6.
This is most likely due to the larger model not being able to separate grapes as well as the smaller model.
It is worth mentioning that the grape class has the lowest number of training images of all the other classes.

\subsubsection*{Analysing Probabilities}
The probabilities of predictions for grapes in these false positive classifications are all low, in both models.
In contrast, some of the correctly classified segments were run through the model and the results all had probabilities in the high nineties.
When these background segments are run through the model, some prediction has to be made and none of these false predictions have a probability of over 0.4.
The average probability of these segments being grapes is in fact 0.257 (rounded to three decimal places).
A counter measure to this problem could be to disregard any predictions with a probability under a certain threshold, perhaps 0.4.



\section{Scaling Down Images}
\label{scale}
\begin{figure}
    \includegraphics[scale=0.5]{fruitbowl2}
    \caption{Alternative Bowl of fruit}
    \label{fig:newFruit}
\end{figure}

\subsection*{Overview}
In our three previous sliding window oriented experiments, we had only used a
single image. In order to see whether this image had biases unknown
to us, I decided to use another fruit bowl image. This image was selected as
fruit took up a larger portion of the image as seen in Figure \ref{fig:newFruit}

\subsection*{Results}
\subsubsection*{Grid}
\subsubsection*{Row}
\subsubsection*{Column}

\subsection*{Analysis}


\section{Effect of Colour}
\label{colour}
\tocless\subsection{Objective}
As seen in the last experiment, it is possible that colour plays a significant part in the overall classification of some food types, along with shape and texture. Due to this, what would happen if colour was removed from the image? The three images, Figure \ref{fig:bananaPreRes}, Figure \ref{fig:apple_piePreRes} and Figure \ref{fig:pizzaPreRes}, were all converted to greyscale to test this and the resulting images can be seen in Figure \ref{fig:bananaGrey}, Figure \ref{fig:applePieGrey} and Figure \ref{fig:pizzaGrey}. Each of these images were ran through the model before and after converting to greyscale and the results recorded.

\begin{figure}
    \includegraphics[scale=0.08]{banana_grey}
    \caption{Banana Image Greyscale}
    \label{fig:bananaGrey}
\end{figure}

\begin{figure}
    \includegraphics[scale=0.4]{pie_grey}
    \caption{Apple Pie Image Greyscale}
    \label{fig:applePieGrey}
\end{figure}

\begin{figure}
    \includegraphics[scale=0.08]{pizza_grey}
    \caption{Pizza Image Greyscale}
    \label{fig:pizzaGrey}
\end{figure}

\tocless\subsection{Network Architecture}
Inception V-3 architecture. Parameter tuned model was used for this experiment.

\tocless\subsection{Dataset}
Food-101 extended dataset.

\tocless\subsection{Results}
The results for this experiment can be seen in Table \ref{colour}.
The 'Top 1' notation in the table suggests that the food was predicted as the number one prediction. The decimal value is the probability of that food type being the correct classification.

\begin{table}[]
\centering
\caption{Effect of Colour}
\label{colour}
\begin{tabular}{|l|l|l|}
\hline
\textbf{Food Image} & \textbf{Pre-Scaled Image} & \textbf{Greyscale}      \\ \hline
Banana     & Top 1 : 0.9971   & Top 1 : 0.9986 \\ \hline
Apple Pie  & Top 1 : 0.7986   & Top 1 : 0.3462   \\ \hline
Pizza      & Top 1 : 0.9753   & Top 2 : 0.1299 \\ \hline
\end{tabular}
\end{table}

\tocless\subsection{Analysis}
\tocless\subsubsection{Colour}
In regards to the images of the banana (Figure \ref{fig:bananaPreRes} and Figure \ref{fig:bananaGrey}), there is very little difference in classification. They were both classified to a Top 1 accuracy and their probabilities differing by only 0.0015 with the grey scale image being of a higher probability. This would indicate that colour is not an important factor for classifying bananas and the coloured image may even have noise. This can be seen in Table \ref{colour}.

The apple pie images (Figure \ref{fig:apple_piePreRes} and Figure \ref{fig:applePieGrey}) were also both classified correctly but with a large gap in the likelihood of that classification being correct. As per Table \ref{colour}, the coloured image had a probability of 0.7986 as opposed to one of 0.3462. This would indicate that while colour is important, there is enough unique data from the rest of the image to result in correct classification.

Finally, the pizza images (Figure \ref{fig:pizzaPreRes} and Figure \ref{fig:pizzaGrey})were the most contrasting in their results. While the original image was classified to Top 1 accuracy with a likelihood of 0.9753, the coloured image was classified to Top 2 accuracy while a probability of 0.1299. From this we can deduce that colour is vital fro the classification of pizza.

Since this experiment only compared three images, we cannot say for sure if this analysis is biased or not.

\tocless\subsubsection{Colour vs Scale}
In the last experiment, it was seen that the pizza and banana images (Figure \ref{fig:pizzaPreRes} and Figure \ref{fig:bananaPreRes}) were not really affected by image quality. While this seems to be true, the image of apple pie, Figure \ref{fig:apple_piePreRes}, was greatly influenced by only being classified to a Top 5 accuracy when down scaled.

For the greyscale images, bananas were not effected greatly and neither was an apple pie but a pizza was greatly influenced.

From this, it is clear that the prominent unique features for each of these food types are different. The banana (Figure \ref{fig:bananaPreRes}) may have focus on shape and texture, the pie (Figure \ref{fig:apple_piePreRes}) on image quality and some influence of colour while the pizza (Figure \ref{fig:pizzaPreRes}) may have a focus on shape, texture and colour.



\section{Visualising Images Through the Network}
\label{visualise}
\tocless\subsection{Overview}
\tocless\subsection{Network Architecture}
\tocless\subsection{Dataset}
\tocless\subsection{API's}
\tocless\subsection{Script}
\tocless\subsection{Results}
\tocless\subsection{Empirical Analysis}


